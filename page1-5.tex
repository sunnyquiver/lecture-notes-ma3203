\section{Quivers}
\subsection{Quivers, vertices, arrows and paths}
\begin{defin}
A \emph{quiver}\index{quiver} $\G = (\G_0,\G_1)$ is an oriented graph,
\begin{align}
\G_0 & = \{\textit{vertices}\index{quiver!vertices}\}  (= \{1,2,\ldots,n\}).\notag\\
\G_1 & = \{\textit{arrows}\index{quiver!arrows}\}.\notag
\end{align}
\end{defin}
We always assume that $\G_0$ and $\G_1$ are finite sets.

\begin{exam}
$\G\colon \xymatrix{1\ar[r]^\alpha & 2}$, $\G_0=\{1,2\}$ and $\G_1=\{\alpha\}$. 
\end{exam}

\begin{exam}
$\G\colon \xymatrix{1\ar@(ur,dr)[]^\alpha}$, $\G_0=\{1\}$ and $\G_1=\{\alpha\}$. 
\end{exam}

\begin{exam} $\G\colon
  \xymatrix{1\ar@/^.3pc/[rr]^\alpha\ar@/_.3pc/[rr]_\beta & & 2\ar@(ur,dr)[]^\gamma\ar@/_0.3pc/[dl]_\delta\\
    & 3\ar[ul]^\theta\ar@/_0.3pc/[ur]_\epsilon & }$, $\G_0=\{1,2,3\}$ and $\G_1=\{\alpha,\beta,\gamma,\delta,\epsilon, \theta\}$.  
\end{exam}
Have maps: $\mathfrak{s}, \mathfrak{e}\colon \G_1\to \G_0$
\begin{align}
\mathfrak{s}(\alpha) & = \text{the vertex where $\alpha\in\G_1$ starts,}\notag\index{$\mathfrak{s}$}\\
\mathfrak{e}(\alpha) & = \text{the vertex where $\alpha\in\G_1$ ends.}\notag\index{$\mathfrak{e}$}
\end{align}

\begin{defin}
$\G=(\G_0,\G_1)$ quiver.  A \emph{path}\index{path} in $\G$ is either
\begin{enumerate}[\rm(i)]
\item an ordered sequence of arrows $p=\alpha_n\alpha_{n-1}\cdots\alpha_1$, where 
\[\mathfrak{e}(\alpha_t) = \mathfrak{s}(\alpha_{t+1})\]
for $t = 1,2,\ldots,n-1$ (\emph{non-trivial path})\index{path!non-trivial} or
\item $e_i$ for each $i$ in $\G_0$ (\emph{trivial path})\index{path!trivial}.
\end{enumerate}
In addition,
\begin{align}
\mathfrak{s}(p) & = \mathfrak{s}(\alpha_1)  & \mathfrak{s}(e_i) = i\notag\\
\mathfrak{e}(p) & = \mathfrak{e}(\alpha_n)  & \mathfrak{e}(e_i) = i\notag
\end{align}
\end{defin}

\begin{exam}\label{exam:1.6}
$\G\colon \xymatrix{ 1\ar[r]^\alpha & 2 \ar[r]^\beta\ar[d]^\gamma & 3\\
& 4 & }$

Paths: 
\begin{enumerate}[\rm(i)]
\item $\alpha, \beta, \gamma, \beta\alpha, \gamma\alpha$.
\item $e_1$, $e_2$, $e_3$, $e_4$.
\end{enumerate}
\end{exam}

\begin{exam}
$\G\colon \xymatrix{1\ar@(ur,dr)[]^\alpha}$.

Paths: 
\begin{enumerate}[\rm(i)]
\item $\alpha, \alpha^2 = \alpha\alpha, \alpha^3 = \alpha\alpha\alpha, \ldots$.
\item $e_1$.
\end{enumerate}
\end{exam}

\subsection{Path algebras}

Given $\G=(\G_0,\G_1)$, a quiver, and $k$ a field.

The \emph{path algebra}\index{path algebra} $k\G$:  $k\G$ is the
vector space with all the paths in $\G$ as a basis. 

The elements in $k\G$:
\[a_1p_1 + a_2p_2 + \cdots + a_tp_t\]
where $a_i\in k$ and $p_i$ are paths in $\G$.  We write just $p$ for
$1p$, when $p$ is a path in $\G$. 

\begin{exam}
Continuing \hyperref[exam:1.6]{Example \ref*{exam:1.6}}:
\begin{align}
x & = a_1e_1 + a_2e_2 + a_3e_3 + a_4e_4 + a_5\alpha + a_6\beta +
a_7\gamma + a_8\beta\alpha + a_9\gamma\alpha\notag\\
y & = b_1e_1 + b_2e_2 + b_3e_3 + b_4e_4 + b_5\alpha + b_6\beta +
b_7\gamma + b_8\beta\alpha + b_9\gamma\alpha\notag
\end{align}
\begin{multline}
x + y = (a_1 + b_1)e_1 + (a_2 + b_2)e_2 + (a_3 + b_3)e_3 + (a_4 +
b_4)e_4 + (a_5 + b_5)\alpha\notag\\ 
 + (a_6 + b_6)\beta + (a_7 + b_7)\gamma
+ (a_8 + b_8)\beta\alpha + (a_9 + b_9)\gamma\alpha\notag
\end{multline}
\end{exam}

\subsubsection*{Multiplication} $p$, $q$ paths in $\G$:
\begin{enumerate}[\rm(1)]
\item $p$, $q$ both non-trivial 
\[p\cdot q = \begin{cases} pq, & \text{\ if $\mathfrak{e}(q) =\mathfrak{s}(p)$}\notag\\
0, & \text{\ otherwise}\notag
\end{cases}\]
\item $p$ non-trivial, $q$ trivial, $q = e_i$
\[p\cdot q = \begin{cases} p, & \text{\ if $\mathfrak{s}(p) = i =\mathfrak{e}(q)$}\notag\\
0, & \text{\ otherwise}\notag
\end{cases}\]
\[q\cdot p = \begin{cases} p, & \text{\ if $\mathfrak{e}(p) = i = \mathfrak{s}(q)$}\notag\\
0, & \text{\ otherwise}\notag
\end{cases}\]
\item $p = e_i$, $q = e_j$ (both trivial)
\[p\cdot q = \begin{cases} e_i, & \text{\ if $\mathfrak{e}(q) = j = i
    = \mathfrak{s}(p)$}\notag\\
0, & \text{\ otherwise}\notag
\end{cases}\]
\end{enumerate}
 This is extended distributively to an operation on $k\G$ (see
 \cite[page 50]{ARS}).

\begin{exam}
$\Gamma\colon \xymatrix{1\ar[r]^\alpha & 2}$, $k$ field.  

Elements in $k\Gamma$:  $a_1e_1 + a_2e_2 + a_3\alpha = y$. 

\[\begin{array}{c||c|c|c}
      & e_1 & e_2 & \alpha \\ \hline\hline 
e_1 & e_1 &  0   &    0   \\ \hline
e_2 &   0   & e_2 &  \alpha \\ \hline
\alpha & \alpha & 0 & 0
\end{array}\]
\end{exam}
\begin{align}
(e_1 + e_2)\cdot y & = (e_1 + e_2)(a_1e_1 + a_2e_2 +
                     a_3\alpha)\notag\\
& = a_1e_1^2 + a_2\underbrace{e_1e_2}_{=0} + a_3\underbrace{e_1\alpha}_{=0} + a_1\underbrace{e_2e_1}_{=0} + a_2e_2^2 + a_3
  e_2\alpha\notag\\
& = a_1e_1 + a_2e_2 + a_3\alpha = y\notag
\end{align} 
Similarly $y\cdot (e_1 + e_2) = y$.  Hence,  $e_1 + e_2$ acts like $1$
in $k\Gamma$. 

Basis for $k\Gamma$: $\{ e_1, e_2, \alpha\}$, $\dim_kk\Gamma = 3$.

\begin{exam}
$\Gamma\colon \xymatrix{1\ar@(ur,dr)[]^\alpha}$, and $k$ a field.

$k\Gamma$ has basis: $\{e_1, \alpha, \alpha^2, \alpha^3, \ldots\}$,
that is, $\dim_k k\Gamma = \infty$. 

Elements in $k\Gamma$: $a_0e_1 + a_1\alpha + a\alpha^2 + \cdots +
a_t\alpha^t$, with $a_i$ in $k$ and $t\geqslant 0$.  
\end{exam}

\begin{note}
\begin{enumerate}
\item In general, $\{e_i\}_{i \in \G}$ are \emph{orthogonal
    idempotents}\index{idempotents!orthogonal} in $k\Gamma$, that is, 
	\[ \begin{cases}\text{\ $e_i^2 = e_i$}\notag\\
	\text{$e_ie_j = 0$ for $i\neq j$}\notag
	\end{cases}\]
	
    \item Suppose $\G_0 = \{1,2,\ldots,n\}$. Then $e_1 + e_2 + \cdots + e_n$
      acts like 1 in $k\G$. Enough to show that
      \[p = (e_1 + e_2 + \cdots + e_n)p = p(e_1 + e_2 + \cdots +
        e_n)\] 
      for any path $p$. Suppose that $\mathfrak{s}(p) = i$ and
      $\mathfrak{e}(p) = j$. Then 
      \[(e_1 + e_2 + \cdots + e_n)p = e_1p + e_2p + \cdots + e_jp + \cdots +
      e_np = e_jp \defeq p\]
      \[p(e_1 + e_2 + \cdots + e_n) = pe_1 + pe_2 + \cdots + pe_i + \cdots +
      pe_n = e_jp \defeq p\]
	
    $\implies e_1 + e_2 + \cdots + e_n = 1_{k\G}$ = identity in $k\G$
\end{enumerate}
Can show: $k\G$ is a $k$-algebra with $e_1 + e_2 + \cdots + e_n$ as an
identity (see \cite[page 50]{ARS}). 
\end{note}

Recall: $\Lambda$ ring, $k$ field. 
\begin{defin}
  $\Lambda$ is a $k$-\emph{algebra}\index{algebra}, if $\Lambda$ is a
  vector space over $k$ ($\xymatrix{k\times\Lambda \ar[r] & \Lambda}$,
  $\Lambda$ is a module over $k$,
  $\alpha \in k, \lambda \in \Lambda, \alpha\cdot\lambda$) and
\[\alpha(\lambda\cdot\lambda^{'})=(\alpha\cdot\lambda)\cdot\lambda^{'}
  = \lambda(\alpha\cdot\lambda^{'})\]
  $\forall \alpha \in k, \forall \lambda,
  \lambda'\in\Lambda$. 
	
\subsection*{Equivalent}: $\Lambda$ is a $k$-algebra, if
  $\exists\ \phi\colon k \to \Lambda$ a ring homomorphism such that
 \[\Image\phi \subseteq Z(\Lambda)= \{z\in \Lambda \mid 
  z\lambda=\lambda z,\forall \lambda \in \Lambda\}\]
  ($\iff \exists R \subseteq \Lambda$ subring such that $R \simeq k$
  with $R \subseteq Z(\Lambda)$, just define $\phi (a)=a \cdot
  1_{\Lambda}$). 

  For $ k\G $ the ring homomorphism $\phi \colon k \rightarrow k\G $
  is given by $\phi(a) = ae_1 + ae_2 + \cdots + ae_n$
\end{defin}
\begin{exer}
\begin{enumerate}
\item $\Gamma\colon \xymatrix{1\ar[r]^\alpha & 2}$, $k$ field. 

Find a $k$-algebra isomorphism
\[\psi\colon k\G \rightarrow \begin{pmatrix}
    k & 0\\
    k & k
  \end{pmatrix}.\]
\item $\G\colon \xymatrix{1\ar@(ur,dr)[]^\alpha}$,  $k$ field.

Show that $k\G\simeq k[x]$ as $k$-algebras.
\end{enumerate}
\end{exer}

\begin{defin}
  A non-trivial path $p$ in $\G$ is an \emph{oriented
    cycle}\index{oriented cycle} if
  \[\mathfrak{e}(p) = \mathfrak{s}(p).\]
\end{defin}

\begin{exam}
  $\G\colon \xymatrix{\ar@(ul,dl)[]_\alpha 1 \ar@/^.3pc/[r]^\beta &
    \ar@/^.3pc/[l]^\gamma 2}$\\ \newline 

\underline{Cycles}:
  $\alpha,\alpha^3,\gamma\beta\alpha,\beta\alpha^{10}\gamma, \ldots$
  $\dim_k k\G = \infty$
\end{exam}

\begin{prop}\label{prop1}
  $\G = (\G_0,\G_1)$ quiver, $k$ field.
\begin{center}
  $\dim_kk\G < \infty$ if and only $\G$ has no oriented cycles.
\end{center}
\end{prop}
\begin{proof}
  Exercise.
\end{proof}

\begin{prop}
  Assume that $\G=(\G_0,\G_1)$ has no oriented cycles. 
\begin{center}
$k\G$ is semisimple $\iff \G_1 = \emptyset$.
\end{center}
\end{prop}

\begin{proof}
  Proposition \ref{prop1} $ \implies \text{dim}_kk\G < \infty \implies k\G$ is a left artinian ring. \\
  $k\G$ semisimple $\iff $ no non-zero nilpotent left ideals in $k\G$.\\

  $\Mapsto$: Assume that $\G_1\neq\emptyset.$ Let $\alpha_1$ be an
  arrow in $\G$. Want to find a vertex where at least one arrow ends
  and no arrow starts. If
\[\mathfrak{e}(\alpha_1)\] 
is such a vertex, we are done. If not, there is an arrow $\alpha_2$
starting in
\[\mathfrak{e}(\alpha_1).\] 
If also
\[\mathfrak{e}(\alpha_2)\] 
is not as above, we continue. Since $\G$ has no oriented cycles and
$\G$ is finite, we must end up in a vertex $ v$, where arrows only end
and no arrows starts. Say, $\alpha = \alpha_t$ is an arrow ending in
$v$. Then consider $k\G\alpha = k\alpha$.  Since
$(a_1\alpha)(a_2\alpha = (a_1a_2)\underbrace{(\alpha\alpha)}_{=0} =
0$ $\implies (k\G\alpha)^2=(0)$ and $k\G\alpha \neq (0)$, we infer
that $k\G$ is not semisimple.

$\Mapsfrom$: Assume that $\G_1 = \emptyset$. Then 
\[\G\colon \xymatrix{1 & 2\ar@{..}[rr] & &  n} \text{ ($n$ vertices)}\]
  Basis for $k\G\colon \{e_1, e_2, \cdots, e_n\}$. Elements in
  $k\G\colon a_1e_1 + a_2e_2 + \cdots + a_ne_n$ with $a_i \in k$. Have
  a ring homomorphism
\[\psi\colon \underbrace{k \times \cdots \times k}_{n}  \rightarrow k\G\]
  given by
\[\psi(a_1,a_2,\ldots,a_n)=a_1e_1 + a_2e_2+ \cdots a_ne_n\] 
(check this!).  Show that $\psi$ is an isomorphism. Therefore $k\G$ is
  semisimple, since $k\G$ is isomorphic to a finite product of full
  matrix rings over divisjon rings.
\end{proof}

Note: $k\G$ is not always semisimple, but some factor of $k\G$ is.

\begin{prop}
  $\G=(\G_0,\G_1) $ quiver, $k$ field. Let
\[J = \{\text{all linear combinations of non-trivial paths}\}.\] 
Then $J$ is an ideal in $k\G$ and
  $k\G/J \simeq \underbrace{k \times \cdots \times k}_{|\G_0|}$,
  -semisimple
\end{prop}
\begin{proof}[Sketch of proof]
  Define $\psi\colon k\G \rightarrow \underbrace{k \times \cdots \times}_{|\G_0|=n} = k^n $
\[\psi(a_1e_1 + a_2e_2 + \cdots + a_ne_n + \text{ linear combinations
    of non-trivial paths} ) = (a_1, a_2, \cdots, a_n)\]
  \underline{Check:} \begin{enumerate}
		\item  $\psi$ is well-defined
		\item $\psi$ homomorphism of rings
		\item ker$\psi=J$
	\end{enumerate}
	$\implies k\G/J \simeq$ Im$\psi=k^n$.
\end{proof}
