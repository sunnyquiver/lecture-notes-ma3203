\documentclass{amsart}
\usepackage{amsmath, amsthm, amscd, amssymb, latexsym}
\usepackage[all]{xy}
\usepackage{imakeidx,enumerate}

\numberwithin{equation}{section}
\newtheorem{thm}{Theorem}[section]
\newtheorem{cor}[thm]{Corollary}
\newtheorem{lem}[thm]{Lemma}
\newtheorem{prop}[thm]{Proposition}
\theoremstyle{definition}
\newtheorem{defin}[thm]{Definition}
\newtheorem{rem}[thm]{Remark}
\newtheorem{exam}[thm]{Example}

\newcommand{\G}{\Gamma}
\renewcommand{\L}{\Lambda}

\usepackage{hyperref}
\makeindex

\begin{document}
\title{Lecture notes for MA3203 Ring theory}

\author{\O yvind Solberg}
\address{Department of Mathematical Sciences\\
NTNU\\
N-7491 Trondheim, Norway}
\email{oyvind.solberg@math.ntnu.no}

\maketitle
\tableofcontents

\section{Quivers}
\subsection{Quivers, vertices, arrows and paths}
\begin{defin}
A \emph{quiver}\index{quiver} $\G = (\G_0,\G_1)$ is an oriented graph,
\begin{align}
\G_0 & = \{\text{vertices}\index{quiver!vertices}\}  (= \{1,2,\ldots,n\}).\notag\\
\G_1 & = \{\text{arrows}\index{quiver!arrows}\}.\notag
\end{align}
\end{defin}
We always assume that $\G_0$ and $\G_1$ are finite sets.

\begin{exam}
$\G\colon \xymatrix{1\ar[r]^\alpha & 2}$, $\G_0=\{1,2\}$ and $\G_1=\{\alpha\}$. 
\end{exam}

\begin{exam}
$\G\colon \xymatrix{1\ar@(ur,dr)[]^\alpha}$, $\G_0=\{1\}$ and $\G_1=\{\alpha\}$. 
\end{exam}

\begin{exam} $\G\colon
  \xymatrix{1\ar@/^.3pc/[rr]^\alpha\ar@/_.3pc/[rr]_\beta & & 2\ar@(ur,dr)[]^\gamma\ar@/_0.3pc/[dl]_\delta\\
    & 3\ar[ul]^\theta\ar@/_0.3pc/[ur]_\epsilon & }$, $\G_0=\{1,2,3\}$ and $\G_1=\{\alpha,\beta,\gamma,\delta,\epsilon, \theta\}$.  
\end{exam}
Have maps: $s, e\colon \G_1\to \G_0$
\begin{align}
\mathfrak{s}(\alpha) & = \text{the vertex where $\alpha\in\G_1$ starts,}\notag\index{$\mathfrak{s}$}\\
\mathfrak{e}(\alpha) & = \text{the vertex where $\alpha\in\G_1$ ends.}\notag\index{$\mathfrak{e}$}
\end{align}

\begin{defin}
$\G=(\G_0,\G_1)$ quiver.  A \emph{path}\index{path} in $\G$ is either
\begin{enumerate}[\rm(i)]
\item an ordered sequence of arrows $p=\alpha_n\alpha_{n-1}\cdots\alpha_1$, where 
\[\mathfrak{e}(\alpha_t) = \mathfrak{s}(\alpha_{t+1})\]
for $t = 1,2,\ldots,n-1$ (\emph{non-trivial path})\index{path!non-trivial} or
\item $e_i$ for each $i$ in $\G_0$ (\emph{trivial path})\index{path!trivial}.
\end{enumerate}
In addition,
\begin{align}
\mathfrak{s}(p) & = \mathfrak{s}(\alpha_1)  & \mathfrak{s}(e_i) = i\notag\\
\mathfrak{e}(p) & = \mathfrak{e}(\alpha_n)  & \mathfrak{e}(e_i) = i\notag
\end{align}
\end{defin}

\begin{exam}\label{exam:1.6}
$\G\colon \xymatrix{ 1\ar[r]^\alpha & 2 \ar[r]^\beta\ar[d]^\gamma & 3\\
& 4 & }$

Paths: 
\begin{enumerate}[\rm(i)]
\item $\alpha, \beta, \gamma, \beta\alpha, \gamma\alpha$.
\item $e_1$, $e_2$, $e_3$, $e_4$.
\end{enumerate}
\end{exam}

\begin{exam}
$\G\colon \xymatrix{1\ar@(ur,dr)[]^\alpha}$.

Paths: 
\begin{enumerate}[\rm(i)]
\item $\alpha, \alpha^2 = \alpha\alpha, \alpha^3 = \alpha\alpha\alpha, \ldots$.
\item $e_1$.
\end{enumerate}
\end{exam}

\subsection{Path algebras}

Given $\G=(\G_0,\G_1)$, a quiver, and $k$ a field.

The \emph{path algebra}\index{path algebra} $k\G$:  $k\G$ is the
vector space with all the paths in $\G$ as a basis. 

The elements in $k\G$:
\[a_1p_1 + a_2p_2 + \cdots + a_tp_t\]
where $a_i\in k$ and $p_i$ are paths in $\G$.

\begin{exam}
Continueing \hyperref[exam:1.6]{Example \ref*{exam:1.6}}:
\begin{align}
x & = a_1e_1 + a_2e_2 + a_3e_3 + a_4e_4 + a_5\alpha + a_6\beta +
a_7\gamma + a_8\beta\alpha + a_9\gamma\alpha\notag\\
y & = b_1e_1 + b_2e_2 + b_3e_3 + b_4e_4 + b_5\alpha + b_6\beta +
b_7\gamma + b_8\beta\alpha + b_9\gamma\alpha\notag
\end{align}
\begin{multline}
x + y = (a_1 + b_1)e_1 + (a_2 + b_2)e_2 + (a_3 + b_3)e_3 + (a_4 +
b_4)e_4 + (a_5 + b_5)\alpha\notag\\ 
 + (a_6 + b_6)\beta + (a_7 + b_7)\gamma
+ (a_8 + b_8)\beta\alpha + (a_9 + b_9(\gamma\alpha\notag
\end{multline}
\end{exam}

$p$, $q$ paths in $\G$:
\begin{enumerate}[\rm(1)]
\item $p$, $q$ both non-trivial 
\[p\cdot q = \begin{cases} pq, & \text{\ if $\mathfrak{e}(q) =\mathfrak{s}(p)$}\notag\\
0, & \text{\ otherwise}\notag
\end{cases}\]
\item $p$ non-trivial, $q$ trivial, $q = e_i$
\[p\cdot q = \begin{cases} p, & \text{\ if $\mathfrak{s}(p) = i =\mathfrak{e}(q)$}\notag\\
0, & \text{\ otherwise}\notag
\end{cases}\]
\[q\cdot p = \begin{cases} p, & \text{\ if $\mathfrak{e}(p) = i = \mathfrak{s}(q)$}\notag\\
0, & \text{\ otherwise}\notag
\end{cases}\]
\item $p = e_i$, $q = e_j$ (both trivial)
\[p\cdot q = \begin{cases} e_i, & \text{\ if $\mathfrak{e}(q) = j = i
    = \mathfrak{s}(p)$}\notag\\
0, & \text{\ otherwise}\notag
\end{cases}\]
\end{enumerate}
 This is extended distributively to an operator on $k\G$ (see
 \cite[page 50]{ARS}). 

\begin{exam}
$\Gamma\colon \xymatrix{1\ar[r]^\alpha & 2}$, $k$ field.  

Elements in $k\Gamma$:  $a_1e_1 + a_2e_2 + a_3\alpha = y$. 

\[\begin{array}{c||c|c|c}
      & e_1 & e_2 & \alpha \\ \hline\hline 
e_1 & e_1 &  0   &    0   \\ \hline
e_2 &   0   & e_2 &  \alpha \\ \hline
\alpha & \alpha & 0 & 0
\end{array}\]
\end{exam}
\begin{align}
(e_1 + e_2)\cdot y & = (e_1 + e_2)(a_1e_1 + a_2e_2 +
                     a_3\alpha)\notag\\
& = a_1e_1^2 + a_2\underbrace{e_1e_2}_{=0} + a_3\underbrace{e_1\alpha}_{=0} + a_1\underbrace{e_2e_1}_{=0} + a_2e_2^2 + a_3
  e_2\alpha\notag\\
& = a_1e_1 + a_2e_2 + a_3\alpha = y\notag
\end{align} 
Similary $y\cdot (e_1 + e_2) = y$.  Hence,  $e_1 + e_2$ acts like $1$
in $k\Gamma$. 

Basis for $k\Gamma$: $\{ e_1, e_2, \alpha\}$, $\dim_kk\Gamma = 3$.

\begin{exam}
$\Gamma\colon \xymatrix{1\ar@(ur,dr)[]^\alpha}$, and $k$ a field.

$k\Gamma$ has basis: $\{e_1, \alpha, \alpha^2, \alpha^3, \ldots\}$,
that is, $\dim_k k\Gamma = \infty$. 

Elemenets in $k\Gamma$: $a_0e_1 + a_1\alpha + a\alpha^2 + \cdots +
a_t\alpha^t$, with $a_i$ in $k$ and $t\geqslant 0$.  
\end{exam}

\printindex
\begin{thebibliography}{ARS}
\bibitem{ARS} Auslander, M., Reiten, I., Smal\o, S.\ O.,
  \emph{Representation theory of Artin algebras}. Corrected reprint of
  the 1995 original. Cambridge Studies in Advanced Mathematics,
  36. Cambridge University Press, Cambridge, 1997. xiv+425 pp. ISBN:
  0-521-41134-3; 0-521-59923-7.
\end{thebibliography}
\end{document}