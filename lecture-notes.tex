\documentclass{amsart}
\usepackage{amsmath, amsthm, amscd, amssymb, latexsym, mathtools, centernot, faktor, mathrsfs}
\usepackage[all]{xy}
\usepackage{imakeidx,enumerate, stmaryrd}
\numberwithin{equation}{section}
\newtheorem{thm}{Theorem}
\newtheorem{cor}[thm]{Corollary}
\newtheorem{lem}[thm]{Lemma}
\newtheorem{prop}[thm]{Proposition}
\theoremstyle{definition}
\newtheorem{defin}{Definition}[section]
\newtheorem{rem}[defin]{Remark}
\newtheorem{exam}[defin]{Example}
\newtheorem{note}[defin]{Note}
\newtheorem{exer}[defin]{Exercise}
\newtheorem{recall}[defin]{Recall}

\newcommand{\G}{\Gamma}
\renewcommand{\L}{\Lambda}

\newcommand{\extto}{\xrightarrow}
\newcommand{\Ker}{\operatorname{Ker}\nolimits}
\renewcommand{\Im}{\operatorname{Im}\nolimits}
\newcommand {\defeq}{\stackrel{\mathclap{\normalfont\mbox{def}}}{=}}
\newcommand{\blitza}{{\usefont{U}{ulsy}{m}{n}\symbol{'011}}}
\newcommand{\kar}{\operatorname{char}\nolimits}
\newcommand{\rad}{\operatorname{rad}\nolimits}
\newcommand{\Ann}{\operatorname{Ann}\nolimits}
\newcommand{\Image}{\operatorname{Im}\nolimits}
\DeclareMathOperator{\End}{End}

\usepackage{hyperref}
\makeindex

\begin{document}
\title{Lecture notes for MA3203 Ring theory}

\author{\O yvind Solberg}
\address{Department of Mathematical Sciences\\
NTNU\\
N-7491 Trondheim, Norway}
\email{oyvind.solberg@math.ntnu.no}

\maketitle
\tableofcontents

\section*{Acknowledgement}
I thank the following students Nils Olai E.\ Stav and Thor Tunge from
spring 2017 for typesetting parts of the lecture notes up to page
\pageref{2017-pages}.

\section{Quivers}
\subsection{Quivers, vertices, arrows and paths}
\begin{defin}
A \emph{quiver}\index{quiver} $\G = (\G_0,\G_1)$ is an oriented graph,
\begin{align}
\G_0 & = \{\textit{vertices}\index{quiver!vertices}\}  (= \{1,2,\ldots,n\}).\notag\\
\G_1 & = \{\textit{arrows}\index{quiver!arrows}\}.\notag
\end{align}
\end{defin}
We always assume that $\G_0$ and $\G_1$ are finite sets.

\begin{exam}
$\G\colon \xymatrix{1\ar[r]^\alpha & 2}$, $\G_0=\{1,2\}$ and $\G_1=\{\alpha\}$. 
\end{exam}

\begin{exam}
$\G\colon \xymatrix{1\ar@(ur,dr)[]^\alpha}$, $\G_0=\{1\}$ and $\G_1=\{\alpha\}$. 
\end{exam}

\begin{exam} $\G\colon
  \xymatrix{1\ar@/^.3pc/[rr]^\alpha\ar@/_.3pc/[rr]_\beta & & 2\ar@(ur,dr)[]^\gamma\ar@/_0.3pc/[dl]_\delta\\
    & 3\ar[ul]^\theta\ar@/_0.3pc/[ur]_\epsilon & }$, $\G_0=\{1,2,3\}$ and $\G_1=\{\alpha,\beta,\gamma,\delta,\epsilon, \theta\}$.  
\end{exam}
Have maps: $\mathfrak{s}, \mathfrak{e}\colon \G_1\to \G_0$
\begin{align}
\mathfrak{s}(\alpha) & = \text{the vertex where $\alpha\in\G_1$ starts,}\notag\index{$\mathfrak{s}$}\\
\mathfrak{e}(\alpha) & = \text{the vertex where $\alpha\in\G_1$ ends.}\notag\index{$\mathfrak{e}$}
\end{align}

\begin{defin}
$\G=(\G_0,\G_1)$ quiver.  A \emph{path}\index{path} in $\G$ is either
\begin{enumerate}[\rm(i)]
\item an ordered sequence of arrows $p=\alpha_n\alpha_{n-1}\cdots\alpha_1$, where 
\[\mathfrak{e}(\alpha_t) = \mathfrak{s}(\alpha_{t+1})\]
for $t = 1,2,\ldots,n-1$ (\emph{non-trivial path})\index{path!non-trivial} or
\item $e_i$ for each $i$ in $\G_0$ (\emph{trivial path})\index{path!trivial}.
\end{enumerate}
In addition,
\begin{align}
\mathfrak{s}(p) & = \mathfrak{s}(\alpha_1)  & \mathfrak{s}(e_i) = i\notag\\
\mathfrak{e}(p) & = \mathfrak{e}(\alpha_n)  & \mathfrak{e}(e_i) = i\notag
\end{align}
\end{defin}

\begin{exam}\label{exam:1.6}
$\G\colon \xymatrix{ 1\ar[r]^\alpha & 2 \ar[r]^\beta\ar[d]^\gamma & 3\\
& 4 & }$

Paths: 
\begin{enumerate}[\rm(i)]
\item $\alpha, \beta, \gamma, \beta\alpha, \gamma\alpha$.
\item $e_1$, $e_2$, $e_3$, $e_4$.
\end{enumerate}
\end{exam}

\begin{exam}
$\G\colon \xymatrix{1\ar@(ur,dr)[]^\alpha}$.

Paths: 
\begin{enumerate}[\rm(i)]
\item $\alpha, \alpha^2 = \alpha\alpha, \alpha^3 = \alpha\alpha\alpha, \ldots$.
\item $e_1$.
\end{enumerate}
\end{exam}

\subsection{Path algebras}

Given $\G=(\G_0,\G_1)$, a quiver, and $k$ a field.

The \emph{path algebra}\index{path algebra} $k\G$:  $k\G$ is the
vector space with all the paths in $\G$ as a basis. 

The elements in $k\G$:
\[a_1p_1 + a_2p_2 + \cdots + a_tp_t\]
where $a_i\in k$ and $p_i$ are paths in $\G$.  We write just $p$ for
$1p$, when $p$ is a path in $\G$. 

\begin{exam}
Continuing \hyperref[exam:1.6]{Example \ref*{exam:1.6}}:
\begin{align}
x & = a_1e_1 + a_2e_2 + a_3e_3 + a_4e_4 + a_5\alpha + a_6\beta +
a_7\gamma + a_8\beta\alpha + a_9\gamma\alpha\notag\\
y & = b_1e_1 + b_2e_2 + b_3e_3 + b_4e_4 + b_5\alpha + b_6\beta +
b_7\gamma + b_8\beta\alpha + b_9\gamma\alpha\notag
\end{align}
\begin{multline}
x + y = (a_1 + b_1)e_1 + (a_2 + b_2)e_2 + (a_3 + b_3)e_3 + (a_4 +
b_4)e_4 + (a_5 + b_5)\alpha\notag\\ 
 + (a_6 + b_6)\beta + (a_7 + b_7)\gamma
+ (a_8 + b_8)\beta\alpha + (a_9 + b_9)\gamma\alpha\notag
\end{multline}
\end{exam}

\subsubsection*{Multiplication}\index{path algebra!multiplication} $p$, $q$ paths in $\G$:
\begin{enumerate}[\rm(1)]
\item $p$, $q$ both non-trivial 
\[p\cdot q = \begin{cases} pq, & \text{\ if $\mathfrak{e}(q) =\mathfrak{s}(p)$}\notag\\
0, & \text{\ otherwise}\notag
\end{cases}\]
\item $p$ non-trivial, $q$ trivial, $q = e_i$
\[p\cdot q = \begin{cases} p, & \text{\ if $\mathfrak{s}(p) = i =\mathfrak{e}(q)$}\notag\\
0, & \text{\ otherwise}\notag
\end{cases}\]
\[q\cdot p = \begin{cases} p, & \text{\ if $\mathfrak{e}(p) = i = \mathfrak{s}(q)$}\notag\\
0, & \text{\ otherwise}\notag
\end{cases}\]
\item $p = e_i$, $q = e_j$ (both trivial)
\[p\cdot q = \begin{cases} e_i, & \text{\ if $\mathfrak{e}(q) = j = i
    = \mathfrak{s}(p)$}\notag\\
0, & \text{\ otherwise}\notag
\end{cases}\]
\end{enumerate}
 This is extended distributively to an operation on $k\G$ (see
 \cite[page 50]{ARS}).

\begin{exam}
$\Gamma\colon \xymatrix{1\ar[r]^\alpha & 2}$, $k$ field.  

Elements in $k\Gamma$:  $a_1e_1 + a_2e_2 + a_3\alpha = y$. 

\[\begin{array}{c||c|c|c}
      & e_1 & e_2 & \alpha \\ \hline\hline 
e_1 & e_1 &  0   &    0   \\ \hline
e_2 &   0   & e_2 &  \alpha \\ \hline
\alpha & \alpha & 0 & 0
\end{array}\]
\end{exam}
\begin{align}
(e_1 + e_2)\cdot y & = (e_1 + e_2)(a_1e_1 + a_2e_2 +
                     a_3\alpha)\notag\\
& = a_1e_1^2 + a_2\underbrace{e_1e_2}_{=0} + a_3\underbrace{e_1\alpha}_{=0} + a_1\underbrace{e_2e_1}_{=0} + a_2e_2^2 + a_3
  e_2\alpha\notag\\
& = a_1e_1 + a_2e_2 + a_3\alpha = y\notag
\end{align} 
Similarly $y\cdot (e_1 + e_2) = y$.  Hence,  $e_1 + e_2$ acts like $1$
in $k\Gamma$. 

Basis for $k\Gamma$: $\{ e_1, e_2, \alpha\}$, $\dim_kk\Gamma = 3$.

\begin{exam}
$\Gamma\colon \xymatrix{1\ar@(ur,dr)[]^\alpha}$, and $k$ a field.

$k\Gamma$ has basis: $\{e_1, \alpha, \alpha^2, \alpha^3, \ldots\}$,
that is, $\dim_k k\Gamma = \infty$. 

Elements in $k\Gamma$: $a_0e_1 + a_1\alpha + a\alpha^2 + \cdots +
a_t\alpha^t$, with $a_i$ in $k$ and $t\geqslant 0$.  
\end{exam}

\begin{note}
\begin{enumerate}
\item In general, $\{e_i\}_{i \in \G}$ are \emph{orthogonal
    idempotents}\index{idempotents!orthogonal} in $k\Gamma$, that is, 
	\[ \begin{cases}\text{\ $e_i^2 = e_i$}\notag\\
	\text{$e_ie_j = 0$ for $i\neq j$}\notag
	\end{cases}\]
	
    \item Suppose $\G_0 = \{1,2,\ldots,n\}$. Then $e_1 + e_2 + \cdots + e_n$
      acts like 1 in $k\G$. Enough to show that
      \[p = (e_1 + e_2 + \cdots + e_n)p = p(e_1 + e_2 + \cdots +
        e_n)\] 
      for any path $p$. Suppose that $\mathfrak{s}(p) = i$ and
      $\mathfrak{e}(p) = j$. Then 
      \[(e_1 + e_2 + \cdots + e_n)p = e_1p + e_2p + \cdots + e_jp + \cdots +
      e_np = e_jp \defeq p\]
      \[p(e_1 + e_2 + \cdots + e_n) = pe_1 + pe_2 + \cdots + pe_i + \cdots +
      pe_n = e_jp \defeq p\]
	
    $\implies e_1 + e_2 + \cdots + e_n = 1_{k\G}$ = identity in $k\G$
\end{enumerate}
Can show: $k\G$ is a $k$-algebra with $e_1 + e_2 + \cdots + e_n$ as an
identity (see \cite[page 50]{ARS}). 
\end{note}

Recall: $\Lambda$ ring, $k$ field. 
\begin{defin}
  $\Lambda$ is a $k$-\emph{algebra}\index{algebra}, if $\Lambda$ is a
  vector space over $k$ ($\xymatrix{k\times\Lambda \ar[r] & \Lambda}$,
  $\Lambda$ is a module over $k$,
  $\alpha \in k, \lambda \in \Lambda, \alpha\cdot\lambda$) and
\[\alpha(\lambda\cdot\lambda^{'})=(\alpha\cdot\lambda)\cdot\lambda^{'}
  = \lambda(\alpha\cdot\lambda^{'})\]
  $\forall \alpha \in k, \forall \lambda,
  \lambda'\in\Lambda$. 
	
\begin{note}
  $\Lambda$ is a $k$-algebra, if $\exists\ \phi\colon k \to \Lambda$ a
  ring homomorphism such that
 \[\Image\phi \subseteq Z(\Lambda)= \{z\in \Lambda \mid 
  z\lambda=\lambda z,\forall \lambda \in \Lambda\}\]
  ($\iff \exists R \subseteq \Lambda$ subring such that $R \simeq k$
  with $R \subseteq Z(\Lambda)$, just define $\phi (a)=a \cdot
  1_{\Lambda}$). 
\end{note}
  For $ k\G $ the ring homomorphism $\phi \colon k \rightarrow k\G $
  is given by $\phi(a) = ae_1 + ae_2 + \cdots + ae_n$
\end{defin}
\begin{exer}
\begin{enumerate}
\item $\Gamma\colon \xymatrix{1\ar[r]^\alpha & 2}$, $k$ field. 

Find a $k$-algebra isomorphism
\[\psi\colon k\G \rightarrow \begin{pmatrix}
    k & 0\\
    k & k
  \end{pmatrix}.\]
\item $\G\colon \xymatrix{1\ar@(ur,dr)[]^\alpha}$,  $k$ field.

Show that $k\G\simeq k[x]$ as $k$-algebras.
\end{enumerate}
\end{exer}

\begin{defin}
  A non-trivial path $p$ in $\G$ is an \emph{oriented
    cycle}\index{oriented cycle} if
  \[\mathfrak{e}(p) = \mathfrak{s}(p).\]
\end{defin}

\begin{exam}
  $\G\colon \xymatrix{\ar@(ul,dl)[]_\alpha 1 \ar@/^.3pc/[r]^\beta &
    \ar@/^.3pc/[l]^\gamma 2}$\\ \newline 

\subsubsection*{Cycles}
  $\alpha,\alpha^3,\gamma\beta\alpha,\beta\alpha^{10}\gamma, \ldots$
  $\dim_k k\G = \infty$
\end{exam}

\begin{prop}\label{prop1}
  $\G = (\G_0,\G_1)$ quiver, $k$ field.
\begin{center}
  $\dim_kk\G < \infty$ if and only $\G$ has no oriented cycles.
\end{center}
\end{prop}
\begin{proof}
  Exercise.
\end{proof}

\begin{prop}
  Assume that $\G=(\G_0,\G_1)$ has no oriented cycles. 
\begin{center}
$k\G$ is semisimple $\iff \G_1 = \emptyset$.
\end{center}
\end{prop}

\begin{proof}
  Proposition \ref{prop1} $ \implies \text{dim}_kk\G < \infty \implies k\G$ is a left artinian ring. \\
  $k\G$ semisimple $\iff $ no non-zero nilpotent left ideals in $k\G$.\\

  $\Mapsto$: Assume that $\G_1\neq\emptyset.$ Let $\alpha_1$ be an
  arrow in $\G$. Want to find a vertex where at least one arrow ends
  and no arrow starts. If $\mathfrak{e}(\alpha_1)$ is such a vertex,
  we are done. If not, there is an arrow $\alpha_2$ starting in
  $\mathfrak{e}(\alpha_1)$.  If also $\mathfrak{e}(\alpha_2)$ is not
  as above, we continue. Since $\G$ has no oriented cycles and $\G$ is
  finite, we must end up in a vertex $ v$, where arrows only end and
  no arrows starts. Say, $\alpha = \alpha_t$ is an arrow ending in
  $v$. Then consider $k\G\alpha = k\alpha$.  Since
  $(a_1\alpha)(a_2\alpha = (a_1a_2)\underbrace{(\alpha\alpha)}_{=0} =
  0$ $\implies (k\G\alpha)^2=(0)$ and $k\G\alpha \neq (0)$, we infer
  that $k\G$ is not semisimple.

$\Mapsfrom$: Assume that $\G_1 = \emptyset$. Then 
\[\G\colon \xymatrix{1 & 2\ar@{..}[rr] & &  n} \text{ ($n$ vertices)}\]
  Basis for $k\G\colon \{e_1, e_2, \ldots, e_n\}$. Elements in
  $k\G\colon a_1e_1 + a_2e_2 + \cdots + a_ne_n$ with $a_i \in k$. Have
  a ring homomorphism
\[\psi\colon \underbrace{k \times \cdots \times k}_{n}  \rightarrow k\G\]
  given by
\[\psi(a_1,a_2,\ldots,a_n)=a_1e_1 + a_2e_2+ \cdots + a_ne_n\] 
(check this!).  Show that $\psi$ is an isomorphism. Therefore $k\G$ is
  semisimple, since $k\G$ is isomorphic to a finite product of full
  matrix rings over divisjon rings.
\end{proof}

Note: $k\G$ is not always semisimple, but some factor of $k\G$ is.

\begin{prop}
  $\G=(\G_0,\G_1) $ quiver, $k$ field. Let
\[J = \{\text{all linear combinations of non-trivial paths}\}.\] 
Then $J$ is an ideal in $k\G$ and
  $k\G/J \simeq \underbrace{k \times \cdots \times k}_{|\G_0|}$,
  -semisimple
\end{prop}
\begin{proof}[Sketch of proof]
  Define $\psi\colon k\G \rightarrow \underbrace{k \times \cdots \times}_{|\G_0|=n} = k^n $
\[\psi(a_1e_1 + a_2e_2 + \cdots + a_ne_n + \text{ linear combinations
    of non-trivial paths} ) = (a_1, a_2, \cdots, a_n)\]
  \underline{Check:} \begin{enumerate}
		\item  $\psi$ is well-defined
		\item $\psi$ homomorphism of rings
		\item ker$\psi=J$
	\end{enumerate}
	$\implies k\G/J \simeq$ Im$\psi=k^n$.
\end{proof}

\section{Modules}

\begin{exam}
	$\Gamma\colon \xymatrix{1\ar[r]^\alpha & 2}$, k field. 
	What is a module over $k\G$? Let M be a left $k\G$-module.
	\underline{Recall:} 
	$1_{k\G}=e_1 + e_2$, \[e_ie_j=  
	\begin{cases}	\text{\ $e_i^2 = e_i$}\\
	\text{$e_ie_j = 0$ for i$\neq$j}
	\end{cases}\]
	\underline{Claim:} $M = e_1M\oplus e_2M$ as vector space over k.\\\newline
	\begin{proof}
	
	\begin{align*}
	m &= 1_{k\G}*m=(e_1 + e_2)m = e_1m+e_2m \in e_1M + e_2M\\
	\implies M &\subseteq e_1M + e_2M \subseteq M \implies M = e_1M + e_2M\\\\
	\text{Let } m&\in e_1M \cap e_2M \text{, i.e } m=e_1m'=e_2m''\\\\
	e_1m&=e_1(e_1m')=(e_1e_1)m'=e_1m'=m\\
	&= e_1(e_2m'')=\underbrace{(e_1e_2)m''}_{=0}=0\cdot m''=0\\
	\implies m &=0. \text{ hence } e_1M \cap e_2M = (0)\\
	\implies M &= e_1M \oplus e_2M\\ 
	\end{align*}
\end{proof}
\end{exam}

\begin{exam}
\(\Gamma :1\xmapsto{\alpha} 2\), \(k\) field\\
What is a module over \(k\Gamma\)? Let \(M\) be a left \(k\Gamma\)-module.\\
\underline{Recall}: \(1_{k\Gamma} = e_1 + e_2, e_ie_j = \begin{cases}
    e_i &, i=j \\
    0\, & ,i\not=j.
\end{cases}\)\\
\begin{prop}
 \(M = e_1\oplus e_2M\) as a vector space over \(k\).
\end{prop}
\noindent\underline{Proof}: \(m = 1_{k\Gamma}\cdot m = (e_1 + e_2)m = e_1m + e_2m \in e_1M + e_2M \implies M\subseteq e_1M + e_2M \subseteq M \implies M = e_1M + e_2M\). Let \(m\in e_1M\cap e_2M\), i.e. \(m = e_1m' = e_2m''\)\\[0.5cm]
\begin{tabular}{l}
     \(e_1m = e_(e_1m') = (e_1e_1)m' = e_1m' = m\)  \\
     \(\quad\parallel\) \\
     \(e_1(e_2m'') = (e_1e_2)m'' = 0m'' = 0\)
\end{tabular}\\[0.5cm]
\(\implies m= 0.\) Hence \(e_1M\cap e_2M = (0) \implies M = e_1M \oplus e_2M\).\\
Let \(m\in M\) Then \(e_1m = e_1(e_m + e_2m) = e_1^2m + (e_1e_2)m = e_1m)\) and \(e_2m = e_2(e_1m + e_2m) = e_2m\)\\
\(\alpha m = \alpha(e_1m + e_2m) = \alpha(e_1m) + \alpha(e_2m) = \alpha(e_1m) + 0 = \alpha m = \alpha e_1 m = (e_2\alpha)e_1m = e_2(\alpha e_1 m)\in e_2M\).\\[1cm]
\(M \xmapsto{\alpha\cdot-} M \quad\quad\)linear map \(\alpha:e_1M\xmapsto{\alpha\cdot-}e_2M\)\\
\(M\xmapsto{e_1\cdot-}M\quad\quad\)linear map, projection \(M\mapsto e_1M\)\\
\(M\xmapsto{e_2\cdot-}M\quad\quad\)linear map, projection \(M\mapsto e_2M\)\\

\begin{center}
\(e_1M \xmapsto{\alpha\cdot -}e_2M\)
\end{center}
is a representation of \(\Gamma\) over \(k\). A vector-space in each vertex and a linear map as the arrow.
\end{exam}
\noindent Given \(V\xmapsto{f}V',\quad V,V'\) vector spaces over \(k\), \(f\) a linear map. How can we construct a left \(k\Gamma\)-module?\\
From above: \(M = V\oplus V'\) as a vector space. Let \(m = (v, v')\), then\\
\begin{center}
\begin{tabular}{l}
\(e_1m \defeq (v, 0)\)\\
\(e_2m \defeq (0, v)\)\\
\(\alpha m \defeq (0, f(v))\)
\end{tabular}
\end{center}
\underline{Check}: \(M\) becomes a left \(k\Gamma\)-module!
\begin{defin}
A representation \((V, f)\) of a quiver \(\Gamma = (\Gamma_0, \Gamma_1)\) over a field \(k\) is a collection of vector spaces \(\{V(i)\}_{i\in\Gamma_0}\) over \(k\) and \(k\)-linear maps \(f_\alpha:V(i)\mapsto V(j)\) for each arrow \(\alpha: i\mapsto j\) in \(\Gamma_1\). (We assume that \(\text{dim}_kV(i) < \infty\) for all \(i\in\Gamma_0\).
\end{defin}
\begin{exam}
\begin{enumerate}[(1)]
    \item \(\Gamma: 1.\) A representation of \(\Gamma\) over \(k\) is just a vector space over \(k\).
    \item \(\Gamma: 1\xmapsto{\alpha}2\). Representation \(V(1)\xmapsto{f_\alpha}V(2)\). For example \\
    \(k\xmapsto{1}k\quad\quad k\xmapsto{0}0\quad\quad 0\xmapsto{0}k\)\quad\quad
    \(k^2\xmapsto{\tiny\begin{pmatrix} 1 & 2\\ 0 & 3 \\ -1 & 1 \end{pmatrix}}k^3\) \\
    \item \(\Gamma: \xymatrix{ & 1\ar[dl]_\alpha \ar[dr]^\beta & \\
    2\ar[dr]_{\gamma} & & 3 \ar[dl]^\delta \\
    & 4 & }
    \quad\quad \text{Representation: }\xymatrix{ & V(1)\ar[dl]_{f_\alpha} \ar[dr]^{f_\beta} & \\
    V(2)\ar[dr]_{f_\gamma} & & V(3) \ar[dl]^{f_\delta} \\
    & V(4) & }\)\\
    For example:\\
    \(\xymatrix{ & k\ar[dl]_{1} \ar[dr]^{1} & \\
    k\ar[dr]_{1} & & k \ar[dl]^{1}\\
    & k & }\quad\quad\xymatrix{ & k^2\ar[dl]_{\small\begin{pmatrix} 1 & 0 \end{pmatrix}} \ar[dr]^{\small\begin{pmatrix} 1 & -1 \\ 0 & 1 \end{pmatrix}} & \\
    k\ar[dr]_{\small\begin{pmatrix} 1 \\ 1 \end{pmatrix}} & & k^2 \ar[dl]^{\small\begin{pmatrix} 0 & 0\\ 0 & 1 \end{pmatrix}}\\
    & k^2 & }\)
\end{enumerate}
\end{exam}
\subsection{Maps Between representations}
\begin{exam}
\(\Gamma: 1\xmapsto{\alpha}2\), \(k\) a field\\
Let \(f:M\xmapsto{}N\) be a homomorphism of left \(k\Gamma\)-modules. Then
\begin{align*}
    f(e_1m) = f((e_1e_1)m) = f(e_1(e_1m)) = e_1f(e_1m) \in e_1N \implies f|_{e_1M}:e_1M \xmapsto{}e_1N
\end{align*}
Similarly, \(f|{e_2M}:e2M\xmapsto{}e_2N\). Furthermore, \\[0.5cm]
\begin{tabular}{rcl}
     \(\alpha f(e_1m)\) & \(=\) & \(f(\alpha(e_1m)) \quad\quad (\alpha = e_2\alpha)\) \\
     \(\parallel\quad\) & & \(\quad\parallel\) \\
     \(\alpha f|_{e_1M}(e_1m)\) & & \(f|_{e_2M}(\alpha(e_1m))\)
\end{tabular}
\end{exam}
\noindent Hence\\
\(\xymatrix{e_1M\ar[r]^{f|_{e_1M}}\ar[d]^{\alpha\cdot-} & e_1M\ar[d]^{\alpha\cdot\Large-} \\
e_2M\ar[r]^{f|_{e_2M}} & e_2M}\)
\begin{rem}
\( f\left(\begin{tabular}{c}
     \(1 - 1\) \\ onto \\ isom.
\end{tabular}\right)\)\(\quad\Leftrightarrow\quad\) \(f|_{e_iM}\left(\begin{tabular}{c}
     \(1 - 1\) \\ onto \\ isom.
\end{tabular}\right)\quad\) for all \(i\).
\end{rem}
\begin{defin}
Let \((V, f)\) and \((V', f')\) be two representations of \(\Gamma\) over \(k\). A \underline{homomorphism} \(h:(V, f)\mapsto (V', f')\) is a collection of linear maps
\begin{align*}
    h(i): V(i) \mapsto V'(i)
\end{align*}
for all \(i\in\Gamma_0\), such that \(\forall\alpha :i\mapsto j\in\Gamma_1\) the following diagram commutes:
\begin{align*}
  \xymatrix{V(i)\ar[rr]^h(i)\ar[dd]^{f_\alpha} & & V'(i)\ar[dd]^{f'_\alpha} \\
 & \ar@(ur,dr) & \\
 V(j)\ar[rr]^{h(j)}& & V'(j)}  
\end{align*}
i.e. \(f'_\alpha h(i) = h(j)f_\alpha\quad\forall\alpha\in\Gamma_1\). \(h\) is a(n) isomorphism, monomorphism, epimorphism if \(h(i):V(i)\mapsto V'(i)\) are all isomorphisms, monomorphisms, epimorphisms respectively.
\end{defin}
\begin{exam}
\begin{enumerate}[(1)]
    \item \(\Gamma: \xymatrix{1\ar[r]^{\alpha}& 2}, k\) is a field
    \begin{enumerate}[(a)]
        \item .\xymatrix{k\ar@{-->}[r]^{a\cdot-}\ar[d]^{1} & k\ar[d]^{0} \\
        k\ar@{-->}[r]^{0}\ar@{=}[d] & k\ar@{=}[d]\\
        (V, f) & (V', f')}
        Here \(h(1) = a\cdot-\) and \(h(2) = 0\) so \(h = (a\cdot-, 0)\)
        \item .\xymatrix{k\ar@{-->}[r]^0\ar[d]^1 & 0\ar[d]^0 \\
        k\ar@{=}[d]\ar@{-->}[r]^0 &  k\ar@{=}[d]\\
        (V, f) & (V', f') } No non-zero homomorphisms
        \item .\xymatrix{k^2\ar@{-->}[r]^{\small\begin{pmatrix}1&0\\0&1 \end{pmatrix}}\ar[d]_{\small\begin{pmatrix}0&1\\1&0 \end{pmatrix}} & k^2\ar[d]^{\small\begin{pmatrix}0&1\\1&0 \end{pmatrix}} \\
        k^2\ar@{=}[d]\ar@{-->}[r]_{\small\begin{pmatrix}1&0\\0&1 \small\end{pmatrix}} &  k^2\ar@{=}[d]\\
        (V, f) & (V', f') }
        \(h = \left(\begin{pmatrix}1&0\\0&1\end{pmatrix}, \begin{pmatrix}0&1\\1&0\end{pmatrix}\right)\) is an isomorphism.
    \end{enumerate}
\item \(\Gamma: \xymatrix{ & 1\ar[dl]_\alpha\ar[dr]^\beta & \\
                            2\ar[dr]_\gamma & & 3\ar[dl]^\delta\\
                            & 4 & }
                            \quad
                            \xymatrix{ & k\ar[dl]_1\ar[dr]^1 & \\
                            k\ar[dr]_1 & & k\ar[dl]^1\\
                            & k\ar@{=}[d] & \\
                            &(V, f) & }
                            \quad
                            \xymatrix{ & k\ar[dl]_1\ar[dr]^1 & \\
                            k\ar[dr]_1 & & k\ar[dl]^0\\
                            & k\ar@{=}[d] & \\
                            & (V', f')& }\)\\
                            here we have no isomorphism between \((V, f)\) and \((V', f')\).
\end{enumerate}
\end{exam}

\subsection{Modules and representations}

$\Gamma = (\Gamma_0,\Gamma_1)$ - quiver, $k$ field.

$M$ left $k\Gamma$-module $\leadsto\begin{cases}
(V,f) \text{, representation of $\Gamma$}\\
V(i) = e_iM\\
\text{for\ } \alpha\colon i\to j \in \Gamma_1, \text{\ we have\ }  f_\alpha\colon
V(i)=e_iM\extto{\alpha\cdot -} e_jM = V(j)\\
f_\alpha(e_im) = \alpha e_im
\end{cases}$

$(V,f)$ representaion of $\Gamma\leadsto \begin{cases}
M = \oplus_{i\in\Gamma_0} V(i) \textrm{, $k\Gamma$-module}^*\\
m = (v_1,v_2,\ldots, v_n) \in M\\
e_im \extto{\text{def}}{=} (0,\ldots,0,v_i,0,\ldots,0)\\
\text{for $\alpha\colon i\to j$ in $\Gamma_1$, remember $\alpha =
  e_j\alpha e_i$}\\
\alpha m \extto{\text{def}}{=} (0,\ldots,0,f_\alpha(v_i),0,\ldots,0)
\text{\ with $f_\alpha(v_i)$ in the $j$-th coordinate}
\end{cases}$\medskip 
$^*$Can show: This induces a left $k\Gamma$-module structure on $M$
(see \cite[page 57]{ARS}). 
\begin{exam}
$\Gamma\colon \xymatrix{1\ar[r]^\alpha & 2}$, $k$ field.

$(V,f)\colon \xymatrix{k\ar[r]^1 & k} \leadsto M = k\oplus k = k^2$
\begin{align}
e_1\cdot (a,b) & = (a, 0)\notag\\
e_2\cdot (a,b) & = (0,b)\notag\\
\alpha\cdot(a,b) & = (0,a)\notag
\end{align}
Note: $k\Gamma e_1 = k\{ e_1,\alpha\}$.  Define $\varphi\colon M\to
k\Gamma e_1$ by letting 
\[\varphi(1,0) = e_1 \text{\ and\ } \varphi(0,1) = \alpha.\]

Have: 
\begin{align}
\alpha \varphi(a,b)  & = \alpha(ae_1 + b\alpha) =
a\underbrace{\alpha e_1}_{=\alpha} +
                       b\underbrace{\alpha^2}_{=0}\notag\\
& = a\alpha\notag\\
& = \varphi(0,a) = \varphi(\alpha(a,b))\notag
\end{align} 
Similarly, $e_i\varphi(a,b) = \varphi(e_i(a,b))$. This implies that
$\varphi$ is a $k\Gamma$-homomorphism.\medskip 

$\left.\begin{matrix}
\Ker \varphi = (0)\\
\Im\varphi = k\Gamma e_1
\end{matrix}\right\} \Rightarrow M\simeq k\Gamma e_1 \text{\ as a left $k\Gamma$-module}.$
\end{exam}

\begin{exam}
$\Gamma\colon \xymatrix{ & 1\ar[dl]_\alpha\ar[dr]^\beta & \\
2\ar[dr]_\gamma & & 3\ar[dl]^\delta\\
& 4 & }$, $k$ field.

$(V,f)\colon \xymatrix{ & k\ar[dl]_{f_\alpha = 1}\ar[dr]^{f_\beta=1} & \\
k\ar[dr]_{f_\gamma=\left(\begin{smallmatrix} 1\\ 0\end{smallmatrix}\right)} & & k\ar[dl]^{f_\delta=\left(\begin{smallmatrix} 0\\ 1\end{smallmatrix}\right)}\\
& k^2 & }$

$M = V(1) \oplus V(2) \oplus V(3) \oplus V(4) = k\oplus k\oplus
k\oplus k^2$.
\begin{align}
\alpha (v_1,v_2,v_3,v_4)  & = (0, v_1, 0, 0 )\notag\\
\gamma (v_1,v_2,v_3,v_4)  & = (0, 0, 0, (v_2, 0) )\notag\\
\gamma\alpha (v_1,v_2,v_3,v_4)  & = (0, 0, 0, (v_1, 0) )\notag
\end{align}
\end{exam}

\begin{exer} Show that $M\simeq k\Gamma e_1$ as a left
  $k\Gamma$-module.
\end{exer}

\subsection{Special representations}
\begin{itemize}
\item Zero representation:  
$\begin{cases} 
V(i) = (0), \text{\ for all $i\in\Gamma_0$},\notag\\
f_\alpha = 0, \text{\ for all $\alpha\in \Gamma_1$}.\notag
\end{cases}$
\item For each $i\in\Gamma_0$, we have a representation $T_i$ given by
$T_i(j) = \begin{cases} k, \text{\ if $j = i$}\notag\\
(0), \text{\ otherwise}\notag
\end{cases}$
and $f_\alpha = 0$ for all $\alpha\in \Gamma_1$. 

$T_i$ corresponds to a left $\Gamma$-module $S_i$:

$S_i\simeq k$ as a vector space and $e_j v = \begin{cases} v, \text{\
    if $j = i$},\notag\\ 0, \text{\ otherwise},\notag\end{cases}$ and
$\alpha v = 0$ for all $\alpha\in \Gamma_1$. 
\end{itemize}

\begin{recall}
$\Lambda$ $k$-algebra, $k$ field.

\begin{tabular}{lcr}
$M$ left $\Lambda$-module & $\Rightarrow$ & $M$ $k$-vector space\\
 $\cup$ & & \\
$N$ submodule & $\Rightarrow$ & $N\subseteq M$ subspace
\end{tabular}
\end{recall}
\begin{note}
$\dim_k S_i = 1 \Rightarrow S_i$ is a simple $k\Gamma$-module.
\end{note}
\begin{defin}
$\Lambda$ ring, $(0)\neq M$ left $\Lambda$-module.  The module $M$ is
\emph{indecomposable}\index{module!indecomposable} if 
\[M\simeq M_1\oplus M_2\]
implies that $M_1=(0)$ or $M_2 = (0)$.
\end{defin}
\begin{defin}
Let $V = (V,f)$ and $V' = (V',f')$ be two representations of a quiver
$\Gamma$. Define $W = (W,h) = V \oplus V'$, \emph{the direct sum of
  the representations}\index{representation!direct sum} $V$ and $V'$ by 
\[W(i) = V(i) \oplus V'(i)\]
and
\[h_\alpha = f_\alpha\oplus f'_\alpha \colon W(i) = V(i)\oplus
  V'(i)\to V(j)\oplus V'(j) = W(j)\]
for all $i\in \Gamma_0$ and for all $\alpha\in\Gamma_1$. 
\end{defin}

\begin{defin}
$(0)\neq V = (V,f)$ is an
\emph{indecomposable}\index{representation!indecomposable}
representation if 
\[V = V_1\oplus V_2\]
implies that $V_1=(0)$ or $V_2 = (0)$. 
\end{defin}

\begin{exam}
$\Gamma\colon \xymatrix{1\ar[r]^\alpha & 2}$, $k$ field.

\begin{itemize}
\item $\xymatrix{k^2\ar[r]^{\left(\begin{smallmatrix}1 & 0 \\ 0 &
          1\end{smallmatrix}\right)} & 2}
\simeq \xymatrix{k\ar[r]^1 & k}\oplus \xymatrix{k\ar[r]^1 & k}$
\item $\xymatrix{k\ar[r]^1 & k}$ indecomposable? Others?
\end{itemize}
\end{exam}

\subsection{Subrepresentations}

$\Gamma = (\Gamma_0,\Gamma_1)$ quiver, $k$ field.

$M$, $N$ $k\Gamma$-modules, $N\subseteq M$ submodule.

$\Rightarrow e_iN\subseteq e_iM$ subspace.

Given $\alpha\colon i\to j \in \Gamma_1$, the following diagram commutes

\[\xymatrix{
e_i N \ar@{^(->}[r] \ar[d]_{\alpha\cdot-} & e_i M\ar[d]^{\alpha\cdot
  -}\\
e_jN \ar@{^(->}[r] & e_j M
}\]
\begin{defin}
\begin{enumerate}[\rm(a)]
\item $(V,f)\subseteq (V',f')$ is a
\emph{subrepresentation}\index{representation!subrepresentation}\index{subrepresentation}
if
\begin{enumerate}[\rm(i)]
\item $V(i) \subseteq V'(i)$ subspace for all $i\in\Gamma_0$, 
\item \[\xymatrix{
V(i) \ar@{^(->}[r] \ar[d]_{f_\alpha} & V'(i)\ar[d]^{f'_\alpha}\\
V(j) \ar@{^(->}[r] & V'(j)}\]
for $\alpha\colon i\to j\in\Gamma_1$, that is, $f_\alpha =
f'_\alpha|_{V(i)}$. 
\end{enumerate}
\item If $(V,f)\subseteq (V',f')$ is a subrepresentation, then the
  \emph{factor representation}\index{representation!factor
    representation} $W=(W,f'')$ of $V$ and $V'$ is given as
\begin{enumerate}[\rm(i)] 
\item $W(i) = V'(i)/V(i)$, 
\item \[\xymatrix{
V(i) \ar@{^(->}[r] \ar[d]_{f_\alpha} & V'(i)\ar[d]^{f'_\alpha}\ar[r] & V'(i)/V(i)=W(i)\ar@{-->}[d]^{f''_\alpha} \\
V(j) \ar@{^(->}[r] & V'(j) \ar[r] & V'(j)/V(j)=W(j)}\]
where $f''_\alpha(v' + V(i)) = f'_\alpha(v') + V(j)$ for $\alpha\colon
i\to j\in\Gamma_1$. 
\end{enumerate}
\end{enumerate}
\end{defin}
Check: 
\begin{enumerate}[\rm(i)]
\item $f''_{\alpha}$ is well-defined.
\item $(W,f'')$ is a representation of ${\Gamma}$ over $k$. 
\item We have
\[\xymatrix{
(V,f) \ar@{^(->}[r]\ar@{~>}[d] & (V',f') \ar@{~>}[d]&  W = V'/V \ar@{~>}[d] \\
M_V \ar@{^(->}[r] & M_{V'} & M_W \simeq M_{V'}/M_V
}\]
\end{enumerate}

\begin{defin}
$\Lambda$ finite dimensional $k$-algebra, $k$ field.  Then $\Lambda$
is of \emph{finite representation type}\index{representation
  type!finite} if there is only a finite number of non-isomorphic
indecomposable finitely generated left $\Lambda$-modules.
\end{defin}

\begin{exam}
$\Lambda = k$.  The only indecomposable $\Lambda$-module is $k$. 
\end{exam}

\begin{exam}
$\Gamma\colon \xymatrix{1\ar[r]^\alpha & 2}$, $k$ field.

The indecomposable left $k\Gamma$-modules $\leftrightsquigarrow$ \parbox{6cm}{The
indecomposable representations of $\Gamma$ over $k$.} 

Let $(V,f) = V_1\extto{f} V_2$ is an indecomposable representation of
$\Gamma$ over $k$.

Know: $h_1\colon V_1 \extto{~} \Im f\oplus \Ker f$

In particular, $\xymatrix{V\ar@<1ex>[r]^f & \Im f\ar@<1ex>[l]^{f'}}$
such that $ff' = 1_{\Im f}$.

$h_1\colon V_1 \to \Im f \oplus \Ker f$ given by $v\mapsto (f(v), v -
f'f(v))$. 

\[\xymatrix{
V_1 \ar[rr]^f \ar[d]_{h_1} & &  V_2\ar[d]^{h_2=1_{V_2}} & \\
\Im f\oplus \Ker f \ar[rr]^-{(\nu,0)} & & V_2 & \\
& \simeq & & \\
\Im f\ar@{^(->}[rr]^\nu && V_2 & =(0) \text{\ (ii)}\\
 & \oplus & \\
\Ker f \ar[rr]^0 && 0 & = (0) \text{\ (i)}
}\]
where $\nu\colon \Im f \hookrightarrow V_2$.

\textbf{Case (ii)}: \[\xymatrix{
\Ker f \ar[r]^0\ar[d]^\simeq & 0\ar@{=}[d]\\
k^t \ar[r]^0 & 0} \simeq \xymatrix{(k\ar[r]^0 & 0)^t}\]

$(V,f)$ indecomposable $\Rightarrow$ $t = 1$ and $(V,f)\simeq
\xymatrix{k\ar[r]^0 & 0}$ 

\textbf{Case (i)}: $\xymatrix{\Im f \ar@{^(->}[r] & V_2}$

Know: $V_2 = \Im f\oplus V_2'$

\[\xymatrix{
\Im f \ar@{^(->}[rr] \ar@{=}[d] & &  V_2\ar@{=}[d] & \\
\Im f \ar[rr]^-{\left(\begin{smallmatrix}1_{\Im f} \\ 0\end{smallmatrix}\right)} & & V_2 & \\
& \simeq & & \\
\Im f\ar@{=}[rr] && \Im f & =(0) \text{\ (i)'}\\
 & \oplus & \\
0\ar[rr]^0 && V'_2 & = (0) \text{\ (i)''}
}\]

\textbf{Case (i)'}:
\[\xymatrix{
0 \ar[r]^0\ar@{=}[d] & V'_2\ar[d]^\simeq\\
0 \ar[r]^0 & k^t} \simeq \xymatrix{(0\ar[r]^0 & k)^t}\]
$(V,f)$ indecomposable $\Rightarrow$ $t = 1$ and $(V,f)\simeq
\xymatrix{0\ar[r]^0 & k}$ 

\textbf{Case (i)''}: 
\[\xymatrix{
\Im f \ar[r]^{1_{\Im f}}\ar[d]^\varphi & \Im f\ar[d]^\varphi\\
k^t \ar[r]^{1_{k^t}} & k^t} \simeq \xymatrix{(k\ar[r]^1 & k)^t}\]
$(V,f)$ indecomposable $\Rightarrow$ $t = 1$ and $(V,f)\simeq
\xymatrix{0\ar[r]^0 & k}$ 

\textbf{Check}: $\xymatrix{k \ar[r]^1 & k}$, $\xymatrix{k\ar[r]^0 & 0}$ and
$\xymatrix{0 \ar[r]^0 & k}$ are indecomposable. 

\textbf{Hence}: The only indecomposable representations are the ones
above  $\Rightarrow$ The only indecomposable left $k\Gamma$-modules
are $k\Gamma e_1$, $k\Gamma e_1/\langle \alpha e_1\rangle$ and $S_2 =
k\Gamma e_2$. 
\end{exam}

\begin{thm}
$k$ field, $\kar k = p$, $G$ finite group with $p\mid |G|$. Then,

$kG$ of finite representation type $\Leftrightarrow$ All $p$-Sylow
subgroups of $G$ are cyclic.
\end{thm}

\begin{thm}
$\Gamma$ connected quiver without oriented cycles, $k$  field.

$k\Gamma$ is of finite representation type $\Leftrightarrow$ The
underlying graph of $\Gamma$ is a Dynkin diagram.

\[\xymatrix{\mathbb{A}_n\colon & 1\ar@{-}[r] & 2\ar@{-}[r] & \cdots &
    \ar@{-}[r] & n-1 \ar@{-}[r] & n}\]

\[\xymatrix@R=3pt{
& 1\ar@{-}[dr] &  & & & & \\
\mathbb{D}_n\colon &  & 3\ar@{-}[r] & \cdots &  \ar@{-}[r] & n-1 \ar@{-}[r] & n\\
& 2\ar@{-}[ur] & & & & & }\]

\[\xymatrix{
& & & 4\ar@{-}[d] & & \\
\mathbb{E}_6\colon & 1\ar@{-}[r] & 2\ar@{-}[r] & 3\ar@{-}[r] &
   5 \ar@{-}[r] & 6}\]

\[\xymatrix{
& & & 4\ar@{-}[d] & & & \\
\mathbb{E}_7\colon & 1\ar@{-}[r] & 2\ar@{-}[r] & 3\ar@{-}[r] &
   5 \ar@{-}[r] & 6 \ar@{-}[r] & 7}\]

\[\xymatrix{
& & & 4\ar@{-}[d] & & & &\\
\mathbb{E}_6\colon & 1\ar@{-}[r] & 2\ar@{-}[r] & 3\ar@{-}[r] &
   5 \ar@{-}[r] & 6 \ar@{-}[r] & 7 \ar@{-}[r] & 8}\]
\end{thm}

\section{Quiver with relations}
Can all algebras over a field k be represented as $k\G$?

\textbf{No}, since $\Lambda = k[x]/\langle x^2\rangle \ncong k\G$ for
all quivers $\G$.

Why? $\dim_{k}\Lambda = 2$ and $\Lambda$ is not semisimple.

Assume that  $\Lambda \cong k\G$.  We have that $2 = \dim_kk\G \ge
\text{\# vertices in\ } \G$.   If $\G \colon \xymatrix{1& 2}$,  then
$k\G$ is semisimple. This is a  contradiction. Hence $\G$ has one
vertex and contains the quiver  $\xymatrix{1\ar@(ur,dr)^\alpha}$.
This implies $\dim_kk\G = \infty$, another contradiction.  But,
$\G\colon \xymatrix{1\ar@(ur,dr)^\alpha}$, $\Lambda \simeq
\frac{k\G}{\langle\alpha^2\rangle} $.

Let $\G=(\G_0,\G_1)$ be a quiver, k field.\\
\begin{defin}
\begin{enumerate}[(a)]
	~\\ \item A \emph{relation} $\sigma$ in the quiver $\G$ over k is a k-linear combination of paths
	\begin{center}
		$\sigma = a_1p_1 + a_2p_2+ \cdots + a_tp_t$\\
	\end{center}
	where $a_i\in k$, $e(p_i)=e(p_1)$ and $s(p_i)=s(p_1)$ for all $i$, and $l(p_i)\geq 2$ (the length of the path $p_i$ )
	\item if $\varrho = \{\sigma\}_{l \in T}$ is a set of relations in $\G$ over k, then $(\G,\varrho)$ is a \emph{quiver with relations over $k$.}  
\end{enumerate}
\end{defin}
\begin{exam}
	$\xymatrix{ & 1\ar[dl]_\alpha\ar[dr]^\beta & \\
		\G \colon 2\ar[dr]_\gamma & & 3\ar[dl]^\delta\\
		& 4 & }$\newline $k$ field, $\sigma = \gamma\alpha - \delta\beta$.\\\newline
		$ \G=\frac{k\G}{\langle\sigma\rangle}$. Let $M$ be a left $k\Lambda$-module. Any left $\Lambda$-module is a left $k\Lambda$-module, since $k\G \xrightarrow{\pi} \frac{k\G}{\large\sigma\rangle} = \Lambda$.\\
		$\implies M$ gives rise to a representation of $\G$.\\
		
	$\xymatrix{ & e_1M\ar[dl]_{f_\alpha = \alpha\cdot-}\ar[dr]^{\beta\cdot-=f_\beta} & \\
	 e_2M\ar[dr]_{f_\gamma=\gamma\cdot-} & & e_3M\ar[dl]^{\delta\cdot-=f_\delta}\\
	& 4 & }$\newline

$\sigma\in k\G, m \in M.$\\ $ \sigma\cdot m\defeq\pi(\sigma)\cdot m = 0\cdot m = 0, \forall m \in M$\\
$m=e_1m+e_2m+e_3m+e_4m$, $sigma = e_4\sigma e_1$\\
$0 = \sigma\cdot m = (\gamma\alpha-\delta\beta)e_1m = \gamma(\alpha e_1m)-\delta(\beta e_1m) = f_\gamma f_\alpha(e_1m)-f_\sigma f_\beta(e_1m) = \underbrace{f_\gamma f_\alpha-f_\sigma f_\beta}_{f_\sigma}(e_im) \implies f_\sigma(e_1M)=0 \implies f_\sigma = 0 $.\\
Hence $\Lambda$-module $M$ corresponds to a representation of $\G$ satisfying the relation $\sigma(f_\sigma=0)$.\\
Conversely, we claim that a representation $(V,f)$ of $\G$ 	such that
\begin{center}
	$f_\sigma=f_\gamma f_\alpha-f_\sigma f_\beta=0$
\end{center} 
 gives a module over $\Lambda$.\\\newline
\textbf{Recall}: $I \subseteq R$ ideal: $\frac{R}{I}$-module $M$ is the same as an $R$-module $M$ such that $I\cdot M = (0)$.\\
\begin{center}
	$M = V(1) \oplus V(2) \oplus V(3) \oplus V(4) \leftarrow k\G$-module\\
	$e_1\cdot(v_1,v_2,v_3,v_4)=(v_1,0,0,0)$\\
	$\alpha\cdot(v_1,v_2,v_3,v_4)=(0,f_\alpha(v_1),0,0)$\\
\end{center}
~\\$\sigma\cdot(v_1,v_2,v_3,v_4) = (\gamma\alpha-\delta\beta)\cdot(v_1,v_2,v_3,v_4) = (0,0,0,f_\gamma f_\alpha(v_1)-f_\sigma f_\beta(v_1)) =  (0,0,0,(f_\gamma f_\alpha-f_\sigma f_\beta)(v_1)) = (0,0,0,0) $\\\newline
$\implies M$ is a $\Lambda$-module $(\Lambda=\frac{k\G}{\langle\sigma\rangle})$.
\end{exam}

\begin{exam}
$\G\colon \xymatrix{1 \ar@(ur,dr)^\alpha}, P=\{\alpha^2 \}$, $k$ field. $ \Lambda=\frac{k\G}{\langle\alpha^2\rangle}$. Find all induced $\Lambda$-modules.\\
$M$ left $\Lambda$-module $\leadsto 
(V,f) $ representation of $\G$ satisfying the relation $\alpha^2, $i.e.  $ f_{\alpha^2}=(f_\alpha)^2$\\  
	
	$\xymatrix{V \ar@(ur,dr)^{f_\alpha}}, (f_\alpha)^2=0$\\\newline
	$\implies$ The minimal polynomial of $f_\alpha$ is $x$ or $x^2$\\
	$\implies$ The ivariant factor of $f_\alpha$ is $x$ or $x^2$\\
	$\implies$ The matrix of $f_\alpha$ is similar to a direct sum of companion matrices of $x$ or $x^2$, $M_{(x)}=0$ and $M_{(x^2)}= \begin{pmatrix}0&0\\1&0\end{pmatrix}$\\~\\
	
Let T be the matrix of $f_\alpha$ w.r.t some basis $\beta$.\\
Then $\exists$ an invertible matrix $P$ such that\newline
	
$ T = P \underbrace{ \left(\begin{smallmatrix} r\begin{cases} \left(\begin{smallmatrix} 0 \cdots 0  \\\vdots \ddots \vdots \\0 \cdots 0 \end{smallmatrix}\right) \end{cases} & \overbrace{\text{\huge0}}^{2s}\\\text{\huge0} & 
\left(\begin{smallmatrix}  
\left( \begin{smallmatrix} 0 & 0\\ 1 & 0\\ \end{smallmatrix}\right) & & & \text{\huge0}  \\
& \begin{smallmatrix} 0 & 0\\ 1 & 0\\ \end{smallmatrix} & & \\
& &\ddots &  & \\
\text{\huge0} & & &\begin{smallmatrix} 0 & 0\\ 1 & 0\\ \end{smallmatrix} \\
\end{smallmatrix}\right) \end{smallmatrix}\right)}_{T_0}P^{-1} $ $\implies TP= PT_0$ \nolinebreak[4]\\\newline
$\xymatrix{V \ar@(ur,dr)^{T}} \iff \xymatrix{V \ar@(ur,dr)^{T_0}} \simeq (\xymatrix{k \ar@(ur,dr)^{0}})^r \oplus (\xymatrix{k^2 \ar@(ur,dr)^{\left(\begin{smallmatrix}0&0\\1&0\end{smallmatrix}\right)}})^s  $  isomorphisme of representation

\textbf{Show}: $\xymatrix{k \ar@(ur,dr)^{0}} \iff \frac{k\G}{\langle\alpha\rangle}$ and $\xymatrix{k^2 \ar@(ur,dr)^{\left(\begin{smallmatrix}0&0\\1&0\end{smallmatrix}\right)}} \iff \frac{k\G}{\langle\alpha^2\rangle} \implies \Lambda$ is of finite representation type.
\end{exam}
\section{Finite length}
$\Lambda$ ring, $A$ a (left) $\Lambda$-module.
\begin{defin}
	A has \emph{finite length}\index{finite length} if there exists a finite filtration.\\
	\begin{center}
			$\mathscr{F}\colon A = A_0 \supseteq A_1 \supseteq A_2 \supseteq \cdots A_{n-1} \supseteq A_{n} \supseteq A_{n+1} = (0) $\\
	\end{center}
of submodules of $A$ such that $\frac{A_i}{A_{i+1}} = 0$ or simple for i$=0,1,\cdots,n.$. 	$\mathscr{F}$ is a \emph{generalized composition series}\index{generalized composition series} of $A$, and if $\frac{A_i}{A_{i+1}} \ne 0$ for all i, then $\mathscr{F}$ is a \emph{composition series}\index{composition series}. If $S=\frac{A_i}{A_{i+1}} \ne 0$, then $S$ is called a \emph{composition factor of $A$}\index{composition factor}\\

Let $S$ be a simple $\Lambda$-module. Let
\begin{align}
m_s^{\mathscr{F}}(A) & \defeq \{i \mid \frac{A_i}{A_{i+1}}\simeq S \}\mid \text{ , } \notag\\
l_{\mathscr{F}} & \defeq \sum_{\mathclap{\substack{\small[S]
      \text{isomorphism} \\ \small\text{classes of
        simples}}}}m_s^{\mathscr{F}}(A)\notag\\
\intertext{ and }\notag\\
l(A) & \defeq \min_{\mathclap{\substack{\mathscr{F} \text{generalized} \\ \text{composition series}}}} l_{\mathscr{F}}(A)\notag
\end{align}
\end{defin}

\begin{exam}
	\begin{enumerate}[(1)]
		\item $\Lambda$ ring, $S$ simple $\Lambda$-module.\\
		Composition series: $S \supseteq (0)$\\
		composition factors: $\{S\}$\\
		$\implies m_T(S) = \left\{\!\begin{aligned}
			&1 &\text{if } T \simeq S\\[1ex]
			&0 &\text{otherwise}\\[1ex]
		\end{aligned}\right\}\implies l(S)=1$
		\item $\Lambda = k[x], f(x)$ irredicible\\
		$S_f=\frac{k[x]}{\langle f(x) \rangle}$ - simple $\Lambda$- module.\\
		$\implies l(S_f)=1$, while $\dim_kS_f = \deg f(x)$ \\
		\item $\G\colon  \xymatrix{1\ar[r]^\alpha & 2\ar[r]^\beta & }, k \text{ field } , \Lambda=k\G$\\
		$
		\xymatrix
		{
			& k\ar[d]^1 & 0 \ar[l]^0\ar[d]^0  & 0\ar[l]^0\ar[d]^0 & 0\ar[l]^0\ar[d]^0 \\
		\mathscr{F}\colon	&k \ar[d]^1  & k\ar[l]^1\ar[d]^1 & 0\ar[l]^0\ar[d]^0 & 0\ar[l]^0\ar[d]^0 \\
			&k \ar@{}[d]|-*[@]{=}& k\ar@{}[d]|-*[@]{=} & k\ar@{}[d]|-*[@]{=} & 0\ar@{}[d]|-*[@]{=} \\\
			&V_o\ar@{<~>}[d] & V_1\ar@{<~>}[d] & V_2\ar@{<~>}[d] & V_3\ar@{<~>}[d]\\
			M \ar@{}[r]|-*[@]{=} & M_0 \ar@{}[r]|-*[@]{\supseteq}  &  M_1 \ar@{}[r]|-*[@]{\supseteq} & M_2 \ar@{}[r]|-*[@]{\supseteq} & M_3 = (0) \\
		}$\\~\\
	
	$\xymatrix
	{
		& k\ar[d] & \\
		\frac{V_0}{V_1} \ar@{}[r]|-*[@]{\simeq}	& 0\ar[d]\ar@{<~>}[r] & S_1 , \\
		& 0 & \\
	}$
$\xymatrix
{
	& k\ar[d] & \\
	\frac{V_1}{V_2} \ar@{}[r]|-*[@]{\simeq}	& 0\ar[d]\ar@{<~>}[r] & S_2 , \\
	& 0 & \\
}$
$\xymatrix
{
	& k\ar[d] & \\
	\frac{V_2}{V_3} \ar@{}[r]|-*[@]{\simeq}	& 0\ar[d]\ar@{<~>}[r] & S_3 \\
	& 0 & \\
}$\\
$\implies l_{\mathscr{F}}(M)=3$\\~\\

\item $\G \xymatrix{ & 1\ar[dr]^{\beta} \ar[dl]_{\alpha} & \\ 2 & & 3
  },$ $k$ field, $M = k\G e_1 \leftrightsquigarrow \xymatrix{ & k\ar[dr]^{1} \ar[dl]_{1} & \\ k & & k }\\$

\[\xymatrix@C=2pt@R=10pt{
& k\ar[dr]^{1} \ar[dl]_{1} & & & & 0\ar[dr]^{0} \ar[dl]_{0} & & & & 0\ar[dr]^{0} \ar[dl]_{0} & & & & 0\ar[dr]^{0} \ar[dl]_{0} &\\
k & & k & \supseteq & k & & k & \supseteq & k & & 0 & \supseteq & 0 & & 0\\
& & & & & \ar@{}[u]|-*[@]{\subseteq} & & & & & & & & & \\
& & & & & 0\ar[dr]^{0} \ar[dl]_{0} & & & & 0\ar[dr]^{0} \ar[dl]_{0} & & & & & \\
& & & & 0 & & k & \supseteq & 0 & & 0 &  &  & & 
}\]
 Hence, we have two different composition series for the module $M$!
Computing the factors of neigbouring modules in the compositions
series, gives the same simple modules, but in a permuted order. 
\end{enumerate}
\end{exam}

\begin{note}
	\begin{enumerate}[(1)]
		\item Composition serice are not unique!
		\item $l_{\mathscr{F}}(M) = l_{\mathscr{G}}(M)$
		\item The set of composition factors is the same for $\mathscr{F}$ and $\mathscr{G}$
	\end{enumerate}
\end{note}

The proof of Jordan-Hølder theorem goes by induction on length and using short exact sequences.
\begin{defin}
	$\xymatrix{0 \ar[r] & A\ar[r]^f & B\ar[r]^g & C \ar[r] & 0}$
        is a \emph{(short) exact sequence}\index{exact sequence} of
        (left) $\Lambda$-module if 
	\begin{enumerate}[(i)]
		\item $f$ is injective (1-1),
		\item $g$ is surjective (onto),
		\item Im(f) = ker(g).
	\end{enumerate}
\end{defin} 
\begin{note}
	\begin{enumerate}[(1)]
		\item $A \subseteq B$ two  $\Lambda$-modules.  Then\\
		\begin{center}
			$\xymatrix{0 \ar[r] & A\ar@{^(->}[r] & B\ar[r] & B/A \ar[r] & 0}$
		\end{center}
	is an exact sequence.
	\item If $\xymatrix{0 \ar[r] & A\ar[r]^f & B\ar[r]^g & C \ar[r] & 0}$ is an exact sequence, then
	\begin{enumerate}[(a)]
        \item \begin{align}
                C & = \text{Im}(g)\notag\\
                  & \simeq \frac{B}{\text{Ker(g)}}\notag\\
                  & \simeq \frac{B}{\text{Im(f)}}\notag\\
                \text{Im}(f) & \simeq A\notag.
                   \end{align}
		\item $B = (0) \implies A = (0) \text{ and } C=(0). $ 
	\end{enumerate}
	\end{enumerate}
\end{note}

\begin{exam}
	\begin{enumerate}[(1)]
		\item  $\xymatrix{0 \ar[r] & \mathbb{Z}\ar[r]^{-\cdot n} & \mathbb{Z}\ar[r] & \frac{\mathbb{Z}}{n\mathbb{Z}} \ar[r] & 0}$ exact
		\item $M, N$ $\Lambda$-modules.
		$\xymatrix
		{0\ar[r] & M\ar[r]^{\left(\begin{smallmatrix} 1 \\ 0 \end{smallmatrix}\right)} & M\oplus N\ar[r]^{\left(\begin{smallmatrix} 1 & 0 \end{smallmatrix}\right)} & N\ar[r] & 0 \\
			  & m \ar@{|->}[r]& (m,0) \\
			  &   & (m,n) \ar@{|->}[r]    & n\\
		}$ exact
	\item $\Lambda = \frac{k\G}{\langle p \rangle}$, $\xymatrix{0 \ar[r] & A\ar[r]^f & B\ar[r]^g & C \ar[r] & 0}$ exact sequence of $\Lambda$-modules.
	$\xymatrix
	{
		 & V_A(i)\ar@{}[d]|-*[@]{=} \ar[r]^{f\vert_{V_A(i)}} & V_B(i)\ar@{}[d]|-*[@]{=} \ar[r]& V_C(i)\ar@{}[d]|-*[@]{=} & \\
		0\ar[r]& e_iA \ar[r]^{f\vert_{e_iA}}  & e_iB  \ar[r]^{g\vert_{e_iB}}  & e_iC \ar[r]  & 0\\
	}$ exact sequence for all i\\
Hence, 	$\xymatrix{0 \ar[r] & (V',f')\ar[r]^g & (V,f)\ar[r]^h & (V'',f'') \ar[r] & 0}$ is an exact sequence of representation if $\xymatrix{0 \ar[r] & V'(i)\ar[r]^{g(i)} & V(i)\ar[r]^{h(i)} & V''(i) \ar[r] & 0}$ is exact for all i $\in\G_0$.
	
	\end{enumerate}
\end{exam}

\begin{exer}
		$f\colon\xymatrix{A\ar[r] & B}$ and $g\colon\xymatrix{B\ar[r] & C}$, $\Lambda$-homomorphisme, $B' \subseteq B$ submodule.
		\begin{enumerate}[(1)]
			\item $f^{-1}(B')= \{ a\in A\mid f(a)\in B'\}\subseteq A$ submodule.
			\item $g(B')= \{ g(b')\mid b'\in B'\}\subseteq C$ submodule.
		\end{enumerate}
\end{exer}

Let $\xymatrix{0 \ar[r] & A\ar[r]^f & B\ar[r]^g & C \ar[r] & 0}$ be an exact sequence and let $\mathscr{F}$ be a generalized composition series of B.
\[\xymatrix
{0\ar[r] & A\ar[r]^f & B\ar[r]^g\ar & C\ar[r] & 0\\
			& A_o=f^{-1}(B_0)\ar@{}[u]|-*[@]{=}	  & B_0	\ar@{}[u]|-*[@]{=}				& g(B_0)= C_0\ar@{}[u]|-*[@]{=} \\
			& A_1=f^{-1}(B_1)\ar@{}[u]|-*[@]{\subseteq} & B_1\ar@{}[u]|-*[@]{\subseteq}& g(B_1)= C_1\ar@{}[u]|-*[@]{\subseteq} \\
			& A_2=f^{-1}(B_2)\ar@{}[u]|-*[@]{\subseteq} & B_2\ar@{}[u]|-*[@]{\subseteq}& g(B_2)= C_2\ar@{}[u]|-*[@]{\subseteq} \\
			& \vdots\ar@{}[u]|-*[@]{\subseteq} & \vdots\ar@{}[u]|-*[@]{\subseteq}& \vdots\ar@{}[u]|-*[@]{\subseteq} \\
			& A_n=f^{-1}(B_n)=(0)\ar@{}[u]|-*[@]{\subseteq} & B_n=(0)\ar@{}[u]|-*[@]{\subseteq}& g(B_n)= C_n=(0)\ar@{}[u]|-*[@]{\subseteq} \\
			& \mathscr{F}'\ar@{}[u]|-*[@]{\colon} & \mathscr{F}\ar@{}[u]|-*[@]{\colon}& \mathscr{F}''\ar@{}[u]|-*[@]{\colon} \\
}\]
We have $A_n=(0)$, since $f$ is $1$-$1$.
\addtocounter{thm}{1}
\begin{prop}\label{prop:7}
\begin{enumerate}[(a)]
\item $\mathscr{F}'$ is generalized composition series
  of $A$, and $\mathscr{F}''$ is generalized composition series of $C$.
\item $m^{\mathscr{F}}_S(B)=m^{\mathscr{F}'}_S(A) +
  m^{\mathscr{F}''}_S(C)$ for all simple module $S$. 
\end{enumerate}
\end{prop}
\begin{proof}
  (a) (I) \textbf{Claim}:
  \[(*)\colon 0\to A_i\extto{f|_{A_i}} B_i \extto{g|_{B_i}} C_i\to 0\]
  is exact.

  By definition we have that $f(A_i)\subseteq B_i$ and $g(B_i) = C_i$
  for all $i$.  Furthermore
  \begin{itemize}
    \item $f|_{A_i}\colon A_i\to B_i$ is $1$-$1$, since $f$ is
      $1$-$1$.
    \item $g|_{B_i}\colon B_i\to C_i$ is onto by the definition of
      $C_i$.
    \item Since $f(A_i) \subseteq B_i$ and $gf = 0$, then
      $g|_{B_i}f|_{A_i} = 0$. This implies that $\Im f|_{A_i}
      \subseteq \Ker g|_{B_i}$.
    \end{itemize}
Let $b\in \Ker g|_{B_i}$.  This implies that $b\in\Ker g \subseteq \Im
f$.  So there exists $a\in A$ such that  $f(a) = b$, that is, $a\in
f^{-1}(B_i) = A_i$.  This is turn implies that
\[\Ker g|_{B_i} \subseteq \Im f|_{A_i} \Rightarrow \Ker g|_{B_i} =
  \Im f|_{A_i}.\]
This proves that $(*)$ is exact.

(II) \textbf{Claim}: The following diagram is exact and commutative:
\[\xymatrix{
           & 0\ar[d]              & 0\ar[d]              & 0\ar[d]             & \\
0\ar[r] & A_{i+1} \ar[r]^f\ar@{^(->}[d] & B_{i+1} \ar[r]^g\ar@{^(->}[d] & C_{i+1} \ar[r]\ar@{^(->}[d] &
0\\
0\ar[r] & A_{i} \ar[r]^f\ar[d]^{p'_i} & B_{i} \ar[r]^g\ar[d]^{p_i} & C_{i} \ar[r]\ar[d]^{p''_i} &
0\\
0\ar[r] & A_{i}/A_{i+1} \ar[r]^{\overline{f}}\ar[d] & B_{i}/B_{i+1} \ar[r]^{\overline{g}}\ar[d] & C_{i}/C_{i+1} \ar[r]\ar[d] & 0\\
           & 0              & 0              & 0             & 
}\]
where $p'_i$, $p_i$ and $p''_i$ are the natural projections,
$\overline{f}(a_i + A_{i+1}) = f(a_i) + B_{i+1}$ and $\overline{g}(b_i
+ B_{i+1}) = g(b_i) + C_{i+1}$.  Denote by $\eta_i$ the lower row in
the above diagram.  Easy to see that the diagram is
commutative, given that the everything is well-defined.

(i) \textbf{$\overline{f}$ well-defined}: Assume that $a_i + A_{i+1} =
a'_i + A_{i+1}$, that is, $a_i - a'_i\in A_{i+1}$. Then
$f(a_i-a'_i) = f(a_i) - f(a'_i)\in B_{i+1}$.  This means that
\[\overline{f}(a_i + A_{i+1}) = f(a_i) + A_{i+1} = f(a'_i) + A_{i+1} =
  \overline{f}(a'_i + A_{i+1}).\]
This shows that $\overline{f}$ is well-defined.

(ii) \textbf{$\overline{g}$ well-defined}: Similar.

(iii) \textbf{$\eta_i$ is exact}: (1) $\overline{f}$ is $1$-$1$:
Assume that $\overline{f}(a_i+A_{i+1}) = f(a_i) + A_{i+1} =0$ for
$a_i\in A_i$. This means that $f(a_i)\in B_{i+1}$.  Then by definition
$a_i\in A_{i+1}$ and consequently $a_i+A_{i+1} = 0$ and $\overline{f}$
is $1$-$1$.

(2) \textbf{$\overline{g}$ is onto}: Since $g$ and $p''_i$ are onto,
the composition $p''_ig = \overline{g}p_i$ is onto.  It follows from
this that $\overline{g}$ is onto.

(3) $\Im \overline{f} = \Ker \overline{g}$: Since $gf=0$, we have that
\[\overline{g}\overline{f}(a_i+A_{i+1}) - \overline{g}(f(a_i)+B_{i+1})
    = gf(a_i) + C_{i+1}.\]
  This implies that $\Im \overline{f} \subseteq \Ker \overline{g}$.

  Let $b_i +B_{i+1} \in \Ker \overline{g}$ for $b_i\in B_i$, that is,
  \[0 = \overline{g}(b_i + B_{i+1}) = g(b_i) + C_{i+1},\]
  hence $g(b_i)\in C_{i+1}$.  Choose $b_{i+1}\in B_{i+1}$ such that
  $g(b_{i+1}) = g(b_i)$.  Then $b_i - b_{i+1} \in B_i$, since $b_i$
  and $b_{i+1}$ are in $B_i$.  We infer that $g(b_i - b_{i+1}) = 0$,
  that is, $b_i - b_{i+1}\in \Ker g = \Im f$.  Choose $a_i\in A_i$
  such that $f(a_i) = b_i - b_{i+1}\in B_i$.  It follows from this
  that
  \[\overline{f}(a_i + A_{i+1}) = f(a_i) + B_{i+1} = b_i - b_{i+1} +
    B_{i+1} = b_i + B_{i+1},\]
  since $b_{i+1}\in B_{i+1}$.  This proves that $\Ker \overline{g}
  \subseteq \Im \overline{f}$. Combining the above we obtain that $\Im
  \overline{f} = \Ker \overline{g}$ and that $\eta_i$ is exact.

  We have that
  \[B_i/B_{i+1} = (0) \Rightarrow A_i/A_{i+1} = C_i/C_{i+1} = (0).\]
  If $B_i/B_{i+1}\simeq S$ is simple, then $A_i/A_{i+1} = (0)$ or $A_i/A_{i+1}
  \simeq S$.  If $A_i/A_{i+1} = (0)$, then $(0) = \Im \overline{f} =
  \Ker \overline{g}$, which means that $S\simeq B_i/B_{i+1} \simeq
  C_i/C_{i+1}$.  If $A_i/A_{i+1} \simeq S$, then $B_i/B_{i+1} = \Im \overline{f} =
  \Ker \overline{g}$.  This implies that $\overline{g} = (0)$ and that
  $C_i/C_{i+1} = \Im\overline{g} = (0)$.  Summing up, we get that
  either is
  \[A_i/A_{i+1} = (0) \textrm{\ and\ } C_i/C_{i+1} \simeq S\]
  or
  \[A_i/A_{i+1} \simeq S \textrm{\ and\ } C_i/C_{i+1} = (0).\]
This proves that $\mathscr{F}'$ is a generalized composition series of
$A$ and that $\mathscr{F}''$ is a generalized composition series of
$C$.

(b) Immediate consequence of the proof of (a).
\end{proof}

\begin{cor}\label{cor:8}
Given $A$, $B$ and $C$ three $\L$-modules with $B$ of finite length
and an exact sequence
\[0\to A\to B\to C\to 0.\]
Then
\[l(A) + l(C) \leq l(B).\]
In particular, $A$ and $C$ have finite length. 
\end{cor}
\begin{proof}
  Let $\mathscr{F}$ be a generalized composition series of $B$ such
  that $l(B) = l_{\mathscr{F}}(B)$.  Keeping the notation from before,
  we get
  \[l_{\mathscr{F}'}(A) + l_{\mathscr{F}''}(C) = l_{\mathscr{F}}(B) =
    l(B).\]
  Since $l_{\mathscr{F}'} \geq l(A)$ and $l_{\mathscr{F}''}(C) \geq
  l(C)$, we obtain that 
\[l(A) + l(C) \leq l(B).\]
\end{proof}

\begin{thm}\label{thm:9}
Let $B$ be a $\L$-module of finite length with $\mathscr{F}$ and
$\mathscr{G}$ two generalized composition series of $B$. Then
\begin{enumerate}[\rm(a)]
\item $m^{\mathscr{F}}_S(B) = m^{\mathscr{G}}_S(B) \defeq m_S(B)$.
\item $l_{\mathscr{F}}(B) = l_{\mathscr{G}}(B) \defeq l(B)$. 
\end{enumerate}
\end{thm}
\begin{proof}
(a) We use induction on $l(B)$. 

(1) $l(B) =1$: Then $B=S$ is a simple module and $m^{\mathscr{F}}_S(B)
= 1$ for all $\mathscr{F}$ and $m^{\mathscr{F}}_{S'}(B) = 0$ for all
simple modules $S'\not\simeq S$.  Hence, (a) holds.

(2) Assume that $n = l(B) > 1$, and assume that (a) is shown for $C$
with $l(C) < n$.  Since $B$ is not simple, choose $A\subseteq B$ with
$(0)\subsetneq A \subsetneq B$.  Have an exact sequence
\[0\to A\to B\to B/A\to 0.\]
Then $A$ and $B/A$ have finite length by Corollary \ref{cor:8} and
$l(A) + l(B/A) \leq l(B)$.  Since $A\neq (0)$ and $B/A\neq (0)$, we
have $l(A) > 0$ and $l(B/A) > 0$. Let $\mathscr{F}'$ and
$\mathscr{G}'$, and $\mathscr{F}''$ and $\mathscr{G}''$ be the induced
generalized composition series for $A$ and $B/A$, respectively of
$\mathscr{F}$ and $\mathscr{G}$.  Then by Proposition \ref{prop:7} we
have
\begin{align}
m^{\mathscr{F}}_S(B) & = m^{\mathscr{F}'}_S(A) + m^{\mathscr{F}''}_S(B/A) \notag\\
\intertext{and}\notag\\
m^{\mathscr{G}}_S(B) & = m^{\mathscr{G}'}_S(A) +
                       m^{\mathscr{G}''}_S(B/A) \notag
\end{align}
By induction we have $m^{\mathscr{F}'}_S(A) = m^{\mathscr{G}'}_S(A)$
and $m^{\mathscr{F}''}_S(B/A) = m^{\mathscr{G}''}_S(B/A)$, since $l(A)
< l(B)$ and $l(B/A) < l(B)$.  It follows that
\[m^{\mathscr{F}}_S(B) = m^{\mathscr{G}}_S(B)\]
for all simple $\L$-modules $S$.  

(b) Follows directly from (a), and $l_{\mathscr{F}}(B) =
l_{\mathscr{G}}(B) \defeq l(B)$. 
\end{proof}

\begin{note}
(1) It follows that if $B$ has finite length, then the set of
composition factors is uniquely determined up to isomorphism and
multiplicity.

(2) If $B\simeq C$, then $l(B) = l(C)$. 
\end{note}

\begin{prop}\label{prop:10}
If $0\to A\to B\to C\to 0$ is exact and $l(B) < \infty$, then 
\[l(B) = l(A) + l(C).\]
\end{prop}
\begin{proof}
Let $\mathscr{F}$ be a generalized composition series of $B$. Then
\begin{align}
l(B) & = l_{\mathscr{F}}(B) \defeq \sum_{[S]}
  m^{\mathscr{F}}_S(B)\notag\\
& = \sum_{[S]} (m^{\mathscr{F}'}_S(A) +
  m^{\mathscr{F}''}_S(C))\notag\\
& = l_{\mathscr{F}'}(A) + l_{\mathscr{F}''}(C)\notag
& = l(A) + l(C)\notag
\end{align}
since $l_{\mathscr{F}'}(A) = l(A)$ and $l_{\mathscr{F}''}(C) = l(C)$. 
\end{proof}

\begin{exam}
(1) Let $\Gamma = \xymatrix{1\ar@(ur,dr)^\alpha }$, and let $\Lambda =
k\Gamma/\langle \alpha^2\rangle$ for some field $k$. 

Left $\Lambda$-module $M = \Lambda$ correspond to the representation
$\xymatrix{k^2\ar@(ur,dr)^{\begin{smallmatrix} 0 & 0\\ 1 &
      0\end{smallmatrix}}}$. 

\[M = M_0 = \xymatrix{k^2\ar@(ur,dr)^{\begin{smallmatrix} 0 & 0\\ 1 &
        0\end{smallmatrix}} & & \ar@{_(->}[l] k\ar@(ur,dr)^{0} & &
    \ar@{_(->}[l] 0\ar@(ur,dr)^{0}\\
& &  M_1\ar@{=}[u] & &  M_2\ar@{=}[u] & &}\] 
Then $M_0/M_1 \simeq M_1/M_2 \simeq \xymatrix{k\ar@(ur,dr)^{0}}$.
Then $M$ has composition factors $\{S_1,S_1\}$, that is, the simple
module $S_1$ has multiplicity $2$. 

(2) Let $\Gamma\colon \xymatrix{ & 1\ar[dl]_\alpha \ar[dr]^\beta & \\
  2\ar[dr]_\gamma & & 3\ar[dl]^\delta \\
& 4 & }$ with relations $\rho = \{ \gamma\alpha - \delta\beta\}$.  Let
$\Lambda = k\Gamma/\langle \rho\rangle$, and let $M = \Lambda e_1$.
Find $l(M)$. The left $\Lambda$-module $M$ correspond to the
representation 
\[\xymatrix@C=10pt{ & e_1M\ar[dl]_{\alpha\cdot-} \ar[dr]^{\beta\cdot-} & \\
  e_2M\ar[dr]_{\gamma\cdot -} & & e_3M\ar[dl]^{\delta\cdot - } \\
& e_4M & }  \quad \simeq \quad 
\xymatrix@C=10pt{ & k\overline{e_1} \ar[dl]_{\alpha\cdot-} \ar[dr]^{\beta\cdot-} & \\
  k\overline{\alpha}\ar[dr]_{\gamma\cdot -} & & k\overline{\beta}\ar[dl]^{\delta\cdot - } \\
& k\overline{\gamma\alpha} = k\overline{\delta\beta} & } 
\quad \simeq \xymatrix@C=10pt{ & k\ar[dl]_{1} \ar[dr]^{1} & \\
  k\ar[dr]_{1} & & k\ar[dl]^{1} \\
& k & } 
\]
A composition series of $M$ is given by 
\[\xymatrix@C=6pt{
& k\ar[dl]_{1} \ar[dr]^{1} &     &                  & & 0\ar[dl]_{0}
\ar[dr]^{0} &     &                  & & 0\ar[dl]_{0} \ar[dr]^{0} &
&                  && 0\ar[dl]_{0} \ar[dr]^{0} &   & &  &  0\ar[dl]_{0}
\ar[dr]^{0} &  \\
k\ar[dr]_{1} & & k\ar[dl]^{1} & \supseteq & k\ar[dr]_{1} & &
k\ar[dl]^{1} & \supseteq & 0\ar[dr]_{0} & & k\ar[dl]^{1} & \supseteq &
0\ar[dr]_{0} & & 0\ar[dl]^{0} & \supseteq & 0\ar[dr]_{0} & & 0\ar[dl]^{0} \\
& k &                                    &                 & & k &
&                 & & k &                                    &
& & k &  & & & 0 & 
}\]
where we label the modules as $M = M_0 \supseteq M_1 \supseteq M_2
\supseteq M_3 \supseteq M_4 = (0)$. Then the factor modules $M_0/M_1$,
$M_1/M_2$, $M_2/M_3$ and $M_3/M_4$ correspond to the representations 
\[
\xymatrix@C=10pt{ & k\ar[dl]_{0} \ar[dr]^{0} & \\
  0\ar[dr]_{0} & & 0\ar[dl]^{0} \\
& 0 & }, \quad 
\xymatrix@C=10pt{ & 0\ar[dl]_{0} \ar[dr]^{0} & \\
  k\ar[dr]_{0} & & 0\ar[dl]^{0} \\
& 0 & }, \quad 
\xymatrix@C=10pt{ & 0\ar[dl]_{0} \ar[dr]^{0} & \\
  0\ar[dr]_{0} & & k\ar[dl]^{0} \\
& 0 & }, \quad 
\xymatrix@C=10pt{ & 0\ar[dl]_{0} \ar[dr]^{0} & \\
  0\ar[dr]_{0} & & 0\ar[dl]^{0} \\
& k & }
\]
It follows that $l(M) = 4$ with composition factors
$\{S_1,S_2,S_3,S_4\}$. 
\end{exam}

\begin{note}
The module $M$ has composition factors $\{S_1,S_2,S_3,S_4\}$. The
module $N = S_1\oplus S_2\oplus S_3\oplus S_4$ has the same
composition factors:
\[N = N_0 = S_1\oplus S_2\oplus S_3\oplus S_4 \supseteq N_1 = S_2\oplus
  S_3\oplus S_4 \supseteq N_2 = S_3\oplus S_4 \supseteq N_3 = S_4
  \supseteq N_4 = (0)\]
We have that $N_i/N_{i+1} \simeq S_{i+1}$ for $i = 0,1,2,3$.  So in
general, a module is not uniquely determined by its composition factors.
\end{note}
\begin{prop}
\label{Prop 11}
Given a \(\Lambda\)-module \(A\) of finite length and a \(\Lambda\)-homomorphism \(f:A\mapsto A\). The following are equivalent
\begin{enumerate}[(a)]
    \item \(f\) is an isomorphism
    \item \(f\) is a monomorphism (1-1)
    \item \(f\) is an epimorphism (onto)
\end{enumerate}
\end{prop}
\begin{proof}
Clearly \((a)\implies (b)\) and \((a)\implies (c)\) by def. We have the exact sequence
\begin{align*}
    0\rightarrow f(A) \hookrightarrow A \rightarrow A/f(A) \rightarrow 0
\end{align*}
\noindent\underline{\((a)\implies (b)\)}: \(f\) 1-1 \(\implies A \simeq f(A) \implies l(A) = l(f(A))
\implies l(A/f(A)) = 0 \implies f(A) = A \implies f \text{  onto}.\)\\[0.3cm]
\noindent\underline{\((c) \implies (a)\)}: \(f\) onto \(\implies f(A) = A \implies l(A/f(A)) = 0 \implies l(A) = l(f(A))\\
c\rightarrow \Ker f \hookrightarrow A \rightarrow f(A) \rightarrow 0 \text{  exact} \implies l(\Ker f) = 0 \implies \Ker f = (0) \implies f \text{ 1-1 } \implies f \text{ isomorphism}.\)
\end{proof}
\begin{rem}
The proof of \ref{Prop 11} holds for all \(f:A\mapsto B\) with \(l(A) = l(B)\), so if \(l(A) = l(B)\) and \(f:A\mapsto B\), then 
\begin{align*}
    f \text{ isomorphism } \Leftrightarrow f \text{ 1-1 } \Leftrightarrow f \text{  onto}.
\end{align*}
\end{rem}

\noindent\textbf{Recall}: \(0\rightarrow A \rightarrow B \rightarrow C \rightarrow 0\) exact and \(l(B) < \infty\), then \(l(A)\) and \(l(C)\) are finite too.
\begin{prop}
\label{Prop 12}
If \(\xymatrix{0\ar[r] & A\ar[r]^f & B\ar[r]^g & C\ar[r] & 0}\) is exact and \(A\) and \(C\) have finite length, then also \(B\) has finite length and \(l(B) = l(A) + l(C)\).
\end{prop}
\begin{proof}
Let 
\begin{align*}
    \mathcal{F}': A = A_0 \supseteq A_1 \supseteq\dots\supseteq A_{n-1}\supseteq A_n = (0)
\end{align*}
and
\begin{align*}
    \mathcal{F}'': C = C_0 \supseteq C_1 \supseteq\dots\supseteq C_{m-1}\supseteq C_m = (0)
\end{align*}
be two comp. series of \(A\) and \(C\), respectivley. Consider the following chain of submods of \(B\):\\[0.2cm]
\begin{align*}
    \mathcal{F}: B = g^{-1}(C) \supseteq g^{-1}(C_1) \supseteq g^{-1}(C_2) \supseteq\dots\supseteq g^{-1}(C_m) = \Ker g
\end{align*}
and
\begin{align}
    \Ker g = \Im f = f(A) \subseteq f(A_1) \subseteq f(A_2) \subseteq\dots\subseteq f(A_{n-1}) \subseteq f(A_n) = (0)
\end{align}
We want to show that \(\mathcal{F}\) is a comp. series of \(B\). Let \(g_i = g|_{g^{-1}C_i}: g^{-1}(C_i)\mapsto C_i \big(b_i\mapsto g(b_i)\big)\). Then \(g_i\) is clearly surjective (since \(g\) is). The composition
\begin{align*}
\xymatrix{\Psi_i:\Pi_i g_i: g^{-1}(C_i)\ar[r] & C_i\ar[r]^{\Pi_i} & C_i/C_{i+1}  \\
 & & c_i \mapsto c_i + C_{i+1}}
\end{align*} 
 onto (comp. of two onto maps) and we have \(b_i\in \Ker \Psi_i \Leftrightarrow \Pi_i(g_i(b_i)) = 0\) in \(C_i/C_{i+1} \Leftrightarrow g(b_i) + C_{i+1} = 0\) so \(g(b_i))\in C_{i+1} \Leftrightarrow b_i\in g^{-1}(C_{i+1})\implies \Ker\Psi_i = g^{-1}/C_{i+1})\implies g^{-1}(C_i)/g^{-1}(C_{i+1}) \simeq \Im \Psi_i\) since \(g^{-1}(C_{i+1}) = \Ker \Psi_i\) and \(C_i/C_{i+1}\) is simple by definition of \(\mathcal{F}''\)
 \end{proof}

Let \(f_i = f|_{A_i}:A_i\rightarrow f(A_i\) (i.e. \(a_i\mapsto
f(a_i)\)) which clearly is onto. The composition 
\begin{align*}
    \xymatrix{\theta_i = p_if_i:A_i\ar[r]^{f_i} & f(A_i)\ar[r]^{p_i} & f(A_i)/f(A_{i+1})}
\end{align*}
where \(p_i(b_i) = b_i + f(A_{i+1})\), is onto (composition of two onto maps) and we have
\begin{gather*}
    a_i\in\Ker \theta_i \Leftrightarrow p_if_i(a_i) = 0 \text{ in } f(A_i)/f(A_{i+1})\\
    \Leftrightarrow f(a_i) + f(A_{i+1}) = 0\\
    \Leftrightarrow f(a_i)\in f(A_{i+1})\\
    \exists a_{i+1}\in A_{i+1}\text{ such that } f(a_{i+1} = f(a_i) \text{   (because } f 1-1)\\
    \Leftrightarrow a_i = a_{i+1}\\
    \Leftrightarrow a_i\in A_{i+1}\\
    \implies \Ker\theta_i = A_{i+1}\\
    \implies A_i/\Ker\theta_i \simeq \Im\theta_i = f(A_i)/f(A_{i+1})
\end{gather*}
\(A_i/\Ker\theta_i = A_i/A_{i+1}\) is simple by definition of \(\mathcal{F}'\). Hence \(\mathcal{F}\) is a composition series of \(B\) and
\begin{align*}
    l(B) = l(A) + l(C)
\end{align*}

\begin{defin}
    A collection \(\mathcal{C}\) of modules (a full subcategory) is \underline{closed under extensions} if for each exact sequence 
    \begin{align*}
        \xymatrix{0\ar[r] & A\ar[r] & B\ar[r] & C\ar[r] & 0}
    \end{align*}
    with \(A, C\in\mathcal{C}\), then \(B\) is also in \(\mathcal{C}\)
\end{defin}
\index{Closed under extensions}
Let \(fl(\Lambda\) be the collection of \(\Lambda\)-modules of finite length.
\begin{prop}
\label{Proposition 13}
\begin{enumerate}[(a)]
    \item \(fl(\Lambda)\) is closed under extensions and contains the simples. Furthermore, \(fl(\Lambda)\) is closed under submodules and factor modules.
    \item Let \(\mathcal{C}\) be a collection of \(\Lambda\)-modules that is closed under extensions and contains the simple \(\Lambda\)-modules. Then \(fl(\Lambda)\subseteq\mathcal{C}\)
\end{enumerate}
\end{prop}
\begin{proof}(a): [FIX REFERENCES]\\
(b) Let \(B\in fl(\Lambda)\) with \(l(B) = n\). Induction on \(n\).\\
\noindent \(n = 1\): Then \(B\) is simple and \(B\in\mathcal{C}\).\\
\noindent\(n>1\): Choose \(0\not\subseteq A\not\subseteq B\) submodule. This is possible since \(B\) is not simple. Then
\begin{align*}
    \xymatrix{0\ar[r] & A\ar[r] & B\ar[r] & C\ar[r] & 0}
\end{align*}
is exact with \(l(A),l(B/A)<l(B) = n\). Induction \(\implies A, B/A\in\mathcal{C}\). \(\mathcal{C}\) is closed under extensions so \(B\in\mathcal{C}\) and hence \(fl(\Lambda) \subseteq \mathcal{C}\).
\end{proof}
\noindent\underline{Recall:} A module \(M\) is Noetherian (Artinian) if for every ascending (descending) chain of submodules of M:
\begin{gather*}
    M_1\subseteq M_2\subseteq\dots\subseteq M_n \subseteq M_{n+1} \subseteq\dots\subseteq M \\
    (M \supseteq M_1\supseteq M_2\supseteq\dots\supseteq M_n \supseteq M_{n+1} \supseteq\dots)
\end{gather*}
\(\exists n\) such that \(M_n = M_{n+1} = \dots\).

\noindent \(M\) is Noetherian (Artinian) if, and only if, every non-empty set of sub-modules of \(M\) has a maximal (minimal) element.

\begin{prop}
A \(\Lambda\)-module. \(l(A) < \infty \Leftrightarrow\) A is  Artinian and Noetherian.
\end{prop}
\begin{proof}\underline{\(\implies\):} Want to show: \(fl(\Lambda) \subseteq \text{art}(\Lambda) = \) collection of Artinian \(\Lambda\)-modules. Will use Proposition \ref{Proposition 13} (b). Let 
\begin{align*}
    \xymatrix{0\ar[r] & A\ar[r] & B\ar[r] & C\ar[r] & 0}
\end{align*}
be exact. We claim that \(A, C\in\text{art}(\Lambda) \implies B\in\text{art}(\Lambda)\). Consider a descending chain \( B = B_0 \subseteq B_1 \subseteq \dots\). As before we get induced chains of submodules in \(A\) and \(C\). 
\begin{gather*}
    A = A_0 \supseteq A_1 \supseteq\dots\supseteq A_i\supseteq\dots\quad\text{with } A_i = f^{-1}(B_i)\\
    C = C_0 \supseteq C_1 \supseteq\dots\supseteq C_i\supseteq\dots\quad\text{with } C_i = g(B_i)
\end{gather*}
and induced exact squence
\begin{align*}
    \xymatrix{0\ar[r] & A_i/A_{i+1}\ar[r]^{\bar{f}} & B_i/B_{i+1}\ar[r]^{\bar{g}} & C_i/C-{i+1}\ar[r] & 0}
\end{align*}
since \(A, C\in\text{art}(\Lambda)\) there exists \(N\) such that
\begin{gather*}
A_i = A_{i+1} \text{ and } C_i = C_{i+1}\quad i\geq N\\
\implies A_i/A_{i+1} = (0) \text{ and } C_i/C_{i+1} = (0) \quad i \geq N \\
\implies B_i/B_{i+1} = 0 \implies B_i = B_{i+1} \quad i\geq N\\
\implies B\in\text{art}(\Lambda) \\
\implies \text{art}(\Lambda) \text{ is closed under extensions.}
\end{gather*}
Clear that \(\text{art}(\Lambda)\) contains the simple \(\Lambda\)-modules: Proposition \ref{Proposition 13}(b)\(\implies  \text{fl}(\Lambda)\subseteq\text{art}(\Lambda)\)

\noindent\underline{Exercise:} Similarly, \(\text{fl}(\Lambda)\subseteq\text{noeth}(\Lambda)\), collection of noehterian \(\Lambda\)-modules. This implies that \(\text{fl}(\Lambda)\subseteq\text{art}(\Lambda)\cap\text{noeth}(\Lambda)\).\\[0.5cm]

\noindent\underline{\Large\(\Leftarrow\)}: Assume that \(B\neq (0)\) is Artinian and Noetherian. Since \(B\) is Artinian \(B\) has a simple submodule \(S\subseteq B\). (\(\mathcal{F} = \{U\subseteq B\,\, |\,\, U\neq (0)\}\) has a minimal element). Consider \(\mathcal{F}' = \{U\subseteq B\,\, |\,\, l(U) < \infty \}\). Then \(\mathcal{F}' \neq \emptyset\) since \(S\in\mathcal{F}'\). Since \(B\) is Noetherian \(\mathcal{F}'\) has a maximal element \(A\subseteq B\) and \(A \in \text{fl}(\Lambda)\). Assume that \(A\subsetneq B\), i.e. \(B/A\neq (0)\). \(B\) Artinian \(\implies B/A\) Artinian \(\implies \exists T\subseteq B/A\) simple submodule. Consider the natural projection \(p:B\rightarrow B/A\). Then 
\begin{align*}
    p|_{p^{-1}} :p^{-1}(T)\rightarrow T \quad(\subseteq B/A)
\end{align*}
is onto and \(\Ker p|_{p^{-1}} = A\). Hence \((p^{-1}(T))/A\simeq T\) and we have an exact sequence
\begin{align*}
    \xymatrix{0\ar[r] & A\ar[r] & p^{-1}(T)\ar[r] & T\ar[r] & 0}
\end{align*}
Now \(l(p^{-1}(T)) = l(A) + 1\), a contradiction and we conclude that \(A = B\) and \(B\in\text{fl}(\Lambda)\).
\end{proof}
\noindent\underline{Note: } 
\begin{enumerate}[(1)]
    \item If \(\Lambda\) is a ring with 1 then \(\Lambda\) is artinian if, and only if, \(l(_{\Lambda}\Lambda)<\infty\)\\
    \begin{proof} \underline{\(\Rightarrow\)}: \(\Lambda\) left Artinian \(\implies \Lambda\) left Noetherian \(\implies _\Lambda\Lambda\in\text{art}(\Lambda)\cap\text{noeth}(\Lambda)\implies l(_\Lambda\Lambda)<\infty\)\\
    \underline{\(\Leftarrow: \)} \(l(_\Lambda\Lambda) < \infty \implies _\Lambda\Lambda \in \text{art}(\Lambda)\cap\text{noeth}(\Lambda)\implies\Lambda\) left Artinian.
    \end{proof}
    \item \(\Lambda\) is left Artinian. \textit{Challenge:} \(\text{fl}(\Lambda) = \text{mrd(?)}(\Lambda)\) - finitely generated \(\Lambda\)-modules. 
    \item \(\Lambda = \mathbb{Z}, M = \mathbb{Z}/(n)\) \\
    \(n = p_1^{m_1}p_2^{m_2}\dots p_t^{m_t}\quad,\quad p_i\) different primes and \(m_i\geq 1\)\\
    \(m_{\mathbb{Z}/p}(M) = \begin{cases}m_i & \text{if } p = p_i \\ 0 & \text{otherwise}\end{cases}\)
\end{enumerate}
\begin{prop}
    \(\Lambda\) a ring, \(B\) semisimple \(\Lambda\)-module. TFAE:
    \begin{enumerate}[(a)]
        \item \(l(B) <\infty \)
        \item \(B\) is Artinian. 
        \item \(B\) is Noetherian
    \end{enumerate}
\end{prop}
\begin{proof}
    Excercise
\end{proof}

\section{Radical}

\begin{defin}
$\Lambda$ ring.  The \emph{(left) radical of $\Lambda$} is the left ideal
\[\mathfrak{r} = \rad \Lambda = \cap_{\mathfrak{m} \text{maximal left ideal}} \mathfrak{m}\]
(Also called the Jacobsen radical of $\Lambda$).
\end{defin}

\textbf{Know}: $\mathfrak{r}$ is a left ideal.

\textbf{Show}: $\mathfrak{r}$ is an ideal.

\begin{exam}
If $\Lambda$ is a division ring, then $\mathfrak{r} = (0)$. 
\end{exam} 

\begin{exam}
$\Lambda = \mathbb{Z}$, $\langle p\rangle$ -- maximal ideal if $p$ is
a prime.

\[\mathfrak{r} = \cap_{p \text{\ prime}} \langle p\rangle = \langle
n\rangle = (0)\]

\[\langle n\rangle \subseteq \langle p\rangle \Rightarrow p\mid n,
\forall p \text{\ prime} \Rightarrow n = 0.\]
\end{exam}

\begin{exam}
$\Lambda = \mathbb{Q}\times \mathbb{Q}$, $\mathfrak{m}_1 =
\mathbb{Q}\times (0)$, $\mathfrak{m}_2 = (0)\times\mathbb{Q}$ - both are
maximal ideals.

\[(0) = \mathfrak{m}_1 \cap \mathfrak{m}_2 \subseteq \mathfrak{r}
  \Rightarrow \mathfrak{r} = (0).\] 
\end{exam}
In general, if we can find a finite set of maximal ideals
$\{\mathfrak{m}_i\}_{i=1}^t$ such that $\cap_{i=1}^t \mathfrak{m}_i =
(0)$, then $\mathfrak{r} = (0)$. 

\begin{exer}
Show that $\Lambda$ semisimple $\Rightarrow$ $\rad\Lambda = (0)$.
\end{exer}

\begin{exam}
$\Gamma\colon \xymatrix{ & 1\ar[dl]^\alpha\ar[dr]^\beta & \\
2\ar[dr]^\gamma & & 3\ar[dl]^\delta \\
& 4 & 
}$, $\rho=\{\gamma\alpha - \delta\beta\}$, $k$ a field and $\Lambda =
k\Gamma/\langle \rho\rangle$.  What is $\rad\Lambda$?

\textbf{Know}: $1_\Lambda = \overline{e_1} + \overline{e_2} +
\overline{e_3} + \overline{e_4}$, where
$\overline{e_i}\cdot\overline{e_j} = \begin{cases} \overline{e_i},
  \text{\ if $i = j$}\\
0, i\neq j
\end{cases}$

\begin{exer}
\begin{align}
\Lambda & = \Lambda\overline{e_1} \oplus \Lambda\overline{e_2} \oplus \Lambda\overline{e_3} \oplus \Lambda\overline{e_4}\notag\\
\mathfrak{m}_1 & = \Lambda\{\alpha,\beta\} \oplus \Lambda\overline{e_2} \oplus \Lambda\overline{e_3} \oplus \Lambda\overline{e_4}\notag\\
\mathfrak{m}_2 & = \Lambda\overline{e_1} \oplus \Lambda\overline{\gamma} \oplus \Lambda\overline{e_3} \oplus \Lambda\overline{e_4}\notag\\
\mathfrak{m}_3 & = \Lambda\overline{e_1} \oplus \Lambda\overline{e_2}  \oplus \Lambda\overline{\delta} \oplus \Lambda\overline{e_4}\notag\\
\mathfrak{m}_4 & = \Lambda\overline{e_1} \oplus \Lambda\overline{e_2} \oplus \Lambda\overline{e_3} \oplus (0)\notag\\
\end{align}
\[\mathfrak{m}_1 \cap \mathfrak{m}_2 \cap \mathfrak{m}_3 \cap
  \mathfrak{m}_4 = \langle
  \overline{\alpha},\overline{\beta},\overline{\gamma},\overline{\delta}\rangle
  \xrightarrow{\text{?}} \rad\Lambda.\]
\end{exer}
\end{exam}

\begin{prop}\label{prop:radcharacterization}
For any ring $\Lambda$ and any $\lambda\in\Lambda$, the following are
equivalent.
\begin{enumerate}[\rm(i)]
\item $\lambda\in\rad\Lambda$, 
\item $1 - x\lambda$ is \emph{left invertible} for all $x\in \Lambda$
  (i.e.\ $\exists x'\in\Lambda$ such that $x'(1-x\lambda) = 1$), 
\item $\lambda S = (0)$ for any simple $\Lambda$-module $S$.
\end{enumerate}
\end{prop}
\begin{proof}
(i) $\Rightarrow$ (ii): Suppose $\exists x \in \Lambda$ such that $1 -
x\lambda$ is not left invertible with $\lambda\in\Lambda$.  Then
$\Lambda(1 - x\lambda)$ is a proper left ideal in $\Lambda$.  Any
proper left ideal is contained in a maximal left ideal.  If
$\lambda\in\mathfrak{m}$, then $1\in\mathfrak{m}$, contradiction. So
$\lambda\not\in\mathfrak{m}$ and in particular
$\lambda\not\in\rad\Lambda$. 

(ii) $\Rightarrow$ (iii): Suppose $\exists$ a simple $\Lambda$-module
$S$ such that $\lambda S \neq (0)$, i.e.\ $\exists 0\neq s \in S$ with
$\lambda s \neq 0$.  Have $(0)\neq \Lambda(\lambda s)\subseteq S$.  

$S$ simple $\Rightarrow$ $\Lambda \lambda s = S$.

Hence, $\exists x \in \Lambda$ such that $x\lambda s = s$
$\Rightarrow$ $(1 - x\lambda) s = 0$

If $1 - x\lambda$ is left invertible, then $s=0$ $\Rightarrow$
$1-x\lambda$ is not left invertible.

(iii) $\Rightarrow$ (i): Let $\mathfrak{m}$ be a maximal left ideal in
$\Lambda$.  Then $\Lambda/\mathfrak{m}$ is a simple left
$\Lambda$-module.  By assumption
\[\lambda\cdot \Lambda/\mathfrak{m} = (0),\]
in particular
\[\lambda(1 + \mathfrak{m}) = \lambda + \mathfrak{m} = \overline{0}\]
and $\lambda\in\mathfrak{m}$ for all maximal left ideals
$\mathfrak{m}$ in $\Lambda$.   Hence $\lambda\in\rad\Lambda$. 
\end{proof}

\begin{defin}
Let $M$ be a (left) $\Lambda$-module, and let 
\[\Ann_\Lambda(M) = \{\lambda\in\Lambda\mid \lambda m = 0, \forall
  m\in M\}.\]
The set $\Ann_\Lambda(M)$ is called the
\emph{annihilator}\index{module!annihilator}\index{annihilator} of
$M$. 
\end{defin}

\textbf{Note}: $\Ann_\Lambda(M)$ is a two-sided ideal in $\Lambda$. 

\begin{cor}
Given a ring $\Lambda$
\[\rad\Lambda = \cap_{S \text{\ simple}\\ \text{left
      $\Lambda$-module}} \Ann_\Lambda(S).\]
In particular, $\rad\Lambda$ is a two-sided ideal in $\Lambda$. 
\end{cor}
\begin{proof}
Follows from (i) $\Leftrightarrow$ (iii) in Proposition
\ref{prop:radcharacterization}. 
\end{proof}

Can we find $\rad\Lambda$ from this?

$S\simeq S' \Rightarrow \Ann_\Lambda(S) = \Ann_\Lambda(S')$. 

\begin{thm}[Nakayama Lemma]
Given a ring $\Lambda$ and a finitely generated $\Lambda$-module $M$.
If $\mathfrak{a}$ is an ideal in $\Lambda$ with $\mathfrak{a}\subseteq
\rad\Lambda$, then $\mathfrak{a} M = M$ implies that $M = (0)$. 
\end{thm}
\begin{proof}
Suppose that $M\neq (0)$ and $\mathfrak{a}M = M$.  Let
$\{m_1,m_2,\ldots,m_t\}$ be a minimal set of generators for $M$ as a
$\Lambda$.  Since $\mathfrak{a}M=M$, we have that 
\[m_1 = \sum_{i=1}^t \lambda_im_i\]
for $\lambda_i\in\mathfrak{a}\subseteq \rad\Lambda$. 
\[\Rightarrow (1 - \lambda_1)m_1 = \sum_{i=2}^t \lambda_im_i\]
Since $\lambda_1\in\mathfrak{a}\subseteq \rad\Lambda
\stackrel{\textrm{Proposition \ref{prop:radcharacterization}}}{\Rightarrow} 1 -
\lambda_1$ has a left inverse, say $u$.
\[\Rightarrow m_1 = u(1 - \lambda_1) m_1 = \sum_{i=2}^t
  u\lambda_im_i\]
\[\Rightarrow \text{\ $M$ can be generated by $\{m_2,\ldots,m_t\}$.  Contradiction!}\]
If $t=1$, then $M = (0)$.  Contradiction!  If $t > 1$, then we have a
contradiction to the choice of generating set
$\{m_1,m_2,\ldots,m_t\}$. $\Rightarrow \mathfrak{a} M \neq M$. 
\end{proof}

\begin{recall}
A left ideal $\mathfrak{a}\subseteq \Lambda$ is
\emph{nilpotent}\index{ideal!nilpotent}\index{nilpotent!ideal} if
$\exists n \geq 1$ such that $\mathfrak{a}^n = (0)$. 
\end{recall}

\begin{lem}
$\Lambda$ ring.
\begin{enumerate}[\rm(a)]
\item If $\Lambda$ is a left (right) artinian, then $\rad\Lambda$ is
  nilpotent. 
\item If $\mathfrak{a}\subseteq \Lambda$ is a nilpotent left ideal,
  then $\mathfrak{a} \subseteq \rad\Lambda$
\end{enumerate}
\end{lem}
\begin{proof}
(a) $\mathfrak{r} = \rad\Lambda$. 
\[\cdots \supseteq \mathfrak{r}^i \supseteq \mathfrak{r}^{i+1}
  \supseteq \cdots\]
is a descending chain of left ideals in $\Lambda$.

$\Lambda$ left artinian $\Rightarrow$ $\mathfrak{r}^m =
\mathfrak{r}^{m+1} = \cdots$ for some $m$
\[M=\mathfrak{r}^m = \mathfrak{r}^{m+1} = \mathfrak{r}\mathfrak{r}^m = \mathfrak{r}M\]
$\Lambda$ left artinian $\Rightarrow$ $\Lambda$ left noetherian.

$\mathfrak{r}^m = M \subseteq \Lambda$ left ideal $\Rightarrow$
$M=\mathfrak{r}^m$ finitely generated $\Lambda$-module.

Nakayama Lemma $\Rightarrow$ $\mathfrak{r}^m = (0)$ and $\rad\Lambda$
is nilpotent.

(b)  Assume that $\mathfrak{a}$ is a nilpotent left ideal in
$\Lambda$, say $\mathfrak{a}^n = (0)$ for some $n\geq 1$.  Let $a\in
\mathfrak{a}$. Then for all $x\in\Lambda$ we have $xa\in\mathfrak{a}$
and $(xa)^n=0$.

$\Rightarrow  (1 + (xa) + (xa)^2 + \cdots (xa)^{n-1} = 1 - (xa)^n = 1$

$\Rightarrow 1 -xa$ has a left inverse for all $x\in\Lambda$.

Proposition \ref{prop:radcharacterization} $\Rightarrow a\in\rad\Lambda \Rightarrow
\mathfrak{a}\subseteq \rad\Lambda$. 
\end{proof}

\begin{recall}
\begin{align}
\Lambda \textrm{\ semisimple}\index{ring!semisimple} & \Leftrightarrow {_\Lambda\Lambda}
                                \textrm{\ semisimple $\Lambda$-module}\notag\\
& \Leftrightarrow \Lambda\simeq M_{n_1}(D_1)\times
  M_{n_2}(D_2)\times\cdots \times M_{n_t}(D_t)\notag\\
& \hphantom{\Leftrightarrow}\qquad \qquad n_i\geq 1, D_i \textrm{\ division ring}, t
  < \infty\notag\\
& \Leftrightarrow \Lambda \textrm{\ left artinian and has no nilpotent
  (left) ideals}\notag
\end{align}
\end{recall}

\begin{thm}\label{thm:semisimplechar}
$\Lambda$ ring.

$\Lambda$ semisimple $\Leftrightarrow$ $\Lambda$ left artinian and
$\rad\Lambda = (0)$. 
\end{thm}
\begin{proof}
$\Mapsto$ 
\[\Lambda \textrm{\ semisimple} \Rightarrow \Lambda \textrm{\ left
  artinian}\Rightarrow \rad\Lambda \textrm{\ is nilpotent}\]
Using this we obtain:
\[\Lambda \textrm{\ semisimple} \Rightarrow \Lambda \textrm{\ has no non-zero nilpotent left ideals} 
\Rightarrow \rad\Lambda = (0)\]

$\Mapsfrom$ Assume that $\Lambda$ is left artinian with $\rad\Lambda =
(0)$

Lemma \ref{lem:} $\Rightarrow$ $\Lambda$ has no non-zero nilpotent
left ideals.

$\Rightarrow$ $\Lambda$ is semisimple.
\end{proof}

\begin{thm}
$\Lambda$ left artinian, $\mathfrak{r} = \rad\Lambda$.
Then
\begin{enumerate}[\rm(a)]
\item $\Lambda/\mathfrak{r}$ is a semisimple ring. 
\item A left $\Lambda$-module $M$ is semisimple if and only if
  $\mathfrak{r}M=(0)$. 
\item There are only finitely many non-isomorphic simple
  $\Lambda$-modules, and they all occur as direct summands of
  $\Lambda/\mathfrak{r}$. 
\item $\Lambda$ is left noetherian.
\end{enumerate}
\end{thm}
\begin{proof}
(a) $\Lambda$ left artinian $\Rightarrow$ $\Lambda/\mathfrak{r}$ left
artinian.

$\rad(\Lambda/\mathfrak{r}) = (\rad\Lambda)/\mathfrak{r}$.

Theorem \ref{thm:semisimplechar} $\Rightarrow$ $\Lambda/\mathfrak{r}$
is semisimple. 

(b)--(d):  Exercise, see the book Proposition 3.1 page 9. 
\end{proof}

\begin{recall}
$\Lambda$ left artinian $\Leftrightarrow$ $l(_\Lambda\Lambda) <
\infty$. 
\end{recall}

\begin{cor}
$\Lambda$ ring. TFAE:
\begin{enumerate}[\rm(a)]
\item $\Lambda$ left artinian.
\item Every finitely generated $\Lambda$-module has finite length.
\item $\mathfrak{r} = \rad\Lambda$ is nilpotent and
  $\mathfrak{r}^i/\mathfrak{r}^{i + 1}$ is finitely generated
  semisimple $\Lambda$-module for all $i\geq 0$. 
\end{enumerate}
\end{cor}
\begin{proof}
(b) $\Rightarrow$ (a): In particular, $\Lambda$ as a left
$\Lambda$-module has finite length. 

$\Rightarrow$ $\Lambda$ is left artinian (and left noetherian). 

(a) $\Rightarrow$ (c):  $\mathfrak{r}$ is nilpotent by Lemma
\ref{lem:} (a), say $\mathfrak{r}^n = (0)$. 
\begin{center}
\begin{tabular}{rl}
Theorem \ref{thm:} (d) $\Rightarrow$ & $\Lambda$ left noetherian.\\
$\Rightarrow$ & $\mathfrak{r}^i$ finitely generated for all $i$ (as a
left ideal)\\
$\Rightarrow$ & $\mathfrak{r}^i/\mathfrak{r}^{i+1}$ finitely generated for all $i$ (as a
left ideal)\\
\end{tabular}
\end{center}
Theorem \ref{thm:} (b) $\Rightarrow$
$\mathfrak{r}^i/\mathfrak{r}^{i+1}$ semisimple $\Lambda$-module for
all $i\geq 0$, since $\mathfrak{r}\cdot \mathfrak{r}^i/\mathfrak{r}^{i+1}=(0)$

(c) $\Rightarrow$ (b):  Suppose $\mathfrak{r}^n = (0)$ for some $n\geq
1$.  Consider: $\Lambda \supseteq \mathfrak{r}\supseteq \mathfrak{r}^2
\supseteq \mathfrak{r}^3\supseteq \cdots \supseteq
\mathfrak{r}^{n-1}\supseteq \mathfrak{r}^n =(0)$.

\label{2017-pages}

In particular, we have exact sequences 


\end{proof}
	
%page 40
\section{Radical of a module}
\begin{defin}
$\Lambda$ ring, $A \subseteq B$ two $\Lambda$-modules. Then 
$A$ is \emph{small} in $B$ if $A+X=B$ implies that $X=B$ for every submodule $X$ of $B$. 
\end{defin}

\begin{exam}
\begin{enumerate}
\item[(1)] $\Lambda=\mathbb{Z}$ and $B=\Lambda$ then the only small
  submodule of $B$ is $(0)$. If $(0)\neq A \subsetneq B$, then $A =
  \mathbb{Z}n$ for some $n \neq 0,\; 1$. Choose an integer $m \neq 0,
  \; 1$ such that $\gcd(n, m) = 1$. Then $B = \mathbb{Z}n +
  \mathbb{Z}m = A + X$ but $X \neq B$. Hence $A$ is not small. 

\item[(2)] $\G \colon \xymatrix{1\ar[r]^\alpha & 2\ar[r]^\beta & 3}$, $k$ a field $\Lambda = k\G$, $B=k\G e_1$\\
$\xymatrix{
							&					&			& && A\ar@{}[d]|-*[@]{=} &&\\
							& k \ar@{->}[d]^1 	&			&0\ar@{->}[d]&&0\ar@{->}[d]&&0\ar@{->}[d]\\
B = k\G e_1	\ar@{~>}[r] 	& k \ar@{->}[d]^1 	&\supseteq 	&k\ar@{->}[d]^1&\supseteq&0\ar@{->}[d]&\supseteq&0\ar@{->}[d]\\
						 	& k 				&			&k&&k&&0\\
}$\\
There are no other proper subrepresentations of $B$.

$\left.\begin{matrix}
\xymatrix@C-2pc@R-1pc{
(i) & A &+&(0)& =& A &\neq& B\\
(ii) & A&+&A&=&A & \neq & B\\
&&& 0\ar@{->}[d] && 0\ar@{->}[d] &&\\
(iii) & A &+& k\ar@{->}[d]^1 &=& k\ar@{->}[d]^1 &\neq &B\\
&&& k && k &&
}
\end{matrix}\right\rbrace \implies A$ is small in $B$.
\\
\item[(3)] $\G \colon \vcenter{\xymatrix
{& 1\ar[dr]\ar[dl] && 4\ar[dl]\\
2 && 3 &}}, \Lambda = k\G,$ 
$$B = 
\vcenter{\xymatrix@C-1pc
{& k\ar[dr]^1\ar[dl]_1 && k\ar[dl]_1\\
k && k &}} 
\supset 
\vcenter{\xymatrix@C-1pc
{& k\ar[dr]^1\ar[dl]_1 && 0\ar[dl]\\
k && k &}} 
= A' \supset 
\vcenter{\xymatrix@C-1pc
{& 0\ar[dr]\ar[dl] && 0\ar[dl]\\
k && 0 &}}
=A$$
\begin{exer}
\begin{enumerate}
\item[]
\item[$\cdot$]$A$ small in $B$
\item[$\cdot$]$A'$ not small in $B$
\end{enumerate}
\end{exer}
\end{enumerate}
\end{exam}

%page 41
\begin{defin}
$\Lambda$ a ring, $B$ a $\Lambda$-module
$$\rad B = \bigcap_{A \text{\ max.\ submod.\ of }B} A = \text{the radical of }B$$
\begin{note}
$\rad_\Lambda \Lambda = $ the radical of $\Lambda$ as a ring.
\end{note}
\end{defin}

\begin{prop}\label{prop:25}
$\Lambda$ a ring, $B$ a finitely generated $\Lambda$-module. Then
$A \subseteq B$ is small in $B \iff A \subseteq \rad B$
\begin{proof}
  $\Leftarrow :$ Assume $A \subseteq \rad B$. Let $X \subsetneq B$.
  WTS: $A+X \neq B$.\\ 
  Consider $\mathfrak{F} = \{ M \mid M \subsetneq B \text{ submodule
  }, X \subseteq M \}$. Then $\mathfrak{F} \neq \emptyset$, since
  $X \in \mathfrak{F}$.
 Let $\{ C_\alpha \}_{\alpha \in I}$ be a chain of submodules  of $B$
 in $\mathfrak{F}$. Let $U = \bigcup_{\alpha \in I} C_\alpha$, then $U$
 is a submodule of $B$. If $U = B$ each element in a set of generators
 $\{ b_1, b_2, \ldots, b_n \}$ of $B$ must be in one $C_\alpha$. Say
 $b_i \in C_{\alpha_i}$. The chain condition implies that $\{ b_1,
 b_2, \ldots, b_n \} \subseteq C_\alpha$ for some $\alpha \in I
 \implies C_\alpha = B$. Contradiction! This implies that each chain in
 $\mathfrak{F}$ has an upper bound in $\mathfrak{F}$.  Then  
 Zorn's Lemma implies $\mathfrak{F}$ has a maximal element $B_1$, i.e.
 $B_1$ is a maximal submodule of $B$.  Then $A \subseteq \rad B
 \subseteq B_1$ and $X \subseteq B_1$, so that $A+X \subseteq B_1 \subsetneq
 B$.   Hence $A$ is small in $B$.

 $\Rightarrow:$ Suppose that $A \not\subseteq \rad B$, that is,
  $\exists $ maximal submodule $B_1 \subseteq B$ such that
  $A \not\subseteq B_1$. Then $B_1 \not\subseteq A + B_1 \subseteq B$,
  and consequently $A+B_1 = B$ (since $B_1$ is maximal)
  $B_1 \subsetneq B \implies A$ is not small in $B$.
\end{proof}
\end{prop}

%page 42
\begin{thm}\label{thm:26}
$\Lambda$ left artinian, $A$ a finitely generated $\Lambda$-module.
Then $\rad A=\mathfrak{r}A$ where $\mathfrak{r} = \rad \Lambda$. 
\begin{proof}
1) $\mathfrak{r}A \subseteq \rad A$: WTS: $\mathfrak{r}A$ is small in $A$.

Then Proposition \ref{prop:25} $\implies \mathfrak{r}A \subseteq \rad
A$.

Let $X$ be a submodule of $A$ and suppose that $\mathfrak{r}A + X =
A$. 
\[\xymatrix@C-1pc@R-1pc{
\implies & \mathfrak{r}^2A &+& \mathfrak{r}X &=& \mathfrak{r}A\\
\implies & \mathfrak{r}^2A &+& \relax\underbrace{\mathfrak{r}X + X}\ar@{}[d]|-*[@]{=} &=& A\\
\implies & \mathfrak{r}^2A &+& X &=& A
}\]
\textbf{Induction}: $\mathfrak{r}^nA + X = A$ for all $n \geq 1$.\\
%Lemma 19 page 35
Lemma \ref{lem:19} $\implies \mathfrak{r}$ nilpotent $\implies X = A
\implies \mathfrak{r}A$ is small in $A \implies \mathfrak{r}A
\subseteq \rad A$. 

2) $\rad A \subseteq \mathfrak{r}A$:
The module $A/\mathfrak{r}A$ is a semisimple module since
$\mathfrak{r} A/\mathfrak{r}A = (0)$ (Theroem \ref{thm:21} (b)). 
We have that $A/\mathfrak{r}A = \bigoplus_{i=1}^t S_i$, for simple
$\Lambda$-modules $S_i$. Let $A_j =\bigoplus_{i=1, i\neq j}^t S_i
\subseteq A/\mathfrak{r}A$, which is a maximal submodule. Furthermore
$\bigcap_{j=1}^t A_j = (0)$, so that $\rad(A/\mathfrak{r}A) = (0)$.
Then 1) implies that $\mathfrak{r}A$ is contained in all maximal submodules of $A$\\
$\implies \rad (A/\mathfrak{r}A) = (\rad (A)+\mathfrak{r}A)/\mathfrak{r}A = (0)$\\
$\implies \rad A \subseteq \mathfrak{r}A$\\
1) and 2) $\implies \rad A = \mathfrak{r}A$.
\end{proof} 
\end{thm}

\begin{exam}
$\G \colon 
\vcenter{\xymatrix@C-1pc@R-1pc
{
& 1\ar[dl]_\alpha\ar[dr]^\beta & \\
2\ar[dr]_\gamma && 3\ar[dl]^\delta\\
& 4 &
}}, 
\rho = \gamma\alpha-\delta\beta, k$ field, $\Lambda = k\G / \langle \rho \rangle$.
\\
We have seen: $\mathfrak{r} = \langle \overline{arrows} \rangle = \overline{J}$.\\
$\xymatrix@R-1pc
{
&& k\ar[dl]_1\ar[dr]^1 & \\
\Lambda \overline{e_1}\ar@{~>}[r]\ar@{}[dd]|-*[@]{\supseteq} & k\ar[dr]_1 && 3\ar[dl]^1\\
&& 4 &
\\
&& 0\ar[dl]_0\ar[dr]^0 & \\
\rad \Lambda \overline{e_1} = \mathfrak{r}(\Lambda \overline{e_1}) = \mathfrak{r}\overline{e_1}\ar@{~>}[r] &  k\ar[dr]_1 && k\ar[dl]^1\\
&& k &
}$
\end{exam}
%page 43
\section{The radical of representation}
$(\G, \rho)$ quiver with relations, $J^t \subseteq \langle \rho \rangle \subseteq J^2$, $k$ field, $\Lambda = k\G / \langle \rho \rangle$, $\G_0 = \{ 1, 2, \ldots , n \}$, $\mathfrak{r} = \overline{J}$.\\
\\
$\xymatrix
{
(V, f) \text{ representation of } (\G, \rho)\ar@{}[d]|-*[@]{\supseteq}\ar@{~>}[r] & M_{(V, f)} = V(1) \oplus V(2) \oplus \cdots \oplus V(n) \ar@{}[d]|-*[@]{\supseteq}\\
(V', f') \text{ radical of } (V, f)\ar@{<~}[r] & \mathfrak{r}M_{(V, f)} = \{ r_1m_1 + \cdots + r_tm_t \mid r_i \in \mathfrak{r}, m_i \in M_{(V, f)} \}\\
}
$\\
$\mathfrak{r}$ generated by the arrows $\implies \mathfrak{r}M_{(V,f)}$ is generated by elements on the form
$\xymatrix@C-3pc@R-1pc{\beta \cdot (v_1, v_2, \ldots v_n) = (0, \ldots, 0, & f_\beta(v_r), & 0, \ldots , 0 )\\
&s\text{-th coordinate}\ar[u]&}$ for $\beta \colon r \rightarrow s \in \G_1$\\
$$\implies \overline{e_s}\mathfrak{r}M_{(V, f)} = \sum_{\begin{smallmatrix}
\beta \in \G_1,\\ e(\beta)=s
\end{smallmatrix}} \Image f_\beta$$\\
$\implies V'(i) =\overline{e_i}\mathfrak{r}M_{(V, f)}= \sum_{\begin{smallmatrix}
\beta \in \G_1,\\ e(\beta)=i
\end{smallmatrix}} \Image f_\beta$ and\\ $f'_\alpha = f_\alpha \mid_{V'(i)} \colon V'(i) \rightarrow V'(j)$ for $\alpha\colon i \rightarrow j$.\\
The range is by definition OK since $\Image f_\alpha \subseteq V'(j)$.

\begin{exam}
\begin{enumerate}
\item[(1)] $\G \colon \xymatrix{1 \ar[r]^\alpha & 2\ar[r]^\beta & 3 }, k$ field, $\Lambda = k\G$.\\
$\xymatrix@R-1pc{
&k\ar[d]^1\\
\Lambda e_1 \ar@{~>}[r]&k\ar[d]^1\\
&k
}
\;\;\;\; 
\xymatrix@R-1pc{
&0\ar[d]\\
\rad (\Lambda e_1) \ar@{~>}[r]&k\ar[d]^1\\
&k
}$
\\
$\xymatrix@R-1pc{
&0\ar[d]\\
\Lambda e_2 \ar@{~>}[r]&k\ar[d]^1\\
&k
}
\;\;\;\; 
\xymatrix@R-1pc{
&0\ar[d]\\
\rad (\Lambda e_2) \ar@{~>}[r]&0\ar[d]\\
&k
}$\\
$\xymatrix@R-1pc{
&0\ar[d]\\
\Lambda e_3 \ar@{~>}[r]&0\ar[d]\\
&k
}
\;\;\;\; 
\xymatrix@R-1pc{
&0\ar[d]\\
\rad (\Lambda e_3) \ar@{~>}[r]&0\ar[d]\\
&0
}$

%page 44
\item[(2)] $\G \colon \xymatrix{1 \ar[r]^\alpha & 2 \ar@(ur, dr)^\beta}, \rho = \{ \beta^2 \}, \Lambda = k\G / \langle \rho \rangle$\\
$\Lambda e_1 \colon \xymatrix{k \ar[r]^{\begin{smallmatrix} 1\\0 \end{smallmatrix}}  & k^2 \ar@(ur, dr)^{\begin{smallmatrix} 0&0\\1&0 \end{smallmatrix}}}, \rad (\Lambda e_1) \colon \xymatrix{0 \ar[r] & k^2 \ar@(ur, dr)^{\begin{smallmatrix} 0&0\\1&0 \end{smallmatrix}}}$\\
$\Lambda e_2 \colon \xymatrix{0 \ar[r]  & k^2 \ar@(ur, dr)^{\begin{smallmatrix} 0&0\\1&0 \end{smallmatrix}}}, \rad (\Lambda e_2) \colon \xymatrix{0 \ar[r] & (0) \oplus k \ar@(ur, dr)^0}$\\
\end{enumerate}
\end{exam}

\begin{note}
In general, for two $\Lambda$-module $M$ and $N$, then $\rad(M \oplus
N) = \rad M \oplus \rad N$. 
\end{note}

\begin{defin} 
$\Lambda$ left artinian, $\mathfrak{r} = \rad \Lambda$, $A$ finitely
generated $\Lambda$-module. Then $A / \mathfrak{r}A$ is called
\emph{the top of $A$}.  
\end{defin}
\begin{exam}
\begin{enumerate}
\item[(1)] $\G \colon \xymatrix{1 \ar[r]^\alpha & 2\ar[r]^\beta & 3 }, k$ field, $\Lambda = k\G$.\\
\begin{equation*}
\begin{split}
A = \Lambda e_1 \colon& \vcenter{\xymatrix@R-1pc{
k\ar[r]^1 &
k\ar[r]^1 &
k
}}\\
\mathfrak{r}A \colon& \vcenter{\xymatrix@R-1pc{
0\ar[r] &
k\ar[r]^1 &
k
}}\\
A/\mathfrak{r}A \colon& \vcenter{\xymatrix@R-1pc{
k\ar[r] &
0\ar[r] &
0\ar@{~>}[r] & e_1 \Lambda e_1 = ke_1
}}
\end{split}
\end{equation*}

\item[(2)] $\G \colon \xymatrix{1 \ar[r]^\alpha & 2 \ar@(ur, dr)^\beta},
  \rho = \{ \beta^2 \}, \Lambda = k\G / \langle \rho \rangle$. Then 
\[\xymatrix
{A = \Lambda \overline{e}_1  \ar@{~>}[r] & 
k\ar[r]^{\begin{smallmatrix} 1\\0 \end{smallmatrix}} & 
k^2 \ar@(ur, dr)^{\begin{smallmatrix} 0&0\\1&0 \end{smallmatrix}}
} 
\supseteq 
\xymatrix
{
0\ar[r] & 
k^2 \ar@(ur, dr)^{\begin{smallmatrix} 0&0\\1&0 \end{smallmatrix}}
}\]
$\xymatrix
{
A/\mathfrak{r}A = \overline{e}_1\Lambda \overline{e}_1 = k\overline{e}_1  \ar@{~>}[r] & 
k\ar[r] & 0 \ar@(ur, dr)^{0}
}$\\
\end{enumerate}
\end{exam}

In general, if $\Lambda$ is left artinian and $A$ is finitely
generated, then 
\[\xymatrix@C-2pc@R-1pc{
A\ar[rrr] &&& A/\mathfrak{r}A &=& S_1 & \oplus &\cdots& \oplus & S_t\\
        &&&                 & & x_1' \neq 0 \ar@{}[u]|-*[@]{\in} &&&& x_t' \neq 0 \ar@{}[u]|-*[@]{\in} &
      }\]
is semisimple, each $S_i$ simple.
Choose $\{ x_1, x_2, \ldots, x_t \}$ inverse images of $x_i'$ in $A$.
For $a \in A$, then $\exists \lambda_i \in \Lambda$ such that $a -
\sum_{i=1}^t \lambda_i x_i \in \mathfrak{r}A$. This means that 
$$a - \sum_{i=1}^t \lambda_i x_i = \sum_{j=1}^n r_ja_j, $$
for some $r_j \in \mathfrak{r}$ and $a_j \in A$. 
Let $A' = \Lambda\{ x_1, \ldots , x_t \} \subseteq A$ be the submodule
generated by $\{ x_1, \ldots , x_t \}$ of $A$. Then 
$$\mathfrak{r}(A/A') = A/A'$$
and by the Nakayama Lemma $A/A' = 0$, or equivalently $A = A'$ is
generated by $\{ x_1, \ldots , x_t \}$ (or use $\mathfrak{r}^m = (0)
\ldots$). 

%page 45
\begin{lem}\label{lem:f_onto_iff_f-bar_onto}
$\Lambda$ left artinian, $f\colon A \to B$ some $\Lambda$-homomorphism, and
$A$ and $B$ finitely generated $\Lambda$-modules. Then
$$ f\colon A \to B \text{ is onto} \iff \overline{f}\colon A/\mathfrak{r}A \to B/\mathfrak{r}B \text{ is onto.} $$
\end{lem}
\begin{proof}
Let $f\colon A \to B$, then $f(\mathfrak{r}A) = \mathfrak{r}f(A) \subseteq
\mathfrak{r}B$ and the following diagram commutes
$$\xymatrix{
a\ar@{|->}[d] & A\ar[r]^f\ar[d]^{p_A} & B\ar[d]^{p_B} & b\ar@{|->}[d]\\
a + \mathfrak{r}A & A/\mathfrak{r}A\ar[r]^{\overline{f}} & B/\mathfrak{r}B & b + \mathfrak{r}\\
& a + \mathfrak{r}A\ar@{|->}[r] & f(a) + \mathfrak{r}B
}
$$
$\underline{\Rightarrow} :$ Assume that $f\colon A \to B$ is onto. Since
$f$ and $p_B$ are onto, the map $p_B \circ f = \overline{f} \circ p_A$
is onto. This implies that $\overline{f}$ is onto.

$\underline{\Leftarrow} :$ Assume $\overline{f}\colon A \to B$ is onto. The
elements of $\Image \overline{f}$ are $f(a) + \mathfrak{r}B$ for some
$a\in A$. Given $b \in B$, then $\exists a \in A$ such that
$b + \mathfrak{r}B = f(a) + \mathfrak{r}B$. This implies that
$b - f(a) \in \mathfrak{r}B$, and therefore
$B = \Image f + \mathfrak{r}B$. Since $\mathfrak{r}B = \rad B$
(Theorem \ref{thm:26}) is small in $B$ (Proposition \ref{prop:25}), we
have $\Im f = B$ and $f$ is onto.
\end{proof} 

\begin{note}
Only used that $B$ was finitely generated.
\end{note}

\begin{defin}
$f \colon A \to B$ is an \emph{essential epimorphism} if $f$ is and
epimorphism, and if $g\colon X \to A$ is such that $f \circ g \colon X \to B$ is
onto then $g\colon X \to A$ is onto. 
\end{defin}
\begin{exam}
  (1) Let $f\colon A \oplus B \to A$ be given as $f(a,b) = a$ for two
  $\Lambda$-modules $A$ and $B$. Is $f$ an essential epimorphism? The
  map $f$ is an epimorphism. Consider $g\colon X= A \to A \oplus B$,
  $g(a)=(a, 0)$. Then $f \circ g(a) = f(a,0)=a$, which implies that $f
  \circ g$ is onto. If $B \neq (0)$, then $g$ is not onto and we infer
  that $f$ is not an  essential epimorphism.

(2) Let $\G \colon \xymatrix{1\ar[r]^\alpha & 2\ar[r]^\beta & 3}$ for a
field $k$ field, and let $\Lambda = k\G$. Then we have
\[\xymatrix{
&                             k\ar[d]^1 && 0\ar[d]\\
A = \Lambda e_1 \ar@{~>}[r] & k\ar[d]^1 &\supseteq& 0\ar[d]\\
 & k\ar@{}[d]|-*[@]{=:} && k\ar@{}[d]|-*[@]{=:}\\
 & A && B
}\]
Let $f\colon A \to A/B$ be the natural epimorphism/projection. Let $g\colon X
\to A$ and assume $g\colon X \to A$ is not onto.

\underline{WTS:} $f \circ g$ not onto.\\

All proper submodules of $A$ are: 
$\vcenter{\xymatrix@C-2pc@R-1pc
{
0\ar[d] && 0\ar[d] && 0\ar[d]\\
k\ar[d]^1 &\supseteq& 0\ar[d] &\supseteq& 0\ar[d]\\
k && k && 0\\
}}$\\
\\
$g$ not onto $\implies \Im g \subseteq \vcenter{\xymatrix@C-2pc@R-1pc
{
0\ar[d] \\
k\ar[d]^1 \\
k \\
}}$\\
$\implies \Im f \circ g \subseteq f\left(\vcenter{\xymatrix@C-2pc@R-1pc
{
0\ar[d] \\
k\ar[d]^1 \\
k \\
}}\right) 
=
\left.
\vcenter{\xymatrix@C-2pc@R-1pc
{
0\ar[d] \\
k\ar[d]^1 \\
k \\
}}
\middle/
\vcenter{\xymatrix@C-2pc@R-1pc
{
0\ar[d] \\
0\ar[d] \\
k \\
}}
\right.
\simeq
\vcenter{\xymatrix@C-2pc@R-1pc
{
0\ar[d] \\
k\ar[d] \\
0 \\
}}
\subsetneq 
A/B
=
\vcenter{\xymatrix@C-2pc@R-1pc
{
k\ar[d]^1 \\
k\ar[d] \\
0 \\
}}
$\\

$\implies f \circ g$ not onto $\implies f$ essential epimorphism.

(3) Let $\Lambda$ be left artinian, and let $A$ be a finitely
generated $\Lambda$-module.

\underline{Claim:} $p_A \colon A \to A / \mathfrak{r}A$ essential epimorphism.
\begin{proof}
Let $g\colon X \to A$, and assume that $f \circ g$ is onto. We have the
following commutative diagram:
$$
\xymatrix{
X\ar[d]^g \ar[r]^{p_X} & X/\mathfrak{r}X \ar[d]^{\overline{g}}\\
A\ar[r]^{p_A} & A/\mathfrak{r}A\\
}
$$
\underline{Know:} $g\colon X \to A$ onto $\iff \overline{g}\colon X / \mathfrak{r} X \to A / \mathfrak{r}A$ is onto (Lemma \ref{lem:f_onto_iff_f-bar_onto})\\
$p_A \circ g = \overline{g}\circ p_X$ and $p_A \circ g$ onto $\implies
\overline{g}$ onto $\implies g\colon X \to A$ onto \\
$\implies p_A$ essential epimorphism.
\end{proof}
\end{exam}

\begin{exer}
$\left.
\begin{matrix}
f\colon A \to B & \text{ ess. epi}\\
g\colon B \to C & \text{ ess. epi}
\end{matrix}  
\right\rbrace \implies g\circ f \colon A \to C$ essential epimorphism.
\end{exer}

%page 47

\begin{prop}\label{prop:28}
  Let $\Lambda$ be a left artinian algebra and $A$ and $B$ finitely
  generated $\Lambda$-modules. Let $f \colon A \to B$ be onto. TFAE: 
\begin{enumerate}
\item[(a)] $f$ is an essential epimorphism,
\item[(b)] $\Ker f \subseteq \mathfrak{r}A$, $(\mathfrak{r} = \rad
  \Lambda)$, 
\item[(c)] $\overline{f}\colon A / \mathfrak{r}A \to B / \mathfrak{r}B$ is an isomorphism.
\end{enumerate}
\begin{proof}
$\underline{(a)\Rightarrow(b):}$ Assume $f$ is an essential epimorphism. \underline{WTS:} $\Ker f$ is small in $A$, i.e. $\Ker f \subseteq \mathfrak{r}A$ (Proposition \ref{prop:25} and Theorem \ref{thm:26})\\ %mener du theorem 26 ikke 27? side 42
Let $X \subseteq A$. Assume that $\Ker f + X = A$. Then the composition $\xymatrix{
X\ar@{^{(}->}[r]^i & A\ar[r]^f & B
}$ is onto (proof?), such that $\xymatrix{
X\ar@{^{(}->}[r]^i & A
}$ is onto, since $f$ is an essential epimorphism.\\
$\implies X = A \implies \Ker f$ is small in $A \implies \Ker f \subseteq \mathfrak{r}A$\\

$\underline{(b)\Rightarrow(c):}$ Assume that $\Ker f \subseteq \mathfrak{r}A$. \\
We have
$\vcenter{\xymatrix@R-2pc{
A/\Ker f \ar@{}[r]|-*[@]{\xrightarrow{\sim}}^{f'} & \Im f = B\\
a + \Ker f \ar@{|->}[r] & f(a)
}}$ ($f$ onto)\\
and therefore $\mathfrak{r}(A / \Ker f) = \mathfrak{r}A / \Ker f \xrightarrow{\sim}^{f'} \mathfrak{r}B$ and $
\left.\left(A / \Ker f \right) \middle/\left(\mathfrak{r}A / \Ker f
  \right) \right. \simeq B/\mathfrak{r}B$.  We have the following
commutative diagram
$$\xymatrix@C8pc{
a + \mathfrak{r}A \ar@{|->}[r] \ar@{|->}@/_6pc/[ddd] & f(a) + \mathfrak{r}B\\
A / \mathfrak{r}A \ar[r]^{\overline{f}} \ar@{}[d]|-*[@]{\simeq} & B / \mathfrak{r}B\\
\left.\left(A / \Ker f \right) \middle/\left(\mathfrak{r}A / \Ker f \right) \right. \ar[ur]^{\overline{f'}\sim} & f(a) + \mathfrak{r}B\\
(a + \Ker f) + \mathfrak{r}A / \Ker f \ar@{|->}[ur]
}$$
$\overline{f}$ is the composition of two isomorphisms $\implies \overline{f}$ is an isomorphism.\\

$\underline{(c) \Rightarrow (a):}$ Assume that $\overline{f}\colon
A/\mathfrak{r}A \to B / \mathfrak{r}B$ is an isomorphism. Let $g\colon X
\to A$. Then we have the following commutative diagram:
%page 48
$$\xymatrix{
X\ar[r]^g \ar[d]^{p_X} & A\ar[r]^f \ar[d]^{p_A} & B \ar[d]^{p_B}\\
X /\mathfrak{r} X \ar[r]^{\overline{g}} & A/\mathfrak{r}A\ar@{}[r]|-*[@]{\xrightarrow{\sim}}^{\overline{f}} & B/\mathfrak{r}B\\
}$$
Assume $f \circ g$ is onto\\ $\implies \overline{f \circ g} =
\overline{f} \circ \overline{g}\colon X /\mathfrak{r} X \to B /\mathfrak{r} B$ is onto.\\ 
$\implies \overline{g} =
\overline{f}^{-1}\overline{f}\overline{g}\colon X /\mathfrak{r} X \to A /\mathfrak{r} A$ is onto.\\ 
Lemma \ref{lem:f_onto_iff_f-bar_onto} $\implies g$ is onto $\implies f$ is essential epimorphsim.
\end{proof}
\end{prop}

\begin{exam}
$\G \colon 
\vcenter{\xymatrix@C-1pc@R-1pc{
&1\ar[ld]_\alpha \ar[rd]^\beta&\\
2\ar[rd]_\gamma&&3\ar[ld]^\delta\\
&4&
}}
\rho = \{ \gamma\alpha - \delta\beta \}, k$ field, $\Lambda = k\G /
\langle \rho \rangle$. 
Let $B = 
\vcenter{\xymatrix@C-1pc@R-1pc{
&k\ar[ld]_1 \ar[rd]&\\
k\ar[rd] && 0 \ar[ld]\\
&0&
}}.$ 
Find some $f\colon A \to B$ which is essential epimorphism.\\

\[
\xy
\POS(15,5)
\xymatrix"A"@C-1pc@R-1pc{
A=\Lambda \overline{e_1}
}
\POS(42,0)
\xymatrix"Adi"{
&k\ar[ld]_1 \ar[rd]^1&\\
\ar@{<~}["A"u] k\ar[rd]_1 && k \ar[ld]^1\\
&k&
}

\POS(84,0)

\xymatrix"Bdi"{
&k\ar[ld]_1 \ar[rd] \ar@/_1pc/@{<.}["Adi"]_1&\\
\ar@/_1pc/@{<.}["Adi"]_1 k\ar[rd] && 0 \ar[ld] \ar@/^1pc/@{<.}["Adi"]_0\\
&0 \ar@/^1pc/@{<.}["Adi"]_0&
}

\POS(10, -15)
\xymatrix"kerf"{
&k\ar[ld]_1 \ar[rd]& \ar@{^{(}->}["Adi"dll]\\
 k\ar[rd] && 0 \ar[ld]\\
&0&
}

\POS(45, -32)
\xymatrix"rA"@C-1pc@R2pc{
& A\ar@{}["Adi"dd]|-*[@]{=} & \\
 & \mathfrak{r}A \ar@{}[u]|-*[@]{\subseteq} & \\
&&
}

\POS(22, -36)
\xymatrix"ker"@C-1pc@-1pc{
& & \\
 & \Ker f\ar@{}["rA"]|-*[@]{\subseteq} \ar@{}["kerf"d]|-*[@]{=} & \\
&&
}

\POS(60, -36)
\xymatrix"rAdi"@C1pc{
&0\ar[ld] \ar[rd]& \\
 *+[l]{=\;\; k}\ar[rd]_1 && k \ar[ld]^1\\
&k&
}

\endxy
\]

$\implies f$ is essential epimorphism, or equivalently 
\[
\xymatrix{
                                     &           &k\ar[dr]\ar[dl]&                                                &&           &k\ar[dr]\ar[dl]\\
A / \mathfrak{r}A \ar@{}[r]|-*[@]{=} & 0 \ar[dr] && 0 \ar[dl]  \ar@{}[rr]|-*[@]{\xrightarrow{\sim}}^{\overline{f}} &&  0 \ar[dr] && 0 \ar[dl] \ar@{}[r]|-*[@]{=} & B / \mathfrak{r}B\\
                                     &           &0&                                                &&           &0\\
}
\]

\end{exam}

%page 49
\section{Projective Modules}
$\Lambda$ a ring, $P$ a $\Lambda$-module
\begin{defin}
$P$ is \underline{projective} if for every epimorphism $g: B \to C$ and every  $\Lambda$-homomorphism $f: P \to C$ there exists a homomorphism $h: P \to B$ such that
$$\xymatrix{
& P\ar[d]^f \ar@{-->}[ld]_{\exists h}\\
B \ar[r]^g & C
}$$ commutes.
\end{defin}

\begin{exam}
$\Lambda$ is a projective $\Lambda$-module
$$\xymatrix{
& \Lambda\ar[d]^f \ar@{-->}[ld]_{\exists h} & 1\ar@/_2pc/@{|->}[dd]|\hole\\
B \ar[r]^g & C \ar[r] & 0\\
b\ar@{|->}[rr] && f(1)
}$$
$f(\lambda) = f(\lambda \cdot 1) = \lambda f(1)$. Choose $b \in B$ such that $g(b) = f(1)$, and define $h: \Lambda \to B$ by $h(\lambda) = \lambda b$\\
\underline{Check:} $h$ is $\Lambda$-homomorphism.\\
Then $gh(\lambda) = g(\lambda b) = \lambda g(b) = \lambda f(1) = f(\lambda \cdot 1) = f(\lambda), \forall \lambda \in \Lambda$\\
$\implies \Lambda$ projective.
\end{exam}

\begin{exer}
$F$ free $\Lambda$-module $\implies F$ is projective ($F \simeq \bigoplus_{i \in I} \Lambda a_i$, $_\Lambda \Lambda a_i \simeq _\Lambda \Lambda$, $\forall i \in I$)
\end{exer}

\begin{prop}
$\Lambda$ a ring, $P$ a projective $\Lambda$-module\\
$P$ projectiv $\iff$ $\exists$ free $\Lambda$-module $F$ and a $\Lambda$-module $Q$ such that $F \simeq P \oplus Q$
\begin{proof}
$\underline{\Rightarrow :}$ Assume that $P$ is projective. Any module $M$ is a factor of a free module $F_M : $ Let $F_M = \bigoplus_{m \in M} \Lambda_m$, $_\Lambda\Lambda_m = _\Lambda\Lambda$ and define $\varphi: F_M \to M$ by $\varphi((\lambda_m)_{m \in M}) = \sum_{m \in M} \lambda_m m$\\
%page 50
\underline{Check:} $\varphi$ $\Lambda$-homomorphism, $\varphi$ onto, $m = \varphi((\lambda_x)_{x \in M})$, where $\lambda_x = \begin{cases}
0, & x \neq m\\
1, & x = m
\end{cases}$ 

Let $g: F \to P$ be onto with $F$ free.
$$\xymatrix{
& P\ar@{-->}[ld]_h \ar@{}[d]^{1_P}|-*[@]{=} & \text{projective} \ar[l]\\
F \ar[r]^g & C \ar[r] & 0\\
b\ar@{|->}[rr] && f(1)
}$$
$\implies \exists h: P \to F$ such that $gh = 1_P$\\

\underline{Calim:} $F \simeq \Ker g \oplus \Im h$ (Excersice)\\

$\underline{\Leftarrow :}$ Assume that $F \xrightarrow{\sim}^\varphi P \oplus Q$ where $F$ is a free $\Lambda$-module. Suppose that $g: B \to C$ is onto nd let $f: P \to C$.
$$
\xymatrix{
(p, q)\ar@{|->}[r] & p\\
*+[l]{F \xrightarrow{\sim}^\varphi P \oplus Q} \ar[r]^\pi \ar@{-->}[d]_{\exists h'} & P \ar@/^1pc/@{-->}[l]^\nu \ar[d]^f\\
B \ar[r]^g & C \ar[r] & 0
}
$$
$\nu(p) = (p, 0)$, $\pi\nu(p) = \pi(p,0) = p$\\
$F$ projective $\implies \exists h': F \to B$ such that 
\begin{equation*}
\begin{split}
gh' &= f\pi \varphi \;\;\;\;\;\;\; \mid \cdot \varphi^{-1}\\
gh'\varphi^{-1} &= f\pi \;\;\;\;\;\;\;\;\; \mid \cdot \nu\\
g(\underbrace{h'\varphi \nu}_h) &= f\phi\nu = f 1_P = f
\end{split}
\end{equation*}

$\implies P$ is projective.
\end{proof}
\end{prop}

\begin{exam}
$(\G, \rho)$ a quiver with relations, $k$ a field, $J^t \subseteq \langle \rho \rangle \subseteq J^2$, $\G_0 = \{ 1, 2, \cdots , n \}$. $\Lambda = k\G / \langle \rho \rangle$\\
\underline{Have seen:} $_\Lambda\Lambda = \Lambda \overline{e_1} \oplus \Lambda \overline{e_2} \oplus \cdots \oplus \Lambda \overline{e_n}$\\
$\implies \Lambda \overline{e_i}$ projective $\Lambda$-modules
\end{exam}

%page 51
\begin{defin}
Let $f: P \to M$ be a $\Lambda$-homomorphism. Then $f:P\to M$ is a \underline{projective cover of $M$} if P is projectiv and $f$ is an essential epimorphism. 
\end{defin}

\begin{exam}
$\G:
\vcenter{\xymatrix@C-1pc@R-1pc{
&1\ar[dl]_\alpha \ar[dr]^\beta&\\
2 \ar[dr]_\gamma&&3 \ar[dl]^\delta\\
&4&
}},$ 
$\rho = \{ \gamma\alpha -\delta\beta \}$, $\Lambda = k\G / \langle \rho \rangle$, $M = \vcenter{\xymatrix@C-1pc@R-1pc{
&k\ar[dl]_1 \ar[dr]&\\
k \ar[dr]&&0 \ar[dl]\\
&0&
}}$

\[
\xy
\POS(0,0)
\xymatrix"seq"@C5pc{
0 \ar[r] & \Ker f \ar[r] & \Lambda \overline{e_1} \ar[r]^f & M \ar[r] & 0
}
\POS(20, -15)
\xymatrix@C-4pc@R-1pc{
&0\ar[dl] \ar[dr] \ar@{}["seq"]|-*[@]{=}&\\
0 \ar[dr]&&k \ar[dl]^1\\
&k&
}
\POS(50, -15)
\xymatrix{
&k\ar[dl]_1 \ar[dr]^1 \ar@{}["seq"r]|-*[@]{=}&\\
k \ar[dr]_1&&k \ar[dl]^1\\
&k&
}\POS(78, -15)
\xymatrix{
&k\ar[dl]_1 \ar[dr] \ar@{}["seq"rr]|-*[@]{=}&\\
k \ar[dr]&&0 \ar[dl]\\
&0&
}

\endxy
\]
We have $\Ker f \subseteq \mathfrak{r}(\Lambda \overline{e_1})$\\
$\implies f$ is essential epimorphism\\
$\implies f$ is projective cover.
\end{exam}

\begin{thm}
$\Lambda$ left artinian, $A$ finitely generated
\begin{enumerate}
\item[(a)] There exists a projective cover $f:P\to A$ ($P$ finitely generated)
\item[(b)] Let $f_1 : P_1 \to A$ and $f_2 : P_2 \to A$ be two projective covers of $A$. Then there exists an isomorphism $g : P_1 \to P_2$ such that $f_1 = f_2g$
\[
\xymatrix@R-2pc{
P_1\ar@{}[dd]_g|-*[@]{\xrightarrow{\sim}} \ar[dr]^{f_1}\\
& A\\
P_2 \ar[ur]_{f_2}
}
\]
\end{enumerate}

\begin{proof}
\begin{enumerate}
\item[(a)]$A$ finitely generated $\implies \exists$ onto $f: \Lambda^n \to A$, $n < \infty$.\\
Choose $P$ projective, $f : P \to A$ an onto $\Lambda$-homomorphism with $l(P)$ minimal. \underline{WTS:} $f$ projective cover. Consider
\[
\xymatrix{
0\ar[r] & \Ker f \ar@{^{(}->}[r]  & P \ar[r]^f \ar[d]^{\pi_P} & A \ar[r] \ar[d]^{\pi_A} & 0\\
  &        & P/\mathfrak{r}P \ar[r]^{\overline{f}} & A/\mathfrak{r}A \ar[r] & 0
}
\]
Assume that $\Ker f \not\subseteq \mathfrak{r}P$. Then $\pi_P(\Ker f) \neq (0)$ in $P/\mathfrak{r}P$, which is semisimple. We know that
$ P/\mathfrak{r}P = \pi_P(\Ker f) \oplus X $
for some $X \subseteq P/\mathfrak{r}P$.\\
%page 52
Define $e : P/\mathfrak{r}P \to P/\mathfrak{r}P$ by letting $e(v, w) = (0, w)$ for $(v, w) \in P/\mathfrak{r}P = \pi_P(\Ker f) \oplus X$.

\[
\xymatrix{
P\ar@{-->}[r]^{e'} \ar[d]^{\pi_P} & P \ar[d]^{\pi_P}\\
P/\mathfrak{r}P \ar[r]^e & P/\mathfrak{r}P \ar[d]\\
&0
}
\]
Then $\pi_P e'e' = ee\pi_P = e\pi_P = \pi_P e' \implies \pi_P(e'-e'e') = 0$. Hence, if $x = e'(1_P - e') = (1_P-e')e'$ then $\Im x \subseteq \mathfrak{r}P$. Since $\Im x^2 \subseteq \mathfrak{r}^2 P$ and so on, and $\mathfrak{r}^t = (0)$ for some $t \geq 1$, we have that $x^t = 0$. Let $a = e'$ and $b = 1_P - e'$. Then $$a^tb^t = b^ta^t = x^t = 0$$
\centerline{and}
$$1_P = (a+b)^{2t} = \underbrace{a^{2t} + r_1a^{2t-2}b + \cdots r_ta^tb^t}_{\tilde{e}} + \underbrace{r_{t+1}a^{t-1}b^{t+1} + \cdots + r_{2t-1}ab^{2t-1} + b^{2t}}_{\tilde{e}'}$$
It follows that $\tilde{e} = \tilde{e}\cdot 1_P = \tilde{e}(\tilde{e} + \tilde{e}') = \tilde{e}^2 + \tilde{e}\tilde{e}' = \tilde{e}^2 + 0 = \tilde{e}^2$\\
$\implies \tilde{e}$ is aan idempotent.\\
\underline{Check:} $\overline{\tilde{e}} = e^{2t} = e$ and $\overline{f}(P/\mathfrak{r}P) = \overline{f}(X) = \overline{f}(\Im e)$.\\
\leavevmode\\
Then we have 
\[
\xymatrix{
P \ar[r]^{\tilde{e}} \ar[d]_{\pi_P} & P \ar[r]^{f} \ar[d]_{\pi_P} & A \ar[d]_{\pi_A}\\
P/\mathfrak{r}P \ar[r]^e & P/\mathfrak{r}P \ar[r]^{\overline{f}} & A/\mathfrak{r}A
}
\]
\[\xymatrix@R-2pc{
*+[F_\}]{\overline{f}e \pi_P} \ar@{}[r]|-*[@]{=} & *+[F_\}]{\pi_A f \tilde{e}}\\
\text{onto} \ar@{}[r]|-*[@]{\implies} & \text{onto} 
}\]
\underline{Seen:} $\pi_A$ essential epimorphism $\implies f \tilde{e}$ onto $\implies f \mid_{\Im \tilde{e}}: \Im \tilde{e} \to A$ onto.\\

$$\xymatrix@R-1pc@C-1pc{
\text{We have } & P \ar@{}[r]|-*[@]{=} & \Im \tilde{e} \ar@{}[r]|-*[@]{\oplus} & \Ker \tilde{e}\\
\text{Check:} && \text{projective} \ar[u] & \Im(1_P - \tilde{e}) \ar@{}[u]|-*[@]{=}
}$$\\
%page 53
$\overline{(1_P - \tilde{e})} = 1 - e \neq 0 \implies \Im (1_P \tilde{e}) \neq (0)$.\\
$\implies l(\Im \tilde{e}) < l(P)$, Contradiction!\\
$\implies \Ker f \subseteq \mathfrak{r}P$ and $f$ is an essential epimorphsim.\\
$f$ is projective cover.

\item[(b)] Let $f_i : P_i \to A$ be a projective cover of $A$ for $i = 1, 2$.
\[\xymatrix{
P_1 \ar[dr]^{f_1} \ar[d]_{\exists g}&& 0\\
P_2 \ar[r]^{f_2} \ar[d]_{\exists h} & A \ar[r]\ar[dr]\ar[ur] & 0\\
P_1 \ar[ur]_{f_1} && 0
}\]
$P_1$ projective and $f$ epimorphic $\implies \exists g: P_1 \to P_2$ such that $f_1 = f_2g$.\\
$f_2$ essential epimorphism and $f_1$ epimorphic $\implies g$ epimorphic.\\
Similarly $h$ is epimorphic.\\
$\implies l(P_2) \leq l(P_1) \leq l(P_1) \implies l(P_1) = l(P_2) \implies g$ isomorphism. 
\end{enumerate}
\end{proof}
\end{thm}

\begin{prop}
$\Lambda$ left artinian, $f: P \to A$ an epimorphism, $A$ finitely generated, $P$ projective $\Lambda$-module.
\begin{enumerate}
\item[(a)] $f:P \to A$ projective cover $\iff \overline{f}: P/\mathfrak{r}P \to A/\mathfrak{r}A$ isomorphic.
\item[(b)] $\{ f_i : P_i \to A_i \}_{i=1}^m$, $f_i$ onto, $P_i$ projective\\
The map $f: P_1 \oplus P_2 \oplus \cdots \oplus P_m \to A_1 \oplus A_2 \oplus \cdots \oplus A_m $ defined by $(p_1, p_2, \cdots , p_m) \mapsto (f_1(p_1), f_2(p_2), \cdots , f_m(p_m))$ is a projectiv cover $\iff$ each $f_i:P_i \to A$ is a projectiv cover.
\end{enumerate}

\begin{proof}
\begin{enumerate}
\item[]
%prop 28 page 48
\item[(a)] Prop \ref{•}
\item[(b)] \underline{Exercise:} Use that $\mathfrak{r}(P_1 \oplus \cdots \oplus P_m) = \mathfrak{r}P_1 \oplus \cdots \oplus \mathfrak{r}P_m$, $\mathfrak{r} = \rad \Lambda$ and $(a)$
\end{enumerate}
\end{proof}
\end{prop}
%page 54

\printindex
\begin{thebibliography}{ARS}
\bibitem{ARS} Auslander, M., Reiten, I., Smal\o, S.\ O.,
  \emph{Representation theory of Artin algebras}. Corrected reprint of
  the 1995 original. Cambridge Studies in Advanced Mathematics,
  36. Cambridge University Press, Cambridge, 1997. xiv+425 pp. ISBN:
  0-521-41134-3; 0-521-59923-7.
\end{thebibliography}
\end{document})
