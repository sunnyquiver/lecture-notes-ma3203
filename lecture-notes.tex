\documentclass{amsart}
\usepackage{amsmath, amsthm, amscd, amssymb, latexsym, mathtools, centernot, faktor}
\usepackage[all]{xy}
\usepackage{imakeidx,enumerate}

\numberwithin{equation}{section}
\newtheorem{thm}{Theorem}[section]
\newtheorem{cor}[thm]{Corollary}
\newtheorem{lem}[thm]{Lemma}
\newtheorem{prop}[thm]{Proposition}
\theoremstyle{definition}
\newtheorem{defin}[thm]{Definition}
\newtheorem{rem}[thm]{Remark}
\newtheorem{exam}[thm]{Example}
\newtheorem{note}[thm]{Note}

\newcommand{\G}{\Gamma}
\renewcommand{\L}{\Lambda}

\newcommand{\extto}{\xrightarrow}
\newcommand{\Ker}{\operatorname{Ker}\nolimits}
\renewcommand{\Im}{\operatorname{Im}\nolimits}
\newcommand {\defeq}{\stackrel{\mathclap{\normalfont\mbox{def}}}{=}}

\usepackage{hyperref}
\makeindex

\begin{document}
\title{Lecture notes for MA3203 Ring theory}

\author{\O yvind Solberg}
\address{Department of Mathematical Sciences\\
NTNU\\
N-7491 Trondheim, Norway}
\email{oyvind.solberg@math.ntnu.no}

\maketitle
\tableofcontents

\input{page1-5}
\section{Modules}
\begin{exam}
$\Gamma\colon \xymatrix{1\ar[r]^\alpha & 2}$, $k$ field. 
	
What is a module over $k\G$? 

Let $M$ be a left $k\G$-module.
	
\underline{Recall:} 
$1_{k\G}=e_1 + e_2$, 
\[e_ie_j=  \begin{cases} \ e_i^2 = e_i,\\
  e_ie_j = 0, \text{\ for $i\neq j$}.
\end{cases}\]
\textbf{Claim}: $M = e_1M\oplus e_2M$ as vector space over $k$.
\begin{proof}
  \begin{align}
    m & = 1_{k\G}*m\notag\\
        & =(e_1 + e_2)m\notag\\
        & = e_1m+e_2m \in e_1M + e_2M\notag\\
\implies M & \subseteq e_1M + e_2M \subseteq M\notag\\
 \implies M & = e_1M + e_2M\notag
\end{align}
 Let  $m \in e_1M \cap e_2M$, that is, $m=e_1m'=e_2m''$. Then
\begin{align}
 e_1m & = e_1(e_1m')\notag\\
          & = (e_1e_1)m'\notag\\
          & = e_1m' = m\notag\\
          & = e_1(e_2m'')=\underbrace{(e_1e_2)}_{=0}m''=0\cdot
          m''=0\notag
\end{align}
$\implies m = 0$. Hence $e_1M \cap e_2M = (0)$.

\noindent $\implies M = e_1M \oplus e_2M$. 
\end{proof}

Let \(m\in M\).  Then 
\begin{align}
e_1m & = e_1(e_1m + e_2m)\notag\\
         & = e_1^2m + (e_1e_2)m\notag\\
         & = e_1m\notag\\
\intertext{and}\notag\\
 e_2m & = e_2(e_1m + e_2m) = e_2m\notag,\\
\alpha m & = \alpha(e_1m + e_2m)\notag\\
     & = \alpha(e_1m) + \alpha(e_2m)\notag\\
     & = (\alpha e_1)m + (\alpha e_2)m\notag\\
     & = (\alpha e_1)m + 0\cdot m\notag\\
     & = e_2(\alpha e_1) m = (e_2\alpha)e_1m\notag\\
     & = e_2(\alpha e_1 m)\in e_2M\notag
\end{align}

\noindent \(M \xrightarrow{\alpha\cdot-} M \leadsto\) linear map
\(f_\alpha\colon e_1M\xrightarrow{\alpha\cdot-} e_2M\)\\
\(M\xrightarrow{e_1\cdot-}M\leadsto\) linear map, projection \(M\to e_1M\)\\
\(M\xrightarrow{e_2\cdot-}M\leadsto\) linear map, projection \(M\to e_2M\)\\

\begin{center}
\(e_1M \xrightarrow{\alpha\cdot -} e_2M\)
\end{center}
is a representation of \(\Gamma\) over \(k\): A vector space in each
vertex and a linear map for the arrow.
\end{exam}

Given $V\xrightarrow{f} V'$, two vector spaces $V,V'$ over \(k\) and
\(f\) a linear map. How can we construct a left \(k\Gamma\)-module?\\ 
From above: \(M = V\oplus V'\) as a vector space. Let \(m = (v, v')\), then\\
\begin{align}
e_1m & \defeq (v, 0)\notag\\
e_2m & \defeq (0, v)\notag\\
\alpha m & \defeq (0, f(v))\notag
\end{align}
\underline{Check}: \(M\) becomes a left \(k\Gamma\)-module!
\begin{defin}
  A \emph{representation}\index{representation} \((V, f)\) of a quiver
  \(\Gamma = (\Gamma_0, \Gamma_1)\) over a field \(k\) is a collection
  of vector spaces \(\{V(i)\}_{i\in\Gamma_0}\) over \(k\) and
  \(k\)-linear maps \(f_\alpha\colon V(i)\to  V(j)\) for each arrow
  \(\alpha\colon i\to  j\) in \(\Gamma_1\). (We assume that
  \(\text{dim}_kV(i) < \infty\) for all \(i\in\Gamma_0\)).
\end{defin}
\begin{exam}
 \(\Gamma\colon 1.\) A representation of \(\Gamma\) over \(k\) is just
 a vector space over \(k\).
\end{exam}
\begin{exam}
\(\Gamma\colon 1\xrightarrow{\alpha}2\). Representation \(V(1)\xrightarrow{f_\alpha}V(2)\). For example \\
    \(k\xrightarrow{1}k\quad\quad k\xrightarrow{0}0\quad\quad 0\xrightarrow{0}k\)\quad\quad
    \(k^2\xrightarrow{\tiny\begin{pmatrix} 1 & 2\\ 0 & 3 \\ -1 & 1 \end{pmatrix}}k^3\) \\
\end{exam}
\begin{exam}
\(\Gamma\colon \xymatrix{ & 1\ar[dl]_\alpha \ar[dr]^\beta & \\
    2\ar[dr]_{\gamma} & & 3 \ar[dl]^\delta \\
    & 4 & }
    \quad\quad \text{Representation: }\xymatrix{ & V(1)\ar[dl]_{f_\alpha} \ar[dr]^{f_\beta} & \\
    V(2)\ar[dr]_{f_\gamma} & & V(3) \ar[dl]^{f_\delta} \\
    & V(4) & }\)\\
    For example:\\
\(\xymatrix{ & k^2\ar[dl]_{\small\begin{pmatrix} 1 & 0 \end{pmatrix}} \ar[dr]^{\small\begin{pmatrix} 1 & -1 \\ 0 & 1 \end{pmatrix}} & \\
    k\ar[dr]_{\small\begin{pmatrix} 1 \\ 1 \end{pmatrix}} & & k^2 \ar[dl]^{\small\begin{pmatrix} 0 & 0\\ 0 & 1 \end{pmatrix}}\\
    & k^2 & }\)\quad\quad
    \(\xymatrix{ & k\ar[dl]_{1} \ar[dr]^{1} & \\
    k\ar[dr]_{1} & & k \ar[dl]^{1}\\
    & k & }\quad\quad
\xymatrix{ & k\ar[dl]_{1} \ar[dr]^{1} & \\
    k\ar[dr]_{1} & & k \ar[dl]^{0}\\
    & k & }\)
\end{exam}
\subsection{Maps between representations}
\begin{exam}
\(\Gamma\colon 1\xrightarrow{\alpha}2\), \(k\) a field.\\
Let \(f\colon M\xrightarrow{}N\) be a homomorphism of left \(k\Gamma\)-modules. Then
\begin{align}
f(e_1m)  & = f((e_1e_1)m) \notag\\
              & = f(e_1(e_1m)) \notag\\
              & = e_1f(e_1m) \in e_1N\notag
\end{align}
\[\implies f|_{e_1M}:e_1M \to e_1N.\]
Similarly, \(f|_{e_2M}\colon e_2M\xrightarrow{}e_2N\). Furthermore,
\begin{align}
     \alpha f|_{e_1M}(e_1m) & = \alpha f(e_1m)\notag\\
                                       & = f(\alpha(e_1m)) \notag\\
                                       & = f|_{e_2M}(\alpha e_1m)\notag
\end{align}
since $\alpha = e_2\alpha$. 
\end{exam}
\noindent Hence
\[\xymatrix{e_1M\ar[r]^{f|_{e_1M}}\ar[d]^{\alpha\cdot-} & e_1M\ar[d]^{\alpha\cdot\Large-} \\
    e_2M\ar[r]^{f|_{e_2M}} & e_2M}\]
is a commutative diagram. 
\begin{rem}
\( f\left(\begin{tabular}{c}
     \(1 - 1\) \\ onto \\ isom.
\end{tabular}\right)\)\(\quad\Leftrightarrow\quad\) \(f|_{e_iM}\left(\begin{tabular}{c}
     \(1 - 1\) \\ onto \\ isom.
\end{tabular}\right)\quad\) for all \(i\).
\end{rem}
\begin{defin}
Let \((V, f)\) and \((V', f')\) be two representations of \(\Gamma\)
over \(k\). A \emph{homomorphism}\index{representation!homomorphism}
\(h:(V, f)\to  (V', f')\) is a collection of linear maps 
\begin{align*}
    h(i)\colon V(i) \to  V'(i)
\end{align*}
for all \(i\in\Gamma_0\), such that \(\forall\alpha :i\to  j\in\Gamma_1\) the following diagram commutes:
\begin{align*}
  \xymatrix{V(i)\ar[rr]^{h(i)}\ar[dd]^{f_\alpha} & & V'(i)\ar[dd]^{f'_\alpha} \\
 & \ar@(ur,dr) & \\
 V(j)\ar[rr]^{h(j)}& & V'(j)}  
\end{align*}
i.e.\ \(f'_\alpha h(i) = h(j)f_\alpha\) for all \(\alpha\in\Gamma_1\). 
\end{defin}

\begin{note}
\(h\) is a(n) isomorphism, monomorphism, epimorphism if
\(h(i)\colon V(i)\to V'(i)\) are all isomorphisms, monomorphisms,
epimorphisms respectively.
\end{note}

\begin{exam}
\begin{enumerate}[(1)]
\item \(\Gamma\colon \xymatrix{1\ar[r]^{\alpha}& 2}, k\) is a field
    \begin{enumerate}[(a)]
        \item $\xymatrix{k\ar@{-->}[r]^{a\cdot-}\ar[d]^{1} & k\ar[d]^{0} \\
        k\ar@{-->}[r]^{0}\ar@{=}[d] & k\ar@{=}[d]\\
        (V, f) & (V', f')}$
        Here \(h(1) = a\cdot-\) and \(h(2) = 0\) so \(h = (a\cdot-, 0)\)
 \item $\xymatrix{k\ar@{-->}[r]^0="a"\ar[d]^1 & 0\ar[d]^0 \\
        k\ar@{=}[d]\ar@{-->}[r]^0="b" &  k\ar@{=}[d]\ar@{=>} "a";"b"\\
        (V, f) & (V', f') }$ No non-zero homomorphisms
\item
  $\xymatrix{k^2\ar@{-->}[r]^{\small\begin{pmatrix}1&0\\0&1 \end{pmatrix}}\ar[d]_{\small\begin{pmatrix}0&1\\1&0 \end{pmatrix}}
    & k^2\ar[d]^{\small\begin{pmatrix}0&1\\1&0 \end{pmatrix}} \\
   k^2\ar@{=}[d]\ar@{-->}[r]_{\small\begin{pmatrix}0&1\\1&0
       \small\end{pmatrix}} 
   &  k^2\ar@{=}[d]\\
      (V, f) & (V', f') }$
        \(h = \left(\begin{pmatrix}1&0\\0&1\end{pmatrix}, \begin{pmatrix}0&1\\1&0\end{pmatrix}\right)\) is an isomorphism.
    \end{enumerate}
\item \(\Gamma\colon \xymatrix{ & 1\ar[dl]_\alpha\ar[dr]^\beta & \\
                            2\ar[dr]_\gamma & & 3\ar[dl]^\delta\\
                            & 4 & }
                            \quad
                            \xymatrix{ & k\ar[dl]_1\ar[dr]^1 & & & k\ar[dl]_1\ar[dr]^1 & \\
                            k\ar[dr]_1 & & k\ar[dl]^1\ar@/_1pc/[rrr] & k\ar[dr]_1 & & k\ar[dl]^0\\
                            & k\ar@{=}[d]\ar@/_1pc/[rrr]^0  & & & k\ar@{=}[d] & \\
                            &(V, f) & & & (V', f')& }\)
                            here we have no isomorphism between \((V, f)\) and \((V', f')\).
\end{enumerate}
\end{exam}

\subsection{Modules and representations}

$\Gamma = (\Gamma_0,\Gamma_1)$ - quiver, $k$ field.

$M$ left $k\Gamma$-module $\leadsto\begin{cases}
(V,f) \text{\ representation of $\Gamma$}\\
V(i) = e_iM\\
\text{for\ } \alpha\colon i\to j \in \Gamma_1, \text{\ we have\ }  f_\alpha\colon
V(i)=e_iM\extto{\alpha\cdot -} e_jM = V(j)\\
f_\alpha(e_im) = \alpha e_im
\end{cases}$

$(V,f)$ representaion of $\Gamma\leadsto \begin{cases}
M = \oplus_{i\in\Gamma_0} V(i)\\
m = (v_1,v_2,\ldots, v_n) \in M\\
e_im \extto{\text{def}}{=} (0,\ldots,0,v_i,0,\ldots,0)\\
\text{for $\alpha\colon i\to j$ in $\Gamma_1$, remember $\alpha =
  e_j\alpha e_i$}\\
\alpha m \extto{\text{def}}{=} (0,\ldots,0,f_\alpha(v_i),0,\ldots,0)
\text{\ with $f_\alpha(v_i)$ in the $j$-th coordinate}
\end{cases}$

\begin{exam}
$\Gamma\colon \xymatrix{1\ar[r]^\alpha & 2}$, $k$ field.

$(V,f)\colon \xymatrix{k\ar[r]^1 & k} \leadsto M = k\oplus k = k^2$

\begin{align}
e_1\cdot (a,b) & = (a, 0)\notag\\
e_2\cdot (a,b) & = (0,b)\notag\\
\alpha\cdot(a,b) & = (0,a)\notag
\end{align}
Note: $k\Gamma e_1 = k\{ e_1,\alpha\}$.  Define $\varphi\colon M\to
k\Gamma e_1$ by letting 
\[\varphi(1,0) = e_1 \text{\ and\ } \varphi(0,1) = \alpha.\]

Have: 
\begin{align}
\alpha \varphi(a,b)  & = \alpha(ae_1 + b\alpha) =
a\underbrace{\alpha e_1}_{=\alpha} +
                       b\underbrace{\alpha^2}_{=0}\notag\\
& = a\alpha\notag\\
& = \varphi(0,a) = \varphi(\alpha(a,b))\notag
\end{align} 
Similarly, $e_i\varphi(a,b) = \varphi(e_i(a,b))$. This implies that
$\varphi$ is a $k\Gamma$-homomorphism.\medskip 

$\left.\begin{matrix}
\Ker \varphi = (0)\\
\Im\varphi = k\Gamma e_1
\end{matrix}\right\} \Rightarrow M\simeq k\Gamma e_1 \text{\ as a left $k\Gamma$-module}.$



\end{exam}

	
\section{quiver with relations}
Can all algebras over a field k be represented as $k\G$?\\\newline
\underline{No}: 
$\Lambda = \faktor{k[x]}{\langle x^2\rangle} \ncong k\G \text{ for all quivers }\G
\text{dim}_{k}\Lambda = 2, \Lambda\text{ not semisimple}.\\\newline \text{Assume that}  \Lambda \cong k\G.\\
2= \text{dim}_kk\G \ge \text{\# vertices in } \G $





\printindex
\begin{thebibliography}{ARS}
\bibitem{ARS} Auslander, M., Reiten, I., Smal\o, S.\ O.,
  \emph{Representation theory of Artin algebras}. Corrected reprint of
  the 1995 original. Cambridge Studies in Advanced Mathematics,
  36. Cambridge University Press, Cambridge, 1997. xiv+425 pp. ISBN:
  0-521-41134-3; 0-521-59923-7.
\end{thebibliography}
\end{document}