\documentclass{amsart}
\usepackage{amsmath, amsthm, amscd, amssymb, latexsym, mathtools, centernot}
\usepackage[all]{xy}
\usepackage{imakeidx,enumerate}

\numberwithin{equation}{section}
\newtheorem{thm}{Theorem}[section]
\newtheorem{cor}[thm]{Corollary}
\newtheorem{lem}[thm]{Lemma}
\newtheorem{prop}[thm]{Proposition}
\theoremstyle{definition}
\newtheorem{defin}[thm]{Definition}
\newtheorem{rem}[thm]{Remark}
\newtheorem{exam}[thm]{Example}

\newcommand{\G}{\Gamma}
\renewcommand{\L}{\Lambda}
<<<<<<< HEAD
\newcommand{\extto}{\xrightarrow}
\newcommand{\Ker}{\operatorname{Ker}\nolimits}
\renewcommand{\Im}{\operatorname{Im}\nolimits}
=======
\newcommand {\defeq}{\stackrel{\mathclap{\normalfont\mbox{def}}}{=}}

>>>>>>> c8c2f1a5158b1a4bc0996589ae8b6c162e9307a8
\usepackage{hyperref}
\makeindex

\begin{document}
\title{Lecture notes for MA3203 Ring theory}

\author{\O yvind Solberg}
\address{Department of Mathematical Sciences\\
NTNU\\
N-7491 Trondheim, Norway}
\email{oyvind.solberg@math.ntnu.no}

\maketitle
\tableofcontents

\section{Quivers}
\subsection{Quivers, vertices, arrows and paths}
\begin{defin}
A \emph{quiver}\index{quiver} $\G = (\G_0,\G_1)$ is an oriented graph,
\begin{align}
\G_0 & = \{\text{vertices}\index{quiver!vertices}\}  (= \{1,2,\ldots,n\}).\notag\\
\G_1 & = \{\text{arrows}\index{quiver!arrows}\}.\notag
\end{align}
\end{defin}
We always assume that $\G_0$ and $\G_1$ are finite sets.

\begin{exam}
$\G\colon \xymatrix{1\ar[r]^\alpha & 2}$, $\G_0=\{1,2\}$ and $\G_1=\{\alpha\}$. 
\end{exam}

\begin{exam}
$\G\colon \xymatrix{1\ar@(ur,dr)[]^\alpha}$, $\G_0=\{1\}$ and $\G_1=\{\alpha\}$. 
\end{exam}

\begin{exam} $\G\colon
  \xymatrix{1\ar@/^.3pc/[rr]^\alpha\ar@/_.3pc/[rr]_\beta & & 2\ar@(ur,dr)[]^\gamma\ar@/_0.3pc/[dl]_\delta\\
    & 3\ar[ul]^\theta\ar@/_0.3pc/[ur]_\epsilon & }$, $\G_0=\{1,2,3\}$ and $\G_1=\{\alpha,\beta,\gamma,\delta,\epsilon, \theta\}$.  
\end{exam}
Have maps: $s, e\colon \G_1\to \G_0$
\begin{align}
\mathfrak{s}(\alpha) & = \text{the vertex where $\alpha\in\G_1$ starts,}\notag\index{$\mathfrak{s}$}\\
\mathfrak{e}(\alpha) & = \text{the vertex where $\alpha\in\G_1$ ends.}\notag\index{$\mathfrak{e}$}
\end{align}

\begin{defin}
$\G=(\G_0,\G_1)$ quiver.  A \emph{path}\index{path} in $\G$ is either
\begin{enumerate}[\rm(i)]
\item an ordered sequence of arrows $p=\alpha_n\alpha_{n-1}\cdots\alpha_1$, where 
\[\mathfrak{e}(\alpha_t) = \mathfrak{s}(\alpha_{t+1})\]
for $t = 1,2,\ldots,n-1$ (\emph{non-trivial path})\index{path!non-trivial} or
\item $e_i$ for each $i$ in $\G_0$ (\emph{trivial path})\index{path!trivial}.
\end{enumerate}
In addition,
\begin{align}
\mathfrak{s}(p) & = \mathfrak{s}(\alpha_1)  & \mathfrak{s}(e_i) = i\notag\\
\mathfrak{e}(p) & = \mathfrak{e}(\alpha_n)  & \mathfrak{e}(e_i) = i\notag
\end{align}
\end{defin}

\begin{exam}\label{exam:1.6}
$\G\colon \xymatrix{ 1\ar[r]^\alpha & 2 \ar[r]^\beta\ar[d]^\gamma & 3\\
& 4 & }$

Paths: 
\begin{enumerate}[\rm(i)]
\item $\alpha, \beta, \gamma, \beta\alpha, \gamma\alpha$.
\item $e_1$, $e_2$, $e_3$, $e_4$.
\end{enumerate}
\end{exam}

\begin{exam}
$\G\colon \xymatrix{1\ar@(ur,dr)[]^\alpha}$.

Paths: 
\begin{enumerate}[\rm(i)]
\item $\alpha, \alpha^2 = \alpha\alpha, \alpha^3 = \alpha\alpha\alpha, \ldots$.
\item $e_1$.
\end{enumerate}
\end{exam}

\subsection{Path algebras}

Given $\G=(\G_0,\G_1)$, a quiver, and $k$ a field.

The \emph{path algebra}\index{path algebra} $k\G$:  $k\G$ is the
vector space with all the paths in $\G$ as a basis. 

The elements in $k\G$:
\[a_1p_1 + a_2p_2 + \cdots + a_tp_t\]
where $a_i\in k$ and $p_i$ are paths in $\G$.

\begin{exam}
Continueing \hyperref[exam:1.6]{Example \ref*{exam:1.6}}:
\begin{align}
x & = a_1e_1 + a_2e_2 + a_3e_3 + a_4e_4 + a_5\alpha + a_6\beta +
a_7\gamma + a_8\beta\alpha + a_9\gamma\alpha\notag\\
y & = b_1e_1 + b_2e_2 + b_3e_3 + b_4e_4 + b_5\alpha + b_6\beta +
b_7\gamma + b_8\beta\alpha + b_9\gamma\alpha\notag
\end{align}
\begin{multline}
x + y = (a_1 + b_1)e_1 + (a_2 + b_2)e_2 + (a_3 + b_3)e_3 + (a_4 +
b_4)e_4 + (a_5 + b_5)\alpha\notag\\ 
 + (a_6 + b_6)\beta + (a_7 + b_7)\gamma
+ (a_8 + b_8)\beta\alpha + (a_9 + b_9(\gamma\alpha\notag
\end{multline}
\end{exam}

$p$, $q$ paths in $\G$:
\begin{enumerate}[\rm(1)]
\item $p$, $q$ both non-trivial 
\[p\cdot q = \begin{cases} pq, & \text{\ if $\mathfrak{e}(q) =\mathfrak{s}(p)$}\notag\\
0, & \text{\ otherwise}\notag
\end{cases}\]
\item $p$ non-trivial, $q$ trivial, $q = e_i$
\[p\cdot q = \begin{cases} p, & \text{\ if $\mathfrak{s}(p) = i =\mathfrak{e}(q)$}\notag\\
0, & \text{\ otherwise}\notag
\end{cases}\]
\[q\cdot p = \begin{cases} p, & \text{\ if $\mathfrak{e}(p) = i = \mathfrak{s}(q)$}\notag\\
0, & \text{\ otherwise}\notag
\end{cases}\]
\item $p = e_i$, $q = e_j$ (both trivial)
\[p\cdot q = \begin{cases} e_i, & \text{\ if $\mathfrak{e}(q) = j = i
    = \mathfrak{s}(p)$}\notag\\
0, & \text{\ otherwise}\notag
\end{cases}\]
\end{enumerate}
 This is extended distributively to an operator on $k\G$ (see
 \cite[page 50]{ARS}).

\begin{exam}
$\Gamma\colon \xymatrix{1\ar[r]^\alpha & 2}$, $k$ field.  

Elements in $k\Gamma$:  $a_1e_1 + a_2e_2 + a_3\alpha = y$. 

\[\begin{array}{c||c|c|c}
      & e_1 & e_2 & \alpha \\ \hline\hline 
e_1 & e_1 &  0   &    0   \\ \hline
e_2 &   0   & e_2 &  \alpha \\ \hline
\alpha & \alpha & 0 & 0
\end{array}\]
\end{exam}
\begin{align}
(e_1 + e_2)\cdot y & = (e_1 + e_2)(a_1e_1 + a_2e_2 +
                     a_3\alpha)\notag\\
& = a_1e_1^2 + a_2\underbrace{e_1e_2}_{=0} + a_3\underbrace{e_1\alpha}_{=0} + a_1\underbrace{e_2e_1}_{=0} + a_2e_2^2 + a_3
  e_2\alpha\notag\\
& = a_1e_1 + a_2e_2 + a_3\alpha = y\notag
\end{align} 
Similary $y\cdot (e_1 + e_2) = y$.  Hence,  $e_1 + e_2$ acts like $1$
in $k\Gamma$. 

Basis for $k\Gamma$: $\{ e_1, e_2, \alpha\}$, $\dim_kk\Gamma = 3$.

\begin{exam}
$\Gamma\colon \xymatrix{1\ar@(ur,dr)[]^\alpha}$, and $k$ a field.

$k\Gamma$ has basis: $\{e_1, \alpha, \alpha^2, \alpha^3, \ldots\}$,
that is, $\dim_k k\Gamma = \infty$. 

Elemenets in $k\Gamma$: $a_0e_1 + a_1\alpha + a\alpha^2 + \cdots +
a_t\alpha^t$, with $a_i$ in $k$ and $t\geqslant 0$.  
\end{exam}


<<<<<<< HEAD
\subsection{Modules and representations}

$\Gamma = (\Gamma_0,\Gamma_1)$ - quiver, $k$ field.

$M$ left $k\Gamma$-module $\leadsto\begin{cases}
(V,f) \text{\ representation of $\Gamma$}\\
V(i) = e_iM\\
\text{for\ } \alpha\colon i\to j \in \Gamma_1, \text{\ we have\ }  f_\alpha\colon
V(i)=e_iM\extto{\alpha\cdot -} e_jM = V(j)\\
f_\alpha(e_im) = \alpha e_im
\end{cases}$

$(V,f)$ representaion of $\Gamma\leadsto \begin{cases}
M = \oplus_{i\in\Gamma_0} V(i)\\
m = (v_1,v_2,\ldots, v_n) \in M\\
e_im \extto{\text{def}}{=} (0,\ldots,0,v_i,0,\ldots,0)\\
\text{for $\alpha\colon i\to j$ in $\Gamma_1$, remember $\alpha =
  e_j\alpha e_i$}\\
\alpha m \extto{\text{def}}{=} (0,\ldots,0,f_\alpha(v_i),0,\ldots,0)
\text{\ with $f_\alpha(v_i)$ in the $j$-th coordinate}
\end{cases}$

\begin{exam}
$\Gamma\colon \xymatrix{1\ar[r]^\alpha & 2}$, $k$ field.

$(V,f)\colon \xymatrix{k\ar[r]^1 & k} \leadsto M = k\oplus k = k^2$

\begin{align}
e_1\cdot (a,b) & = (a, 0)\notag\\
e_2\cdot (a,b) & = (0,b)\notag\\
\alpha\cdot(a,b) & = (0,a)\notag
\end{align}
Note: $k\Gamma e_1 = k\{ e_1,\alpha\}$.  Define $\varphi\colon M\to
k\Gamma e_1$ by letting 
\[\varphi(1,0) = e_1 \text{\ and\ } \varphi(0,1) = \alpha.\]

Have: 
\begin{align}
\alpha \varphi(a,b)  & = \alpha(ae_1 + b\alpha) =
a\underbrace{\alpha e_1}_{=\alpha} +
                       b\underbrace{\alpha^2}_{=0}\notag\\
& = a\alpha\notag\\
& = \varphi(0,a) = \varphi(\alpha(a,b))\notag
\end{align} 
Similarly, $e_i\varphi(a,b) = \varphi(e_i(a,b))$. This implies that
$\varphi$ is a $k\Gamma$-homomorphism.\medskip 

$\left.\begin{matrix}
\Ker \varphi = (0)\\
\Im\varphi = k\Gamma e_1
\end{matrix}\right\} \Rightarrow M\simeq k\Gamma e_1 \text{\ as a left $k\Gamma$-module}.$



\end{exam}

=======
Notes\index{note}
\begin{enumerate}
	\item  In general, $\{e_i\}_{i\epsilon\G}$ are orthogonal idempotents in $k\Gamma$, ie. 
	\[ \begin{cases}\text{\ $e_i^2 = e_i$}\notag\\
	\text{$e_ie_j = 0$ for i$\neq$j}\notag
	\end{cases}\]
	
	\item Suppose $\G_0 = \{1,2,...,n\}$. Then $e_1 + e_2 + ... + e_n$ acts like 1 in $k\G$. Enough to show that $p = (e_1 + e_2 + ... + e_n)p = p(e_1 + e_2 + ... + e_n)$ for any path p. Suppose that $\mathfrak{s}(p) = i$ and $\mathfrak{e}(p) = j$ Then
	\newline $(e_1 + e_2 + ... + e_n)p = e_1p + e_2p + ... + e_jp + ... + e_np = e_jp \defeq p$ \newline
	\newline $p(e_1 + e_2 + ... + e_n) = pe_1 + pe_2 + ... + pe_i + ... + pe_n = e_jp \defeq p$ \newline
	
	\item $\implies e_1 + e_2 + ... + e_n = 1_{k\G}$ = identity in $k\G$ 
\end{enumerate}

Can show: $k\G$ is a k-algebra with $e_1 + e_2 + ... + e_n$ as an identity (see \cite[page 50]{ARS})\newline

recall: $\Lambda$ ring, $k$ field
\begin{defin}
	$\Lambda$ is a $k$-algebra, if $\Lambda$ is a vector space over $k$ ($\xymatrix{k\times\Lambda \ar[r] & \Lambda}$, $\Lambda$ is a module over $k$, $\alpha \in k, \lambda \in \Lambda, \alpha\cdot\lambda$) and $\alpha(\lambda\cdot\lambda^{'})=(\alpha\cdot\lambda)\cdot\lambda^{'} = \lambda(\alpha\cdot\lambda^{'})$ \newline $\forall \alpha \in k, \forall \lambda, \lambda^{'}\in\Lambda$\newline
	
	\underline{Equivalent}: $\Lambda$ is a $k$-algebra, if $\exists \phi\colon k \to \Lambda$ a ring homomorphism such that \newline
	Im $\phi \subseteq Z(\Lambda)= \{z\in \Lambda \arrowvert z\lambda=\lambda z,\forall \lambda \in \Lambda\}$ ($\iff \exists R \subseteq \Lambda$ subring such that $R \simeq k$ with $R \subseteq Z(\Lambda)$)\newline
	
	$\phi (a)=a \cdot 1_{\Lambda}$ For $ k\G $ the ring homomorphisme $\phi \colon k \rightarrow k\G $ is given by $\phi(a) = ae_1 + ae_2 + ... + ae_n$ 
\end{defin}
\newpage
\underline{Exercises}:
\begin{enumerate}
	\item $\Gamma\colon \xymatrix{1\ar[r]^\alpha & 2}$, k field. \newline
	Find a $k$-algebra isomorphisme \newline
	
	$\psi\colon k\G \rightarrow $
	$\begin{pmatrix}
	$k$ & 0\\
	$k$ & $k$
	\end{pmatrix}
	$\newline
	\item $\G\colon \xymatrix{1\ar@(ur,dr)[]^\alpha}$. $k$ field. \newline
	Show that $k\G\simeq k[x]$ as $k$-algebra's. 
\end{enumerate}

\begin{defin}
	A non-trival path $p$ in $\G$ is an oriented cycle if \[\mathfrak{e}(p) = \mathfrak{s}(p)\]
\end{defin}

wrong path!!!!!!!!!!!!!!!!!!!!!!!!!!!!!!!!!!!! 
\begin{exam}
	$\G\colon \xymatrix{1\ar@(ur,dr)[]^\alpha}$\\ \newline
	\underline{Cycles}: $\alpha,\alpha^3,\gamma\beta\alpha,\beta\alpha^10\gamma, ...$  $\text{dim}_k k\G = \infty$ 
\end{exam}

\begin{prop}\label{prop1}
	$\G = (\G_0,\G_1)$ quiver, $k$ field. $\text{dim}_kk\G < \infty \iff \G$ has no oriented cycles.
\end{prop}

\begin{proof}
	Exercise
\end{proof}

\begin{prop}
	Assume that $\G=(\G_0,\G_1)$ has no oriented cycles. $k\G$ is semisimpel $\iff \G_1 = \emptyset$
\end{prop}

\begin{proof}
	proposition \ref{prop1} $ \implies \text{dim}_kk\G < \infty \implies k\G$ is a left artinian ring. \\
	$k\G$ semisimpel $\iff $ no non-zero nilpotent left ideals in $k\G$\\
	$\implies\colon$ Assume that $\G_1\neq\emptyset.$ Let $\alpha_1$ be an arrow in $\G$. Want to find a vertex where at least one arrow ends and no arrow starts. if \[\mathfrak{e}(\alpha_1)\] is such a vertex, we are done. If not, there is an arrow $\alpha_2$ starting in \[\mathfrak{e}(\alpha_1) \]. If also\[\mathfrak{e}(\alpha_2)\] is not as above, we continue. Since $\G$ has no oriented cycles and $\G$ is finite, we must end up in a vertex $ v$, where arrows only end and no arrows starts. Say, $\alpha = \alpha_t$ is an arrow ending in $v$. Then consider $k\G\alpha = k\alpha$ Since $(a_1\alpha)(a_2\alpha = (a_1a_2)\underbrace{(\alpha\alpha)}_{=0} = 0$ $\implies (k\G\alpha)^2=(0)$ and $k\G\alpha \neq (0)$\\
	$\implies k\G$ is not semisimpel.\\\newline
	$\impliedby\colon$ assume that $\G_1 = \emptyset$. then $\G$ 1 2 $\cdots$ n (n vertice)\\
	Basis for $k\G\colon \{e_1, e_2, \cdots, e_n\}$. Elements in $k\G\colon a_1e_1 + a_2e_2 + \cdots + a_ne_n$ with $a_i \in k$. Have a ring homomorphisme. $ \psi\colon \underbrace{k \times \cdots \times k}_{n}  \rightarrow k\G$\\
	given by $\psi(a_1,a_2,\cdots,a_n)=a_1e_1 + a_2e_2+ \cdots a_ne_n$(check this!). Show that $\psi$ is an isomorphisme. Therefore $k\G$ is semisimpel, since $k\G$ is isomorphic to a finite product of full matrix rings over divisjon rings.
\end{proof}

$k\G$ is not always semisimpel, but some factor of $k\G$ is.\newline

\begin{prop}
	$\G=(\G_0,\G_1) $ quiver, k field. Let $J = \{\text{all linear combinations of non-trivial paths}\}$
	Then $J$ is an ideal in $k\G$ and $k\G/J \simeq \underbrace{k \times \cdots \times k}_{|\G_0|}$, -semisimpel
\end{prop}

\begin{proof}"proof"\\
	Define $\psi\colon k\G \rightarrow \underbrace{k \times \cdots \times}_{|\G_0|=n} = k^n $\\
	$\psi(a_1e_1 + a_2e_2 + \cdots + a_ne_n + \text{ linear combinations of non-trivial paths} ) = (a_1, a_2, \cdots, a_n)$\\\newline
	\underline{Check:} \begin{enumerate}
		\item  $\psi$ is well-defined
		\item $\psi$ homomorphism of rings
		\item ker$\psi=J$
	\end{enumerate}
	$\implies k\G/J \simeq$ Im$\psi=k^n$
\end{proof}

\subsection{Modules}
\begin{exam}
	$\Gamma\colon \xymatrix{1\ar[r]^\alpha & 2}$, k field
	What is a module over $k\G$? Let M be a left $k\G$-module.
	\underline{Recall:} 
	$1_{k\G}=e_1 + e_2$, \[e_ie_j=  
	\begin{cases}	\text{\ $e_i^2 = e_i$}\\
	\text{$e_ie_j = 0$ for i$\neq$j}
	\end{cases}\]
	\underline{Claim:} $M = e_1M\oplus e_2M$ as vector space over k.\\\newline
	\underline{Proof:}
	
	\begin{align*}
	m &= 1_{k\G}*m=(e_1 + e_2)m = e_1m+e_2m \in e_1M + e_2M\\
	\implies M &\subseteq e_1M + e_2M \subseteq M \implies M = e_1M + e_2M\\\\
	\text{Let } m&\in e_1M \cap e_2M \text{, i.e } m=e_1m'=e_2m''\\\\
	e_1m&=e_1(e_1m')=(e_1e_1)m'=e_1m'=m\\
	&= e_1(e_2m'')=\underbrace{(e_1e_2)m''}_{=0}=0\cdot m''=0\\
	\implies m &=0. \text{ hence } e_1M \cap e_2M = (0)\\
	\implies M &= e_1M \oplus e_2M\\ 
	\end{align*}
	
	
	
\end{exam}


>>>>>>> c8c2f1a5158b1a4bc0996589ae8b6c162e9307a8
\printindex
\begin{thebibliography}{ARS}
\bibitem{ARS} Auslander, M., Reiten, I., Smal\o, S.\ O.,
  \emph{Representation theory of Artin algebras}. Corrected reprint of
  the 1995 original. Cambridge Studies in Advanced Mathematics,
  36. Cambridge University Press, Cambridge, 1997. xiv+425 pp. ISBN:
  0-521-41134-3; 0-521-59923-7.
\end{thebibliography}
\end{document}