\documentclass{amsart}
\usepackage{amsmath, amsthm, amscd, amssymb, latexsym, mathtools, centernot, faktor}
\usepackage[all]{xy}
\usepackage{imakeidx,enumerate}

\numberwithin{equation}{section}
\newtheorem{thm}{Theorem}[section]
\newtheorem{cor}[thm]{Corollary}
\newtheorem{lem}[thm]{Lemma}
\newtheorem{prop}[thm]{Proposition}
\theoremstyle{definition}
\newtheorem{defin}[thm]{Definition}
\newtheorem{rem}[thm]{Remark}
\newtheorem{exam}[thm]{Example}
\newtheorem{note}[thm]{Note}

\newcommand{\G}{\Gamma}
\renewcommand{\L}{\Lambda}

\newcommand{\extto}{\xrightarrow}
\newcommand{\Ker}{\operatorname{Ker}\nolimits}
\renewcommand{\Im}{\operatorname{Im}\nolimits}
\newcommand {\defeq}{\stackrel{\mathclap{\normalfont\mbox{def}}}{=}}

\usepackage{hyperref}
\makeindex

\begin{document}
\title{Lecture notes for MA3203 Ring theory}

\author{\O yvind Solberg}
\address{Department of Mathematical Sciences\\
NTNU\\
N-7491 Trondheim, Norway}
\email{oyvind.solberg@math.ntnu.no}

\maketitle
\tableofcontents

\section{Quivers}
\subsection{Quivers, vertices, arrows and paths}
\begin{defin}
A \emph{quiver}\index{quiver} $\G = (\G_0,\G_1)$ is an oriented graph,
\begin{align}
\G_0 & = \{\textit{vertices}\index{quiver!vertices}\}  (= \{1,2,\ldots,n\}).\notag\\
\G_1 & = \{\textit{arrows}\index{quiver!arrows}\}.\notag
\end{align}
\end{defin}
We always assume that $\G_0$ and $\G_1$ are finite sets.

\begin{exam}
$\G\colon \xymatrix{1\ar[r]^\alpha & 2}$, $\G_0=\{1,2\}$ and $\G_1=\{\alpha\}$. 
\end{exam}

\begin{exam}
$\G\colon \xymatrix{1\ar@(ur,dr)[]^\alpha}$, $\G_0=\{1\}$ and $\G_1=\{\alpha\}$. 
\end{exam}

\begin{exam} $\G\colon
  \xymatrix{1\ar@/^.3pc/[rr]^\alpha\ar@/_.3pc/[rr]_\beta & & 2\ar@(ur,dr)[]^\gamma\ar@/_0.3pc/[dl]_\delta\\
    & 3\ar[ul]^\theta\ar@/_0.3pc/[ur]_\epsilon & }$, $\G_0=\{1,2,3\}$ and $\G_1=\{\alpha,\beta,\gamma,\delta,\epsilon, \theta\}$.  
\end{exam}
Have maps: $\mathfrak{s}, \mathfrak{e}\colon \G_1\to \G_0$
\begin{align}
\mathfrak{s}(\alpha) & = \text{the vertex where $\alpha\in\G_1$ starts,}\notag\index{$\mathfrak{s}$}\\
\mathfrak{e}(\alpha) & = \text{the vertex where $\alpha\in\G_1$ ends.}\notag\index{$\mathfrak{e}$}
\end{align}

\begin{defin}
$\G=(\G_0,\G_1)$ quiver.  A \emph{path}\index{path} in $\G$ is either
\begin{enumerate}[\rm(i)]
\item an ordered sequence of arrows $p=\alpha_n\alpha_{n-1}\cdots\alpha_1$, where 
\[\mathfrak{e}(\alpha_t) = \mathfrak{s}(\alpha_{t+1})\]
for $t = 1,2,\ldots,n-1$ (\emph{non-trivial path})\index{path!non-trivial} or
\item $e_i$ for each $i$ in $\G_0$ (\emph{trivial path})\index{path!trivial}.
\end{enumerate}
In addition,
\begin{align}
\mathfrak{s}(p) & = \mathfrak{s}(\alpha_1)  & \mathfrak{s}(e_i) = i\notag\\
\mathfrak{e}(p) & = \mathfrak{e}(\alpha_n)  & \mathfrak{e}(e_i) = i\notag
\end{align}
\end{defin}

\begin{exam}\label{exam:1.6}
$\G\colon \xymatrix{ 1\ar[r]^\alpha & 2 \ar[r]^\beta\ar[d]^\gamma & 3\\
& 4 & }$

Paths: 
\begin{enumerate}[\rm(i)]
\item $\alpha, \beta, \gamma, \beta\alpha, \gamma\alpha$.
\item $e_1$, $e_2$, $e_3$, $e_4$.
\end{enumerate}
\end{exam}

\begin{exam}
$\G\colon \xymatrix{1\ar@(ur,dr)[]^\alpha}$.

Paths: 
\begin{enumerate}[\rm(i)]
\item $\alpha, \alpha^2 = \alpha\alpha, \alpha^3 = \alpha\alpha\alpha, \ldots$.
\item $e_1$.
\end{enumerate}
\end{exam}

\subsection{Path algebras}

Given $\G=(\G_0,\G_1)$, a quiver, and $k$ a field.

The \emph{path algebra}\index{path algebra} $k\G$:  $k\G$ is the
vector space with all the paths in $\G$ as a basis. 

The elements in $k\G$:
\[a_1p_1 + a_2p_2 + \cdots + a_tp_t\]
where $a_i\in k$ and $p_i$ are paths in $\G$.  We write just $p$ for
$1p$, when $p$ is a path in $\G$. 

\begin{exam}
Continuing \hyperref[exam:1.6]{Example \ref*{exam:1.6}}:
\begin{align}
x & = a_1e_1 + a_2e_2 + a_3e_3 + a_4e_4 + a_5\alpha + a_6\beta +
a_7\gamma + a_8\beta\alpha + a_9\gamma\alpha\notag\\
y & = b_1e_1 + b_2e_2 + b_3e_3 + b_4e_4 + b_5\alpha + b_6\beta +
b_7\gamma + b_8\beta\alpha + b_9\gamma\alpha\notag
\end{align}
\begin{multline}
x + y = (a_1 + b_1)e_1 + (a_2 + b_2)e_2 + (a_3 + b_3)e_3 + (a_4 +
b_4)e_4 + (a_5 + b_5)\alpha\notag\\ 
 + (a_6 + b_6)\beta + (a_7 + b_7)\gamma
+ (a_8 + b_8)\beta\alpha + (a_9 + b_9)\gamma\alpha\notag
\end{multline}
\end{exam}

\subsubsection*{Multiplication}\index{path algebra!multiplication} $p$, $q$ paths in $\G$:
\begin{enumerate}[\rm(1)]
\item $p$, $q$ both non-trivial 
\[p\cdot q = \begin{cases} pq, & \text{\ if $\mathfrak{e}(q) =\mathfrak{s}(p)$}\notag\\
0, & \text{\ otherwise}\notag
\end{cases}\]
\item $p$ non-trivial, $q$ trivial, $q = e_i$
\[p\cdot q = \begin{cases} p, & \text{\ if $\mathfrak{s}(p) = i =\mathfrak{e}(q)$}\notag\\
0, & \text{\ otherwise}\notag
\end{cases}\]
\[q\cdot p = \begin{cases} p, & \text{\ if $\mathfrak{e}(p) = i = \mathfrak{s}(q)$}\notag\\
0, & \text{\ otherwise}\notag
\end{cases}\]
\item $p = e_i$, $q = e_j$ (both trivial)
\[p\cdot q = \begin{cases} e_i, & \text{\ if $\mathfrak{e}(q) = j = i
    = \mathfrak{s}(p)$}\notag\\
0, & \text{\ otherwise}\notag
\end{cases}\]
\end{enumerate}
 This is extended distributively to an operation on $k\G$ (see
 \cite[page 50]{ARS}).

\begin{exam}
$\Gamma\colon \xymatrix{1\ar[r]^\alpha & 2}$, $k$ field.  

Elements in $k\Gamma$:  $a_1e_1 + a_2e_2 + a_3\alpha = y$. 

\[\begin{array}{c||c|c|c}
      & e_1 & e_2 & \alpha \\ \hline\hline 
e_1 & e_1 &  0   &    0   \\ \hline
e_2 &   0   & e_2 &  \alpha \\ \hline
\alpha & \alpha & 0 & 0
\end{array}\]
\end{exam}
\begin{align}
(e_1 + e_2)\cdot y & = (e_1 + e_2)(a_1e_1 + a_2e_2 +
                     a_3\alpha)\notag\\
& = a_1e_1^2 + a_2\underbrace{e_1e_2}_{=0} + a_3\underbrace{e_1\alpha}_{=0} + a_1\underbrace{e_2e_1}_{=0} + a_2e_2^2 + a_3
  e_2\alpha\notag\\
& = a_1e_1 + a_2e_2 + a_3\alpha = y\notag
\end{align} 
Similarly $y\cdot (e_1 + e_2) = y$.  Hence,  $e_1 + e_2$ acts like $1$
in $k\Gamma$. 

Basis for $k\Gamma$: $\{ e_1, e_2, \alpha\}$, $\dim_kk\Gamma = 3$.

\begin{exam}
$\Gamma\colon \xymatrix{1\ar@(ur,dr)[]^\alpha}$, and $k$ a field.

$k\Gamma$ has basis: $\{e_1, \alpha, \alpha^2, \alpha^3, \ldots\}$,
that is, $\dim_k k\Gamma = \infty$. 

Elements in $k\Gamma$: $a_0e_1 + a_1\alpha + a\alpha^2 + \cdots +
a_t\alpha^t$, with $a_i$ in $k$ and $t\geqslant 0$.  
\end{exam}

\begin{note}
\begin{enumerate}
\item In general, $\{e_i\}_{i \in \G}$ are \emph{orthogonal
    idempotents}\index{idempotents!orthogonal} in $k\Gamma$, that is, 
	\[ \begin{cases}\text{\ $e_i^2 = e_i$}\notag\\
	\text{$e_ie_j = 0$ for $i\neq j$}\notag
	\end{cases}\]
	
    \item Suppose $\G_0 = \{1,2,\ldots,n\}$. Then $e_1 + e_2 + \cdots + e_n$
      acts like 1 in $k\G$. Enough to show that
      \[p = (e_1 + e_2 + \cdots + e_n)p = p(e_1 + e_2 + \cdots +
        e_n)\] 
      for any path $p$. Suppose that $\mathfrak{s}(p) = i$ and
      $\mathfrak{e}(p) = j$. Then 
      \[(e_1 + e_2 + \cdots + e_n)p = e_1p + e_2p + \cdots + e_jp + \cdots +
      e_np = e_jp \defeq p\]
      \[p(e_1 + e_2 + \cdots + e_n) = pe_1 + pe_2 + \cdots + pe_i + \cdots +
      pe_n = e_jp \defeq p\]
	
    $\implies e_1 + e_2 + \cdots + e_n = 1_{k\G}$ = identity in $k\G$
\end{enumerate}
Can show: $k\G$ is a $k$-algebra with $e_1 + e_2 + \cdots + e_n$ as an
identity (see \cite[page 50]{ARS}). 
\end{note}

Recall: $\Lambda$ ring, $k$ field. 
\begin{defin}
  $\Lambda$ is a $k$-\emph{algebra}\index{algebra}, if $\Lambda$ is a
  vector space over $k$ ($\xymatrix{k\times\Lambda \ar[r] & \Lambda}$,
  $\Lambda$ is a module over $k$,
  $\alpha \in k, \lambda \in \Lambda, \alpha\cdot\lambda$) and
\[\alpha(\lambda\cdot\lambda^{'})=(\alpha\cdot\lambda)\cdot\lambda^{'}
  = \lambda(\alpha\cdot\lambda^{'})\]
  $\forall \alpha \in k, \forall \lambda,
  \lambda'\in\Lambda$. 
	
\begin{note}
  $\Lambda$ is a $k$-algebra, if $\exists\ \phi\colon k \to \Lambda$ a
  ring homomorphism such that
 \[\Image\phi \subseteq Z(\Lambda)= \{z\in \Lambda \mid 
  z\lambda=\lambda z,\forall \lambda \in \Lambda\}\]
  ($\iff \exists R \subseteq \Lambda$ subring such that $R \simeq k$
  with $R \subseteq Z(\Lambda)$, just define $\phi (a)=a \cdot
  1_{\Lambda}$). 
\end{note}
  For $ k\G $ the ring homomorphism $\phi \colon k \rightarrow k\G $
  is given by $\phi(a) = ae_1 + ae_2 + \cdots + ae_n$
\end{defin}
\begin{exer}
\begin{enumerate}
\item $\Gamma\colon \xymatrix{1\ar[r]^\alpha & 2}$, $k$ field. 

Find a $k$-algebra isomorphism
\[\psi\colon k\G \rightarrow \begin{pmatrix}
    k & 0\\
    k & k
  \end{pmatrix}.\]
\item $\G\colon \xymatrix{1\ar@(ur,dr)[]^\alpha}$,  $k$ field.

Show that $k\G\simeq k[x]$ as $k$-algebras.
\end{enumerate}
\end{exer}

\begin{defin}
  A non-trivial path $p$ in $\G$ is an \emph{oriented
    cycle}\index{oriented cycle} if
  \[\mathfrak{e}(p) = \mathfrak{s}(p).\]
\end{defin}

\begin{exam}
  $\G\colon \xymatrix{\ar@(ul,dl)[]_\alpha 1 \ar@/^.3pc/[r]^\beta &
    \ar@/^.3pc/[l]^\gamma 2}$\\ \newline 

\subsubsection*{Cycles}
  $\alpha,\alpha^3,\gamma\beta\alpha,\beta\alpha^{10}\gamma, \ldots$
  $\dim_k k\G = \infty$
\end{exam}

\begin{prop}\label{prop1}
  $\G = (\G_0,\G_1)$ quiver, $k$ field.
\begin{center}
  $\dim_kk\G < \infty$ if and only $\G$ has no oriented cycles.
\end{center}
\end{prop}
\begin{proof}
  Exercise.
\end{proof}

\begin{prop}
  Assume that $\G=(\G_0,\G_1)$ has no oriented cycles. 
\begin{center}
$k\G$ is semisimple $\iff \G_1 = \emptyset$.
\end{center}
\end{prop}

\begin{proof}
  Proposition \ref{prop1} $ \implies \text{dim}_kk\G < \infty \implies k\G$ is a left artinian ring. \\
  $k\G$ semisimple $\iff $ no non-zero nilpotent left ideals in $k\G$.\\

  $\Mapsto$: Assume that $\G_1\neq\emptyset.$ Let $\alpha_1$ be an
  arrow in $\G$. Want to find a vertex where at least one arrow ends
  and no arrow starts. If $\mathfrak{e}(\alpha_1)$ is such a vertex,
  we are done. If not, there is an arrow $\alpha_2$ starting in
  $\mathfrak{e}(\alpha_1)$.  If also $\mathfrak{e}(\alpha_2)$ is not
  as above, we continue. Since $\G$ has no oriented cycles and $\G$ is
  finite, we must end up in a vertex $ v$, where arrows only end and
  no arrows starts. Say, $\alpha = \alpha_t$ is an arrow ending in
  $v$. Then consider $k\G\alpha = k\alpha$.  Since
  $(a_1\alpha)(a_2\alpha = (a_1a_2)\underbrace{(\alpha\alpha)}_{=0} =
  0$ $\implies (k\G\alpha)^2=(0)$ and $k\G\alpha \neq (0)$, we infer
  that $k\G$ is not semisimple.

$\Mapsfrom$: Assume that $\G_1 = \emptyset$. Then 
\[\G\colon \xymatrix{1 & 2\ar@{..}[rr] & &  n} \text{ ($n$ vertices)}\]
  Basis for $k\G\colon \{e_1, e_2, \ldots, e_n\}$. Elements in
  $k\G\colon a_1e_1 + a_2e_2 + \cdots + a_ne_n$ with $a_i \in k$. Have
  a ring homomorphism
\[\psi\colon \underbrace{k \times \cdots \times k}_{n}  \rightarrow k\G\]
  given by
\[\psi(a_1,a_2,\ldots,a_n)=a_1e_1 + a_2e_2+ \cdots + a_ne_n\] 
(check this!).  Show that $\psi$ is an isomorphism. Therefore $k\G$ is
  semisimple, since $k\G$ is isomorphic to a finite product of full
  matrix rings over divisjon rings.
\end{proof}

Note: $k\G$ is not always semisimple, but some factor of $k\G$ is.

\begin{prop}
  $\G=(\G_0,\G_1) $ quiver, $k$ field. Let
\[J = \{\text{all linear combinations of non-trivial paths}\}.\] 
Then $J$ is an ideal in $k\G$ and
  $k\G/J \simeq \underbrace{k \times \cdots \times k}_{|\G_0|}$,
  -semisimple
\end{prop}
\begin{proof}[Sketch of proof]
  Define $\psi\colon k\G \rightarrow \underbrace{k \times \cdots \times}_{|\G_0|=n} = k^n $
\[\psi(a_1e_1 + a_2e_2 + \cdots + a_ne_n + \text{ linear combinations
    of non-trivial paths} ) = (a_1, a_2, \cdots, a_n)\]
  \underline{Check:} \begin{enumerate}
		\item  $\psi$ is well-defined
		\item $\psi$ homomorphism of rings
		\item ker$\psi=J$
	\end{enumerate}
	$\implies k\G/J \simeq$ Im$\psi=k^n$.
\end{proof}

\section{Modules}

\begin{exam}
	$\Gamma\colon \xymatrix{1\ar[r]^\alpha & 2}$, k field. 
	What is a module over $k\G$? Let M be a left $k\G$-module.
	\underline{Recall:} 
	$1_{k\G}=e_1 + e_2$, \[e_ie_j=  
	\begin{cases}	\text{\ $e_i^2 = e_i$}\\
	\text{$e_ie_j = 0$ for i$\neq$j}
	\end{cases}\]
	\underline{Claim:} $M = e_1M\oplus e_2M$ as vector space over k.\\\newline
	\begin{proof}
	
	\begin{align*}
	m &= 1_{k\G}*m=(e_1 + e_2)m = e_1m+e_2m \in e_1M + e_2M\\
	\implies M &\subseteq e_1M + e_2M \subseteq M \implies M = e_1M + e_2M\\\\
	\text{Let } m&\in e_1M \cap e_2M \text{, i.e } m=e_1m'=e_2m''\\\\
	e_1m&=e_1(e_1m')=(e_1e_1)m'=e_1m'=m\\
	&= e_1(e_2m'')=\underbrace{(e_1e_2)m''}_{=0}=0\cdot m''=0\\
	\implies m &=0. \text{ hence } e_1M \cap e_2M = (0)\\
	\implies M &= e_1M \oplus e_2M\\ 
	\end{align*}
\end{proof}
\end{exam}

\begin{exam}
\(\Gamma :1\xmapsto{\alpha} 2\), \(k\) field\\
What is a module over \(k\Gamma\)? Let \(M\) be a left \(k\Gamma\)-module.\\
\underline{Recall}: \(1_{k\Gamma} = e_1 + e_2, e_ie_j = \begin{cases}
    e_i &, i=j \\
    0\, & ,i\not=j.
\end{cases}\)\\
\begin{prop}
 \(M = e_1\oplus e_2M\) as a vector space over \(k\).
\end{prop}
\noindent\underline{Proof}: \(m = 1_{k\Gamma}\cdot m = (e_1 + e_2)m = e_1m + e_2m \in e_1M + e_2M \implies M\subseteq e_1M + e_2M \subseteq M \implies M = e_1M + e_2M\). Let \(m\in e_1M\cap e_2M\), i.e. \(m = e_1m' = e_2m''\)\\[0.5cm]
\begin{tabular}{l}
     \(e_1m = e_(e_1m') = (e_1e_1)m' = e_1m' = m\)  \\
     \(\quad\parallel\) \\
     \(e_1(e_2m'') = (e_1e_2)m'' = 0m'' = 0\)
\end{tabular}\\[0.5cm]
\(\implies m= 0.\) Hence \(e_1M\cap e_2M = (0) \implies M = e_1M \oplus e_2M\).\\
Let \(m\in M\) Then \(e_1m = e_1(e_m + e_2m) = e_1^2m + (e_1e_2)m = e_1m)\) and \(e_2m = e_2(e_1m + e_2m) = e_2m\)\\
\(\alpha m = \alpha(e_1m + e_2m) = \alpha(e_1m) + \alpha(e_2m) = \alpha(e_1m) + 0 = \alpha m = \alpha e_1 m = (e_2\alpha)e_1m = e_2(\alpha e_1 m)\in e_2M\).\\[1cm]
\(M \xmapsto{\alpha\cdot-} M \quad\quad\)linear map \(\alpha:e_1M\xmapsto{\alpha\cdot-}e_2M\)\\
\(M\xmapsto{e_1\cdot-}M\quad\quad\)linear map, projection \(M\mapsto e_1M\)\\
\(M\xmapsto{e_2\cdot-}M\quad\quad\)linear map, projection \(M\mapsto e_2M\)\\

\begin{center}
\(e_1M \xmapsto{\alpha\cdot -}e_2M\)
\end{center}
is a representation of \(\Gamma\) over \(k\). A vector-space in each vertex and a linear map as the arrow.
\end{exam}
\noindent Given \(V\xmapsto{f}V',\quad V,V'\) vector spaces over \(k\), \(f\) a linear map. How can we construct a left \(k\Gamma\)-module?\\
From above: \(M = V\oplus V'\) as a vector space. Let \(m = (v, v')\), then\\
\begin{center}
\begin{tabular}{l}
\(e_1m \defeq (v, 0)\)\\
\(e_2m \defeq (0, v)\)\\
\(\alpha m \defeq (0, f(v))\)
\end{tabular}
\end{center}
\underline{Check}: \(M\) becomes a left \(k\Gamma\)-module!
\begin{defin}
A representation \((V, f)\) of a quiver \(\Gamma = (\Gamma_0, \Gamma_1)\) over a field \(k\) is a collection of vector spaces \(\{V(i)\}_{i\in\Gamma_0}\) over \(k\) and \(k\)-linear maps \(f_\alpha:V(i)\mapsto V(j)\) for each arrow \(\alpha: i\mapsto j\) in \(\Gamma_1\). (We assume that \(\text{dim}_kV(i) < \infty\) for all \(i\in\Gamma_0\).
\end{defin}
\begin{exam}
\begin{enumerate}[(1)]
    \item \(\Gamma: 1.\) A representation of \(\Gamma\) over \(k\) is just a vector space over \(k\).
    \item \(\Gamma: 1\xmapsto{\alpha}2\). Representation \(V(1)\xmapsto{f_\alpha}V(2)\). For example \\
    \(k\xmapsto{1}k\quad\quad k\xmapsto{0}0\quad\quad 0\xmapsto{0}k\)\quad\quad
    \(k^2\xmapsto{\tiny\begin{pmatrix} 1 & 2\\ 0 & 3 \\ -1 & 1 \end{pmatrix}}k^3\) \\
    \item \(\Gamma: \xymatrix{ & 1\ar[dl]_\alpha \ar[dr]^\beta & \\
    2\ar[dr]_{\gamma} & & 3 \ar[dl]^\delta \\
    & 4 & }
    \quad\quad \text{Representation: }\xymatrix{ & V(1)\ar[dl]_{f_\alpha} \ar[dr]^{f_\beta} & \\
    V(2)\ar[dr]_{f_\gamma} & & V(3) \ar[dl]^{f_\delta} \\
    & V(4) & }\)\\
    For example:\\
    \(\xymatrix{ & k\ar[dl]_{1} \ar[dr]^{1} & \\
    k\ar[dr]_{1} & & k \ar[dl]^{1}\\
    & k & }\quad\quad\xymatrix{ & k^2\ar[dl]_{\small\begin{pmatrix} 1 & 0 \end{pmatrix}} \ar[dr]^{\small\begin{pmatrix} 1 & -1 \\ 0 & 1 \end{pmatrix}} & \\
    k\ar[dr]_{\small\begin{pmatrix} 1 \\ 1 \end{pmatrix}} & & k^2 \ar[dl]^{\small\begin{pmatrix} 0 & 0\\ 0 & 1 \end{pmatrix}}\\
    & k^2 & }\)
\end{enumerate}
\end{exam}
\subsection{Maps Between representations}
\begin{exam}
\(\Gamma: 1\xmapsto{\alpha}2\), \(k\) a field\\
Let \(f:M\xmapsto{}N\) be a homomorphism of left \(k\Gamma\)-modules. Then
\begin{align*}
    f(e_1m) = f((e_1e_1)m) = f(e_1(e_1m)) = e_1f(e_1m) \in e_1N \implies f|_{e_1M}:e_1M \xmapsto{}e_1N
\end{align*}
Similarly, \(f|{e_2M}:e2M\xmapsto{}e_2N\). Furthermore, \\[0.5cm]
\begin{tabular}{rcl}
     \(\alpha f(e_1m)\) & \(=\) & \(f(\alpha(e_1m)) \quad\quad (\alpha = e_2\alpha)\) \\
     \(\parallel\quad\) & & \(\quad\parallel\) \\
     \(\alpha f|_{e_1M}(e_1m)\) & & \(f|_{e_2M}(\alpha(e_1m))\)
\end{tabular}
\end{exam}
\noindent Hence\\
\(\xymatrix{e_1M\ar[r]^{f|_{e_1M}}\ar[d]^{\alpha\cdot-} & e_1M\ar[d]^{\alpha\cdot\Large-} \\
e_2M\ar[r]^{f|_{e_2M}} & e_2M}\)
\begin{rem}
\( f\left(\begin{tabular}{c}
     \(1 - 1\) \\ onto \\ isom.
\end{tabular}\right)\)\(\quad\Leftrightarrow\quad\) \(f|_{e_iM}\left(\begin{tabular}{c}
     \(1 - 1\) \\ onto \\ isom.
\end{tabular}\right)\quad\) for all \(i\).
\end{rem}
\begin{defin}
Let \((V, f)\) and \((V', f')\) be two representations of \(\Gamma\) over \(k\). A \underline{homomorphism} \(h:(V, f)\mapsto (V', f')\) is a collection of linear maps
\begin{align*}
    h(i): V(i) \mapsto V'(i)
\end{align*}
for all \(i\in\Gamma_0\), such that \(\forall\alpha :i\mapsto j\in\Gamma_1\) the following diagram commutes:
\begin{align*}
  \xymatrix{V(i)\ar[rr]^h(i)\ar[dd]^{f_\alpha} & & V'(i)\ar[dd]^{f'_\alpha} \\
 & \ar@(ur,dr) & \\
 V(j)\ar[rr]^{h(j)}& & V'(j)}  
\end{align*}
i.e. \(f'_\alpha h(i) = h(j)f_\alpha\quad\forall\alpha\in\Gamma_1\). \(h\) is a(n) isomorphism, monomorphism, epimorphism if \(h(i):V(i)\mapsto V'(i)\) are all isomorphisms, monomorphisms, epimorphisms respectively.
\end{defin}
\begin{exam}
\begin{enumerate}[(1)]
    \item \(\Gamma: \xymatrix{1\ar[r]^{\alpha}& 2}, k\) is a field
    \begin{enumerate}[(a)]
        \item .\xymatrix{k\ar@{-->}[r]^{a\cdot-}\ar[d]^{1} & k\ar[d]^{0} \\
        k\ar@{-->}[r]^{0}\ar@{=}[d] & k\ar@{=}[d]\\
        (V, f) & (V', f')}
        Here \(h(1) = a\cdot-\) and \(h(2) = 0\) so \(h = (a\cdot-, 0)\)
        \item .\xymatrix{k\ar@{-->}[r]^0\ar[d]^1 & 0\ar[d]^0 \\
        k\ar@{=}[d]\ar@{-->}[r]^0 &  k\ar@{=}[d]\\
        (V, f) & (V', f') } No non-zero homomorphisms
        \item .\xymatrix{k^2\ar@{-->}[r]^{\small\begin{pmatrix}1&0\\0&1 \end{pmatrix}}\ar[d]_{\small\begin{pmatrix}0&1\\1&0 \end{pmatrix}} & k^2\ar[d]^{\small\begin{pmatrix}0&1\\1&0 \end{pmatrix}} \\
        k^2\ar@{=}[d]\ar@{-->}[r]_{\small\begin{pmatrix}1&0\\0&1 \small\end{pmatrix}} &  k^2\ar@{=}[d]\\
        (V, f) & (V', f') }
        \(h = \left(\begin{pmatrix}1&0\\0&1\end{pmatrix}, \begin{pmatrix}0&1\\1&0\end{pmatrix}\right)\) is an isomorphism.
    \end{enumerate}
\item \(\Gamma: \xymatrix{ & 1\ar[dl]_\alpha\ar[dr]^\beta & \\
                            2\ar[dr]_\gamma & & 3\ar[dl]^\delta\\
                            & 4 & }
                            \quad
                            \xymatrix{ & k\ar[dl]_1\ar[dr]^1 & \\
                            k\ar[dr]_1 & & k\ar[dl]^1\\
                            & k\ar@{=}[d] & \\
                            &(V, f) & }
                            \quad
                            \xymatrix{ & k\ar[dl]_1\ar[dr]^1 & \\
                            k\ar[dr]_1 & & k\ar[dl]^0\\
                            & k\ar@{=}[d] & \\
                            & (V', f')& }\)\\
                            here we have no isomorphism between \((V, f)\) and \((V', f')\).
\end{enumerate}
\end{exam}

\subsection{Modules and representations}

$\Gamma = (\Gamma_0,\Gamma_1)$ - quiver, $k$ field.

$M$ left $k\Gamma$-module $\leadsto\begin{cases}
(V,f) \text{, representation of $\Gamma$}\\
V(i) = e_iM\\
\text{for\ } \alpha\colon i\to j \in \Gamma_1, \text{\ we have\ }  f_\alpha\colon
V(i)=e_iM\extto{\alpha\cdot -} e_jM = V(j)\\
f_\alpha(e_im) = \alpha e_im
\end{cases}$

$(V,f)$ representaion of $\Gamma\leadsto \begin{cases}
M = \oplus_{i\in\Gamma_0} V(i) \textrm{, $k\Gamma$-module}^*\\
m = (v_1,v_2,\ldots, v_n) \in M\\
e_im \extto{\text{def}}{=} (0,\ldots,0,v_i,0,\ldots,0)\\
\text{for $\alpha\colon i\to j$ in $\Gamma_1$, remember $\alpha =
  e_j\alpha e_i$}\\
\alpha m \extto{\text{def}}{=} (0,\ldots,0,f_\alpha(v_i),0,\ldots,0)
\text{\ with $f_\alpha(v_i)$ in the $j$-th coordinate}
\end{cases}$\medskip 
$^*$Can show: This induces a left $k\Gamma$-module structure on $M$
(see \cite[page 57]{ARS}). 
\begin{exam}
$\Gamma\colon \xymatrix{1\ar[r]^\alpha & 2}$, $k$ field.

$(V,f)\colon \xymatrix{k\ar[r]^1 & k} \leadsto M = k\oplus k = k^2$
\begin{align}
e_1\cdot (a,b) & = (a, 0)\notag\\
e_2\cdot (a,b) & = (0,b)\notag\\
\alpha\cdot(a,b) & = (0,a)\notag
\end{align}
Note: $k\Gamma e_1 = k\{ e_1,\alpha\}$.  Define $\varphi\colon M\to
k\Gamma e_1$ by letting 
\[\varphi(1,0) = e_1 \text{\ and\ } \varphi(0,1) = \alpha.\]

Have: 
\begin{align}
\alpha \varphi(a,b)  & = \alpha(ae_1 + b\alpha) =
a\underbrace{\alpha e_1}_{=\alpha} +
                       b\underbrace{\alpha^2}_{=0}\notag\\
& = a\alpha\notag\\
& = \varphi(0,a) = \varphi(\alpha(a,b))\notag
\end{align} 
Similarly, $e_i\varphi(a,b) = \varphi(e_i(a,b))$. This implies that
$\varphi$ is a $k\Gamma$-homomorphism.\medskip 

$\left.\begin{matrix}
\Ker \varphi = (0)\\
\Im\varphi = k\Gamma e_1
\end{matrix}\right\} \Rightarrow M\simeq k\Gamma e_1 \text{\ as a left $k\Gamma$-module}.$
\end{exam}

\begin{exam}
$\Gamma\colon \xymatrix{ & 1\ar[dl]_\alpha\ar[dr]^\beta & \\
2\ar[dr]_\gamma & & 3\ar[dl]^\delta\\
& 4 & }$, $k$ field.

$(V,f)\colon \xymatrix{ & k\ar[dl]_{f_\alpha = 1}\ar[dr]^{f_\beta=1} & \\
k\ar[dr]_{f_\gamma=\left(\begin{smallmatrix} 1\\ 0\end{smallmatrix}\right)} & & k\ar[dl]^{f_\delta=\left(\begin{smallmatrix} 0\\ 1\end{smallmatrix}\right)}\\
& k^2 & }$

$M = V(1) \oplus V(2) \oplus V(3) \oplus V(4) = k\oplus k\oplus
k\oplus k^2$.
\begin{align}
\alpha (v_1,v_2,v_3,v_4)  & = (0, v_1, 0, 0 )\notag\\
\gamma (v_1,v_2,v_3,v_4)  & = (0, 0, 0, (v_2, 0) )\notag\\
\gamma\alpha (v_1,v_2,v_3,v_4)  & = (0, 0, 0, (v_1, 0) )\notag
\end{align}
\end{exam}

\begin{exer} Show that $M\simeq k\Gamma e_1$ as a left
  $k\Gamma$-module.
\end{exer}

\subsection{Special representations}
\begin{itemize}
\item Zero representation:  
$\begin{cases} 
V(i) = (0), \text{\ for all $i\in\Gamma_0$},\notag\\
f_\alpha = 0, \text{\ for all $\alpha\in \Gamma_1$}.\notag
\end{cases}$
\item For each $i\in\Gamma_0$, we have a representation $T_i$ given by
$T_i(j) = \begin{cases} k, \text{\ if $j = i$}\notag\\
(0), \text{\ otherwise}\notag
\end{cases}$
and $f_\alpha = 0$ for all $\alpha\in \Gamma_1$. 

$T_i$ corresponds to a left $\Gamma$-module $S_i$:

$S_i\simeq k$ as a vector space and $e_j v = \begin{cases} v, \text{\
    if $j = i$},\notag\\ 0, \text{\ otherwise},\notag\end{cases}$ and
$\alpha v = 0$ for all $\alpha\in \Gamma_1$. 
\end{itemize}

\begin{recall}
$\Lambda$ $k$-algebra, $k$ field.

\begin{tabular}{lcr}
$M$ left $\Lambda$-module & $\Rightarrow$ & $M$ $k$-vector space\\
 $\cup$ & & \\
$N$ submodule & $\Rightarrow$ & $N\subseteq M$ subspace
\end{tabular}
\end{recall}
\begin{note}
$\dim_k S_i = 1 \Rightarrow S_i$ is a simple $k\Gamma$-module.
\end{note}
\begin{defin}
$\Lambda$ ring, $(0)\neq M$ left $\Lambda$-module.  The module $M$ is
\emph{indecomposable}\index{module!indecomposable} if 
\[M\simeq M_1\oplus M_2\]
implies that $M_1=(0)$ or $M_2 = (0)$.
\end{defin}
\begin{defin}
Let $V = (V,f)$ and $V' = (V',f')$ be two representations of a quiver
$\Gamma$. Define $W = (W,h) = V \oplus V'$, \emph{the direct sum of
  the representations}\index{representation!direct sum} $V$ and $V'$ by 
\[W(i) = V(i) \oplus V'(i)\]
and
\[h_\alpha = f_\alpha\oplus f'_\alpha \colon W(i) = V(i)\oplus
  V'(i)\to V(j)\oplus V'(j) = W(j)\]
for all $i\in \Gamma_0$ and for all $\alpha\in\Gamma_1$. 
\end{defin}

\begin{defin}
$(0)\neq V = (V,f)$ is an
\emph{indecomposable}\index{representation!indecomposable}
representation if 
\[V = V_1\oplus V_2\]
implies that $V_1=(0)$ or $V_2 = (0)$. 
\end{defin}

\begin{exam}
$\Gamma\colon \xymatrix{1\ar[r]^\alpha & 2}$, $k$ field.

\begin{itemize}
\item $\xymatrix{k^2\ar[r]^{\left(\begin{smallmatrix}1 & 0 \\ 0 &
          1\end{smallmatrix}\right)} & 2}
\simeq \xymatrix{k\ar[r]^1 & k}\oplus \xymatrix{k\ar[r]^1 & k}$
\item $\xymatrix{k\ar[r]^1 & k}$ indecomposable? Others?
\end{itemize}
\end{exam}

\subsection{Subrepresentations}

$\Gamma = (\Gamma_0,\Gamma_1)$ quiver, $k$ field.

$M$, $N$ $k\Gamma$-modules, $N\subseteq M$ submodule.

$\Rightarrow e_iN\subseteq e_iM$ subspace.

Given $\alpha\colon i\to j \in \Gamma_1$, the following diagram commutes

\[\xymatrix{
e_i N \ar@{^(->}[r] \ar[d]_{\alpha\cdot-} & e_i M\ar[d]^{\alpha\cdot
  -}\\
e_jN \ar@{^(->}[r] & e_j M
}\]
\begin{defin}
\begin{enumerate}[\rm(a)]
\item $(V,f)\subseteq (V',f')$ is a
\emph{subrepresentation}\index{representation!subrepresentation}\index{subrepresentation}
if
\begin{enumerate}[\rm(i)]
\item $V(i) \subseteq V'(i)$ subspace for all $i\in\Gamma_0$, 
\item \[\xymatrix{
V(i) \ar@{^(->}[r] \ar[d]_{f_\alpha} & V'(i)\ar[d]^{f'_\alpha}\\
V(j) \ar@{^(->}[r] & V'(j)}\]
for $\alpha\colon i\to j\in\Gamma_1$, that is, $f_\alpha =
f'_\alpha|_{V(i)}$. 
\end{enumerate}
\item If $(V,f)\subseteq (V',f')$ is a subrepresentation, then the
  \emph{factor representation}\index{representation!factor
    representation} $W=(W,f'')$ of $V$ and $V'$ is given as
\begin{enumerate}[\rm(i)] 
\item $W(i) = V'(i)/V(i)$, 
\item \[\xymatrix{
V(i) \ar@{^(->}[r] \ar[d]_{f_\alpha} & V'(i)\ar[d]^{f'_\alpha}\ar[r] & V'(i)/V(i)=W(i)\ar@{-->}[d]^{f''_\alpha} \\
V(j) \ar@{^(->}[r] & V'(j) \ar[r] & V'(j)/V(j)=W(j)}\]
where $f''_\alpha(v' + V(i)) = f'_\alpha(v') + V(j)$ for $\alpha\colon
i\to j\in\Gamma_1$. 
\end{enumerate}
\end{enumerate}
\end{defin}
Check: 
\begin{enumerate}[\rm(i)]
\item $f''_{\alpha}$ is well-defined.
\item $(W,f'')$ is a representation of ${\Gamma}$ over $k$. 
\item We have
\[\xymatrix{
(V,f) \ar@{^(->}[r]\ar@{~>}[d] & (V',f') \ar@{~>}[d]&  W = V'/V \ar@{~>}[d] \\
M_V \ar@{^(->}[r] & M_{V'} & M_W \simeq M_{V'}/M_V
}\]
\end{enumerate}

\begin{defin}
$\Lambda$ finite dimensional $k$-algebra, $k$ field.  Then $\Lambda$
is of \emph{finite representation type}\index{representation
  type!finite} if there is only a finite number of non-isomorphic
indecomposable finitely generated left $\Lambda$-modules.
\end{defin}

\begin{exam}
$\Lambda = k$.  The only indecomposable $\Lambda$-module is $k$. 
\end{exam}

\begin{exam}
$\Gamma\colon \xymatrix{1\ar[r]^\alpha & 2}$, $k$ field.

The indecomposable left $k\Gamma$-modules $\leftrightsquigarrow$ \parbox{6cm}{The
indecomposable representations of $\Gamma$ over $k$.} 

Let $(V,f) = V_1\extto{f} V_2$ is an indecomposable representation of
$\Gamma$ over $k$.

Know: $h_1\colon V_1 \extto{~} \Im f\oplus \Ker f$

In particular, $\xymatrix{V\ar@<1ex>[r]^f & \Im f\ar@<1ex>[l]^{f'}}$
such that $ff' = 1_{\Im f}$.

$h_1\colon V_1 \to \Im f \oplus \Ker f$ given by $v\mapsto (f(v), v -
f'f(v))$. 

\[\xymatrix{
V_1 \ar[rr]^f \ar[d]_{h_1} & &  V_2\ar[d]^{h_2=1_{V_2}} & \\
\Im f\oplus \Ker f \ar[rr]^-{(\nu,0)} & & V_2 & \\
& \simeq & & \\
\Im f\ar@{^(->}[rr]^\nu && V_2 & =(0) \text{\ (ii)}\\
 & \oplus & \\
\Ker f \ar[rr]^0 && 0 & = (0) \text{\ (i)}
}\]
where $\nu\colon \Im f \hookrightarrow V_2$.

\textbf{Case (ii)}: \[\xymatrix{
\Ker f \ar[r]^0\ar[d]^\simeq & 0\ar@{=}[d]\\
k^t \ar[r]^0 & 0} \simeq \xymatrix{(k\ar[r]^0 & 0)^t}\]

$(V,f)$ indecomposable $\Rightarrow$ $t = 1$ and $(V,f)\simeq
\xymatrix{k\ar[r]^0 & 0}$ 

\textbf{Case (i)}: $\xymatrix{\Im f \ar@{^(->}[r] & V_2}$

Know: $V_2 = \Im f\oplus V_2'$

\[\xymatrix{
\Im f \ar@{^(->}[rr] \ar@{=}[d] & &  V_2\ar@{=}[d] & \\
\Im f \ar[rr]^-{\left(\begin{smallmatrix}1_{\Im f} \\ 0\end{smallmatrix}\right)} & & V_2 & \\
& \simeq & & \\
\Im f\ar@{=}[rr] && \Im f & =(0) \text{\ (i)'}\\
 & \oplus & \\
0\ar[rr]^0 && V'_2 & = (0) \text{\ (i)''}
}\]

\textbf{Case (i)'}:
\[\xymatrix{
0 \ar[r]^0\ar@{=}[d] & V'_2\ar[d]^\simeq\\
0 \ar[r]^0 & k^t} \simeq \xymatrix{(0\ar[r]^0 & k)^t}\]
$(V,f)$ indecomposable $\Rightarrow$ $t = 1$ and $(V,f)\simeq
\xymatrix{0\ar[r]^0 & k}$ 

\textbf{Case (i)''}: 
\[\xymatrix{
\Im f \ar[r]^{1_{\Im f}}\ar[d]^\varphi & \Im f\ar[d]^\varphi\\
k^t \ar[r]^{1_{k^t}} & k^t} \simeq \xymatrix{(k\ar[r]^1 & k)^t}\]
$(V,f)$ indecomposable $\Rightarrow$ $t = 1$ and $(V,f)\simeq
\xymatrix{0\ar[r]^0 & k}$ 

\textbf{Check}: $\xymatrix{k \ar[r]^1 & k}$, $\xymatrix{k\ar[r]^0 & 0}$ and
$\xymatrix{0 \ar[r]^0 & k}$ are indecomposable. 

\textbf{Hence}: The only indecomposable representations are the ones
above  $\Rightarrow$ The only indecomposable left $k\Gamma$-modules
are $k\Gamma e_1$, $k\Gamma e_1/\langle \alpha e_1\rangle$ and $S_2 =
k\Gamma e_2$. 
\end{exam}

\begin{thm}
$k$ field, $\kar k = p$, $G$ finite group with $p\mid |G|$. Then,

$kG$ of finite representation type $\Leftrightarrow$ All $p$-Sylow
subgroups of $G$ are cyclic.
\end{thm}

\begin{thm}
$\Gamma$ connected quiver without oriented cycles, $k$  field.

$k\Gamma$ is of finite representation type $\Leftrightarrow$ The
underlying graph of $\Gamma$ is a Dynkin diagram.

\[\xymatrix{\mathbb{A}_n\colon & 1\ar@{-}[r] & 2\ar@{-}[r] & \cdots &
    \ar@{-}[r] & n-1 \ar@{-}[r] & n}\]

\[\xymatrix@R=3pt{
& 1\ar@{-}[dr] &  & & & & \\
\mathbb{D}_n\colon &  & 3\ar@{-}[r] & \cdots &  \ar@{-}[r] & n-1 \ar@{-}[r] & n\\
& 2\ar@{-}[ur] & & & & & }\]

\[\xymatrix{
& & & 4\ar@{-}[d] & & \\
\mathbb{E}_6\colon & 1\ar@{-}[r] & 2\ar@{-}[r] & 3\ar@{-}[r] &
   5 \ar@{-}[r] & 6}\]

\[\xymatrix{
& & & 4\ar@{-}[d] & & & \\
\mathbb{E}_7\colon & 1\ar@{-}[r] & 2\ar@{-}[r] & 3\ar@{-}[r] &
   5 \ar@{-}[r] & 6 \ar@{-}[r] & 7}\]

\[\xymatrix{
& & & 4\ar@{-}[d] & & & &\\
\mathbb{E}_6\colon & 1\ar@{-}[r] & 2\ar@{-}[r] & 3\ar@{-}[r] &
   5 \ar@{-}[r] & 6 \ar@{-}[r] & 7 \ar@{-}[r] & 8}\]
\end{thm}

	
\section{quiver with relations}
Can all algebras over a field k be represented as $k\G$?\\\newline
\underline{No}: 
$\Lambda = \faktor{k[x]}{\langle x^2\rangle} \ncong k\G \text{ for all quivers }\G
\text{dim}_{k}\Lambda = 2, \Lambda\text{ not semisimple}.\\\newline \text{Assume that}  \Lambda \cong k\G.\\
2= \text{dim}_kk\G \ge \text{\# vertices in } \G $





\printindex
\begin{thebibliography}{ARS}
\bibitem{ARS} Auslander, M., Reiten, I., Smal\o, S.\ O.,
  \emph{Representation theory of Artin algebras}. Corrected reprint of
  the 1995 original. Cambridge Studies in Advanced Mathematics,
  36. Cambridge University Press, Cambridge, 1997. xiv+425 pp. ISBN:
  0-521-41134-3; 0-521-59923-7.
\end{thebibliography}
\end{document}