%page 82

\section{Duality}
\begin{defin}
$\mathcal{C}$, $\mathcal{D}$ ($R$-)categories, $F\colon \mathcal{C} \to
\mathcal{D}$ a contravariant ($R$-)functor. Then $F$ is a
\emph{duality}\index{functor!duality}\index{category!duality} if there exists a contravariant ($R$-)functor $H:
\mathcal{D} \to \mathcal{C}$ such that \\ 
\centerline{$HF \simeq \id_\mathcal{C}$ and $FH \simeq
  \id_\mathcal{D}$}\\ 
as ($R$-)functors.
\end{defin}

Let $k$ be a field.  Then 
\[ D:= \Hom_k(-,k)\colon \Vec(k) \to \Vec(k) \] 
is a contravariant functor.  Let $V$ be a vector space over $k$.
Assume that $\dim_k V = t < \infty$. If $\{ v_i \}^t_{i=1}$ is a
$k$-basis for $V$, then $\{ v_i^*\}_{i=1}^t$ where $v_i^* \in D(V) =
\Hom_k(V, k)$ is given by 
\[v_i^*\left(\sum_{j=1}^t a_jv_j\right) = a_i, \quad a_j \in k\]
is a basis for $D(V)$, the \emph{dual basis}\index{dual basis}.\\ 
$\implies \dim_k V = \dim_k D(V) = \dim_k DD(V)$.\\
Define $\varphi_V$ by 
\[\xymatrix{
\varphi_V \colon V \ar[r] & DD(V) = \Hom_k(\Hom_k(V, k), k)\\
            v \ar@{|->}[r]& \varphi_V(v) \colon \Hom_k(V, k) \to k, 
}\]
where $\varphi_V(v)(f) = f(v)$ for $f \in D(V)$. 

\begin{exer}
(a) $\varphi_V$ is 1-1 ($\Rightarrow \varphi_V$ is an isomorphism for
$\dim_k V < \infty$). 

(b) $\varphi = \{ \varphi_V \}_{V \in \vec(k)} \colon \id_{\vec(k)} \to DD$ is an isomorphism of functors.
\end{exer}

\underline{Check:}
(i) $\Lambda$ finite dimensional $k$-algebra.\\
For $X \in \mod \Lambda$, then $D(X) = \Hom_k(X, k)$ is a left $\Lambda^{\op}$-module via
\[ (\lambda \cdot f)(x) = f(\lambda x) \] for $f \in D(X)$, $\lambda \in \Lambda^{\op}$ and $x \in X$.

(ii) $f\colon X \to Y \in \mod \Lambda$. Then \[D(f)\colon \xymatrix@C-1pc@R-1pc{
D(Y) \ar@{}[r]|-*[@]{=}& \Hom_k(Y, k) \ar[r]& \Hom_k(X, k)\ar@{}[r]|-*[@]{=} & D(X)\\
&g \ar@{|->}[r]& g\cdot f&}\] is a $\Lambda^{\op}$-homomorphism.\\
$\implies D = \Hom_k(-,k)\colon \mod\Lambda \to \mod\Lambda^{\op}$\\ 
and $D = \Hom_k(-,k)\colon \mod\Lambda^{\op} \to \mod\Lambda$. 

%page 83

$\left[
\begin{tabular}{rl}
\underline{Note:} & $X$ finitely generated $\Lambda$-module\\
& $\implies \dim_k X < \infty$\\
& $\implies \dim_k D(X) < \infty$\\
& $\implies D(X)$ finitely generated $\Lambda^{\op}$-module. 
\end{tabular}
\right]$

\begin{prop}
$\Lambda$ finite dimensional $k$-algebra, $k$ a field.\\
Then $D = \Hom_k(-,k)\colon \mod\Lambda \to \mod\Lambda^{\op}$ is a duality.
\end{prop}

Let $\Lambda = k\G/\langle \rho \rangle$ be a finite dimensional
$k$-algebra, $k$ a field. Then $D$ induces a duality on representations. 
\[\xymatrix{
\Rep(\G, \rho) & \ar@{}[l]|-*[@]{\in} (V, f) \ar@{.>}[d] \ar@{|->}[r] & F(V, f) \ar@{|->}[d] \ar@{}[r]|-*[@]{\in} & \mod\Lambda\\
\Rep(\G^{\op}, \rho^{\op}) & \ar@{}[l]|-*[@]{\in} H(D(F(V,f))) & \ar@{|->}[l] DF(V, f) \ar@{}[r]|-*[@]{\in} & \mod\Lambda^{\op}
}\] 

Suppose that $(\G, \rho)$ satisfies $J^t \subseteq \langle \rho
\rangle \subseteq J^2$ and  $\G_0 = \{ 1, 2, \ldots, n \}$.

Define $(\G^{\op}, \rho^{\op})$ by
\begin{itemize}
\item $\G_0^{\op} = \G_0$
\item $\G_1^{\op}$: for each arrow $\alpha\colon i \to j$ in $\G_1$, there is an arrow $\alpha^{\op} \colon j \to i$ in $\G_1^{\op}$
\item If $p= \alpha_1 \alpha_2 \cdots \alpha_{r-1}\alpha_r$ is a path in $\G$, let
\[p^{\op}= \alpha^{\op}_r \alpha^{\op}_{r-1} \cdots
  \alpha^{\op}_{2}\alpha^{\op}_1\] 
be a path in $\G^{\op}$.
\end{itemize} 

Then $\Lambda^{\op} \simeq k\G^{\op}/\langle \rho^{\op} \rangle$ and $(\G^{\op}, \rho^{\op})$ is equivalent to $\mod\Lambda^{\op}$.\\
Let $(V, f) \in \Rep(\G, \rho)$. Then $F(V,f) \in \mod\Lambda$ and
$DF(V,f) \in \mod\Lambda^{\op}$. The underlying vector space of $DF(V,
f)$ is $D(\bigoplus\limits_{i=1}^n V(i))$. 
 
The representation $HDF(V,f) = (V', f')$ corresponding to $DF(V,f)$
is given by 
\[ V'(i) = e_i^{\op}DF(V,f) = e_i^{\op}\Hom_k\left(\bigoplus\limits_{j=1}^n V(j), k\right) \]
\[ \nu_j: \xymatrix@C-1pc{V_j \ar@{^{(}->}[r] & \bigoplus\limits_{l=1}^n V(l)}, \;\;\;\; g \in \Hom_k\left(\bigoplus_{j=1}^n V(j), k\right) \]

%page 84

Then 
\begin{equation*}
\begin{split}
e_i^{\op}g(v_1, v_2, \ldots, v_n) &= g(e_i(v_1, v_2, \ldots, v_n))\\
&= g((0,0, \ldots,0, v_i,0, \ldots, 0 ))\\
&= g\nu_i(v_i)
\end{split}
\end{equation*}

\[\xymatrix{
e_i^{\op}g \ar@{~>}[r]& g\nu_i \in D(V(i)) = \Hom_k(V(i), k)
}\]
\[ V'(i) = D(V(i)) = \Hom_k(V(i), k) \]
\[ DF(V, f) \ni g \mapsto g\nu_i \]
\[ \alpha^{\op}\colon j \to i \]

\[\xymatrix@R-1pc{
h \ar@{}[d]|-*[@]{=}  & f\cdot f_\alpha \ar@{}[d]|-*[@]{=}\\
g\nu_j \ar@{|->}[r] & g \nu_j f_\alpha\\
V'(j) = D(V(j)) \ar@{}[d]|-*[@]{\simeq} \ar[r] & V'(i) = D(V(i)) \ar@{}[d]|-*[@]{\simeq}\\
e_j^{\op}DF(V, f) \ar[r]^{\alpha^{\op} \cdot -} & e_i^{\op}DF(V,f)\\
\ar@{|->}@/^5pc/[uuu] g \ar@{|->}[r] & g\nu_j(f_\alpha(v_i))
}\]
\begin{equation*}
\begin{split}
(\alpha^{\op}\cdot g)(v) &= g(\alpha \cdot v) = g(\alpha(v_1, v_2, \cdots, v_n))\\
&= \xymatrix@C-2pc@R-2pc{g((0, \cdots, & 0, f_\alpha(v_i), 0, & \cdots , 0))\\
&& \ar@/^1pc/[ul] j\text{-th coordinate} }\\
&= g\nu_j(f_\alpha(v_i))
\end{split}
\end{equation*}
\\
\[\xymatrix@C-1pc@R-1pc{
f'_{\alpha^{\op}} \ar@{}[d]|-*[@]{=} \ar@{}[r]|-*[@]{:} & V'(j) \ar@{}[r]|-*[@]{=} & D(V(j)) \ar[r] & D(V(i)) \ar@{}[r]|-*[@]{=} & V'(i)\\
D(f_\alpha) 
}\]
$\implies (V', f') = \Big( \{ D(V(i)) \}_{i=1}^n, \{ D(f_\alpha) \}_{\alpha \in \G_1} \Big)$.

\begin{exer}
$V$, $W$ finite dimensional vector spaces $B$, $B'$ basis for $V$ and $W$ respectively. $B^*$, $B'^*$ dual basis for $D(V)$ and $D(W)$.\\
Let $f\colon V \to W$. Suppose that \[m_B^{B'}(f) = A\]
Then for $D(f)\colon D(W) \to D(V)$ we have the matrix representation \[ m_{B'^*}^{B^*}(D(f)) = A^T \] for the dual map $D(f)$.
\end{exer}

%page 85

\begin{exam}
$\G: \xymatrix{
1 \ar@/^0.5pc/[r]^\alpha \ar@/_0.5pc/[r]_\beta & 2 \ar[r]^\gamma & 3
}$, $\rho = \{ \gamma\beta \}$, $\Lambda = k\G/\langle \rho \rangle$, $k$ a field.\\
$\G^{\op}: \xymatrix{
1 \ar@{<-}@/^0.5pc/[r]^{\alpha^{\op}} \ar@{<-}@/_0.5pc/[r]_{\beta^{\op}} & 2 \ar@{<-}[r]^{\gamma^{\op}} & 3
}$, $\rho^{\op} = \{ \beta^{\op}\gamma^{\op} \}$

\begin{minipage}{0.5\textwidth}
\[\Lambda\overline{e}_1 : \xymatrix{
k \ar@/^0.5pc/[r]^{\left(\begin{smallmatrix} 1\\0 \end{smallmatrix}\right)} \ar@/_0.5pc/[r]_{\left(\begin{smallmatrix} 0\\1 \end{smallmatrix}\right)} & k^2 \ar[r]^{\left(\begin{smallmatrix} 1 & 0 \end{smallmatrix}\right)} & k
}\]
\vspace*{4pt}
\[
\Lambda\overline{e}_2 : \xymatrix{
0 \ar@/^0.5pc/[r]\ar@/_0.5pc/[r] & k \ar[r]^1 & k
}\]
\vspace*{8pt}
\[
\Lambda\overline{e}_3 : \xymatrix{
0 \ar@/^0.5pc/[r]\ar@/_0.5pc/[r] & 0 \ar[r] & k
}\]
\end{minipage}
\rule[-50pt]{1.5pt}{100pt}
\begin{minipage}{0.5\textwidth}
\[D(\Lambda\overline{e}_1) : \xymatrix{
k \ar@{<-}@/^0.5pc/[r]^{\left(\begin{smallmatrix} 1 & 0 \end{smallmatrix}\right)}
\ar@{<-}@/_0.5pc/[r]_{\left(\begin{smallmatrix} 0 &1 \end{smallmatrix}\right)} &
k^2 \ar@{<-}[r]^{\left(\begin{smallmatrix} 1 \\ 0 \end{smallmatrix}\right)} & k 
}\]
\vspace*{4pt}
\[
D(\Lambda\overline{e}_2) : \xymatrix{
0 \ar@{<-}@/^0.5pc/[r]\ar@{<-}@/_0.5pc/[r] & k \ar@{<-}[r]^1 & k
}\]
\vspace*{8pt}
\[
D(\Lambda\overline{e}_3) : \xymatrix{
0 \ar@{<-}@/^0.5pc/[r]\ar@{<-}@/_0.5pc/[r] & 0 \ar@{<-}[r] & k
}\]
\end{minipage}
\end{exam}

\begin{lem}
$\Lambda$ finite dimensional $k$-algebra, $k$ a field.
\begin{enumerate}[\rm(a)]
\item $\eta: \xymatrix@C-1pc{0 \ar[r] & A \ar[r]^f & B \ar[r]^g & C \ar[r] & 0}$ exact in $\mod\Lambda$\\
$\iff \xymatrix@C-1pc{0 \ar[r] & D(C) \ar[r]^{D(g)} & D(B) \ar[r]^{D(f)} & D(A) \ar[r] & 0}$ exact in $\mod\Lambda^{\op}$.

\item $S$ simple $\Lambda$-module $\iff D(S)$ simple $\Lambda^{\op}$-module.

\item $l(A) = l(D(A))$ for $A \in \mod\Lambda$. 
\end{enumerate}
\end{lem}
\begin{proof}[Sketch of proof]
(a) Use that $\eta$ splits as a sequence of $k$-modules, that $D$ preserves $k$-dimesnion and that $D^2 \simeq \id_{\mod\Lambda}$.

(b) Use (a).

(c) Induction on length.
\end{proof}

Given a statement $S$ in a category $\mathcal{C}$, then \emph{the
  dual statement $S^*$}\index{dual statement} is the statement about $\mathcal{C}$ reversing
the direction of all morphisms and replacing all compositions
$\alpha\beta$ of morphisms with $\beta\alpha$. 

\begin{exam}\[\]
$P$ projective:\\
$\vcenter{\xymatrix{
&P \ar@{.>}[dl]_{\exists h} \ar[d]^f\\
\text{exact } B \ar[r]^-g & C \ar[r] & 0
}}$ $\forall g$, $\forall f$ $\implies \exists h\colon P \to B$ such that $gh=f$.\\
\\
$I$ injective:\\
$\vcenter{\xymatrix{
&I \ar@{<.}[dl]_{\exists h} \ar@{<-}[d]^f\\
\text{exact } B \ar@{<-}[r]^-g & C \ar@{<-}[r] & 0
}}$ $\forall g$, $\forall f$ $\implies \exists h\colon B \to I$ such that $hg=f$.
\end{exam} 

%page 86

\section{Injective modules}

\begin{defin}
$\Lambda$ a ring, $I \in \Mod\Lambda$. Then $I$ is
\emph{injective}\index{module!injective} if for every monomorphism $i\colon A \to B$ in
$\Mod\Lambda$ and every homomorphism $f\colon A \to I$ there exists an
$h:B\to I$ such that $hi = f$ i.e. the following diagram commutes. 

\[\xymatrix{
0 \ar[r] & A \ar[r]^i \ar[d]_f & B \ar@{-->}[dl]^{\exists h}\\
& I
}\]
\end{defin}

\begin{prop}
\label{prop:53}
$\Lambda$ finite dimensional $k$-algebra, $k$ a field, $P \in \mod \Lambda$.
\begin{enumerate}
\item[(a)] $P$ projective $\Lambda$-module $\iff$ $D(P)$ is injective $\Lambda^{\op}$-module.

\item[(b)] Any $\Lambda$-module $M \in \mod\Lambda$ is a submodule of
  an injective $\Lambda$-module in $\mod\Lambda$. 
\end{enumerate}
\end{prop}
\begin{proof} (a) \underline{$\Rightarrow$:} $P$ projective. Consider 
$\vcenter{\xymatrix{
0 \ar[r] & A \ar[r]^i \ar[d]_f & B\\
& D(P)
}}$ in $\Mod\Lambda^{\op}$.\\ Apply $D$:
\[\xymatrix{
& D^2(P) \ar@{-->}[dl]_{\exists h} \ar[d]^{D(f)} \ar@{}[r]|-*[@]{\simeq} & P & \ar[l] \text{projective}\\
D(B) \ar[r]^{D(i)} & D(A) \ar[r] & 0
}\]

$\implies$

\[\xymatrix@C-1pc@R-1pc{
\ar[dd]_f A \ar[rr]^i \ar[dr]_{\alpha_A} && B \ar[dr]^{\alpha_B}\\
& D^2(A) \ar[dd]^{D^2(f)} \ar[rr]^{D^2(i)} && D^2(B) \ar[ddll]^{D(h)}\\
D(P) \ar@{}[dr]|-*[@]{\xrightarrow{\sim}}_{\alpha_{D(P)}}\\
& D^3(P)
}\]

$\implies \alpha_{D(P)}f = D(h)\alpha_Bi$\\
Composing both sides with $\alpha_{D(P)}^{-1}$ we obtain\\
$f = \left( \alpha_{D(P)}^{-1}D(h)\alpha_B \right)i$\\
$\implies D(P) \in \mod\Lambda^{\op}$ is injective.\\
\\
$\underline{\Leftarrow:}$ Baer criterion and $\Lambda$ noetherian.\\
$\implies$ we can restrict ourselves to finitely generated modules. Use dual arguments.

\item[(b)] Let $M \in \mod\Lambda$. Then $D(M) \in
  \mod\Lambda^{\op}$. Let $\xymatrix{P \ar@{->>}[r] & D(M)}$ be the
  projective cover of $D(M)$. Then  
\[\xymatrix@C-1.5pc@R-2pc{M \ar[rrr]^-{\alpha_M}_-\sim &&& D^2(M)
    \ar@{^{(}->}[rrr] &&& D(P) \ar@{}[r]|-*[@]{\in} & \mod\Lambda
    \hspace{1cm}\\ 
              &&&        &&&      & \ar@/^1pc/[ul] \text{injective $\Lambda$-module}}\]

\end{proof}

% page 87

\begin{rem}
$\Lambda$ a ring, $M \in \Mod\Lambda$.\\
\underline{Can be shown:} $\xymatrix{M \ar@{^{(}->}[r] & I}$, $I$ injective $\Lambda$-module.\\
However: Even if $M$ is a finitely generated $\Lambda$-module $I$ need
not be a finitely generated $\Lambda$-module. 
\[\left( \mathbb{Z}\text{-modules} = \Ab : \xymatrix{\mathbb{Z} \ar@{^{(}->}[r] & \mathbb{Q}} \right)\]
\end{rem}

\begin{defin}
$\Lambda$ a ring
\begin{enumerate}
\item[(a)] $A \subseteq X$ $\Lambda$-modules. Then $A$ is an
  \emph{essential submodule of $X$}\index{submodule!essential}, if for each non-zero
  submodule $B$ of $X$, then $A\cap B \neq (0)$. 
\item[(b)] A monomorphism $i\colon A \to X$ is essential if $i(A)$ is an essential submodule of $X$.

\item[(c)] A monomorphism $i\colon A \to I$ is an \emph{injective
    envelope}\index{injective envelope} if 
\begin{enumerate}
\item[(i)] $I$ is injective,
\item[(ii)] $i$ is an essential monomorphism.
\end{enumerate} 
\end{enumerate}
\end{defin}

\underline{Can be shown:} (MA3204) $\Lambda$ a ring. Every $\Lambda$-module has an injective envelope.

\begin{defin}
$\Lambda$ artin $R$-algebra, $A \in \mod\Lambda$. The \emph{socle
  of $A$}\index{socle}, $\soc A$ is the sum of all simple submodules of $A$. 
\end{defin}

\begin{exam}
$\G: \vcenter{\xymatrix@C-2pc@R-1pc{
&1 \ar[dl] \ar[dr]&\\
2&&3}}$, $k$ a field, $\Lambda = k\G$
\[
\xy
\POS(20, 6)

\xymatrix@C-2pc{
&k \ar[dl]_1="l" \ar[dr]^1="r" \\
k&&k
}

\POS(0,0)
\xymatrix{\Lambda e_1 \ar@{{}{ }{}}@(ul,ur)^{}="a" \ar@{~>}"a";"l"}

\POS(40, 6)

\xymatrix{
&0 \ar[dl]_{}="b" \ar[dr] \ar@{<-^{)}}"r";"b" \\
k&&0
}

\POS(60, 6)

\xymatrix{
&0 \ar[dl]_{}="b" \ar[dr]\\
0&&k
}

\POS(55,0)
\xymatrix{+}

\POS(75, 0)

\xymatrix@C-1pc@R-1pc{=  \Lambda e_2 & \oplus & \Lambda e_3\\
                         &  \soc \Lambda e_1 \ar@{}[u]|-*[@]{=}}

\endxy\]
\end{exam}

\begin{note}
(1) In general, $\soc S = S$, when $S$ is a (semi-)simple $\Lambda$-module.

(2) $\soc A \subseteq A$ is a semisimple submodule of $A$.

(3) $\Lambda$ artin $R$-algebra, $A \in \mod\Lambda$, $A \neq (0)$.\\
Since $l(A) < \infty$, then $\exists$ a simple $\Lambda$-submodule $S
\subseteq A$, that is, $\soc A \neq (0)$. Furthermore $A \neq (0) \iff
\soc A \neq (0)$. Also $D(A / \mathfrak{r}A) \simeq \soc D(A)$.  
\end{note}

%page 88

\begin{lem}
\label{lem:54}
$\Lambda$ artin $R$-algebra, $A \subseteq X \in \mod\Lambda$ a submodule.\\
TFAE
\begin{enumerate}[\rm(a)]
\item $A$ essential submodule of $X$, 
\item $\soc X \subseteq A$, 
\item $\soc A \subseteq \soc X$.
\end{enumerate} 
\end{lem}
\begin{proof}
\underline{(a) $\Rightarrow$ (c):} $A \subseteq X$ essential submodule\\
\begin{equation*}
\begin{split}
S \subseteq X \text{ simple} &\implies S \neq (0)\\
&\implies (0) \neq S\cap A \subseteq S\\
\text{Simple } &\implies S \cap A = S \subseteq A\\
\implies \sum_{\begin{smallmatrix} S \subseteq X\\ S \text{ simple} \end{smallmatrix}} S &= \soc X \subseteq A\\
A \subseteq X &\implies \soc A \subseteq \soc X\\
\text{Hence, } \soc A &= \soc X.
\end{split}
\end{equation*}
\underline{(c) $\Rightarrow$ (b):} Obviously, $\soc A = \soc X$.\\
$\implies \soc X \subseteq \soc A \subseteq A$.\\
\\
\underline{(b) $\Rightarrow$ (a):} Assume that $\soc X \subseteq A$.\\
Let $(0) \neq B \subseteq X$. Then $(0) \neq \soc B \subseteq \soc X$\\
Hence, $(0) \neq \soc B = \soc B \cap \soc X \subseteq \soc B \cap A \subseteq B \cap A \implies B \cap A \neq (0)$.
\end{proof}

\begin{prop}
\label{prop:55}
$\Lambda$ artin $R$-algebra, $(0) \neq A \in \mod\Lambda$
\begin{enumerate}[\rm(a)]
\item $\xymatrix{A \ar@{^{(}->}[r] & I}$ injective envelope $\iff I$ injective and $\soc A = \soc I$.

\item Injective envelopes are unique up to isomorphism.

\item Injective envelope of $A$ $\simeq$ injective envelope of $\soc
  A$. 
\end{enumerate}
\end{prop}
\begin{proof}
(a) Use the definition of injective envelope and Lemma \ref{lem:54}. 

(b) Let $\xymatrix{A \ar@{^{(}->}[r]^{\nu_i} & I_i}$ be two injective envelopes 
\[\xymatrix@C-1pc@R-0.3pc{
&0\ar[dr]\\
0 \ar[rr] && A \ar[rr]^{\nu_1} \ar[dr]_{\nu_2} && I_1 \ar@{-->}[dl]^{\exists \varphi}\\
&&& I_2
}\]

%page 89

Assume that $\Ker \varphi \neq (0)$.\\
$\implies A \cap \Ker\varphi \neq (0)$\\
$\implies \exists a \in A \setminus \{ 0 \}$ s.t. $0=\varphi(a) = \varphi\nu_1(a) = \nu_2(a) \implies a= 0$. Contradiction!\\
$\implies \Ker\varphi = (0) \implies I_1 \simeq 
\xymatrix@C-2pc@R-2pc{\Im \varphi \ar@{}[r]|-*[@]{\subseteq} & I_2\\
& \ar@/^1pc/[ul] \text{injective}} $

\begin{recall}
$\xymatrix{0 \ar[r] & I \ar@{^{(}->}[r] & B \ar[r] & C \ar[r] & 0}$ exact, and $I$ injective.\\
$\implies B = I \oplus B'$
\end{recall}

$\implies I_2 = \Im\varphi \oplus I_2'$.\\
$\implies \xymatrix@C+2pc{A \ar@{^{(}->}[r]^{\nu_2 = \varphi\nu_1} & I_2} = \Im\varphi \oplus I_2'$ and $\Im\nu_2 \subseteq \Im \varphi\nu_1$.\\
$\implies A \cap I_2' = (0)$.\\
$A$ essential submodule of $I_2$.\\
$\implies I_2' = (0)$ and $\Im\varphi = I_2$.\\
$\implies \varphi$ is an isomorphism and therefore injective envelopes are unique up to isomorphism.

(c) Consider
\[
\xy

\xymatrix{0 \ar[r] & \soc A \ar@{^{(}->}[r]^{\nu}
  \ar@{^{(}->}[d]_{i}="env"  & A \ar@{-->}[dl]^{\begin{smallmatrix}
      \exists f \text{, since }\\I(\soc A)\text{ is
        inj.} \end{smallmatrix}}\\ 
& I(\soc A)}

\POS(0, -6)

\xymatrix{{\begin{smallmatrix} \text{inj}\\\text{envelope}\end{smallmatrix}} \ar@{{}{ }{}}@(ul,ur)^{}="a" \ar"a";"env"}

\endxy
\]
\begin{enumerate}
\item[$\cdot$] $f\nu=i$ 1-1 $\implies \Ker f \cap \soc A = (0)$.\\ 
$\soc A \hookrightarrow A$ essential $\implies \Ker f = (0) \implies f$ 1-1.

\item[$\cdot$] Let $(0) \neq A' \subseteq I(\soc A)$. WTS: $f(A) \cap A' \neq (0)$.\\
\underline{We have:} $f(A) \cap A' \supseteq \soc A \cap A' \neq (0)$, since $\soc A$ is an essential submodule of $I(\soc A)$.\\
$\implies f(A)$ is an essential submodule of $I(\soc A)$.\\
$\implies \xymatrix{A \ar[r]^-{f} & I(\soc A)}$ is an injective
envelope. 
\end{enumerate}
\end{proof}

\begin{lem}
\label{lem:56}
$\Lambda$ artin $R$-algebra, $A \in \mod\Lambda$, $\rad\Lambda =
\mathfrak{r}$. Then 
\[\soc A = \{ a \in A \mid \mathfrak{r}\cdot a = (0) \} = S_A\]
\end{lem}
%page 90

\begin{proof}
$\soc A$ semisimple $\implies \mathfrak{r}\soc A = (0) \implies \soc A \subseteq S_A$\\
$S_A$ is a submodule of $A$.\\
$\mathfrak{r}S_A=(0) \implies S_A$ is a semisimple submodule\\
$\implies S_A \subseteq \soc A \implies \soc A = S_A$ 
\end{proof}

\begin{exer}
$\Lambda$ artin $R$-algebra, $A, A_1, A_2 \in \mod\Lambda$
\begin{enumerate}[\rm(a)]
\item $\soc A \simeq \Hom_\Lambda(\Lambda/\mathfrak{r}, A)$.
\item $\soc(A_1 \oplus A_2) = \soc A_1 \oplus \soc A_2$.
\end{enumerate}
\end{exer}

\begin{prop}
\label{prop:57}
$\Lambda$ finite dimensional $k$-algebra, $k$ a field.\\
$\xymatrix{P \ar[r]^f & A}$ is a projective cover in
$\mod\Lambda$ \[\Updownarrow\] $\xymatrix{D(A) \ar[r]^{D(f)} & D(P)}$
is a injective envelope in $\mod\Lambda^{\op}$.  
\end{prop}
\begin{proof}
Use $X \in \mod\Lambda \implies \soc D(X) \simeq D(X/\mathfrak{r}X)$. 
\end{proof}

Using duality one can show the following for $\Lambda$ a finite dimensional
$k$-algebra over a field $k$:
\begin{enumerate}[\rm(a)]
\item $A,B \in \mod\Lambda \implies I(A\oplus B) = I(A) \oplus I(B)$
\item $I$ injective in $\mod\Lambda$:\\
\centerline{$I$ indecomposable $\iff$ $\soc I$ simple.}
\item There is a 1-1 correspondance between isomorphism classes
  of simple $\Lambda$-modules and isomorphism classes of
  indecomposable injective $\Lambda$-modules: 
\[\xy

\POS(-25,0)
\xymatrix@C-1pc@R-1pc{
S \ar@{|->}[d] & \soc I \ar@{<-|}[d]\\
I(S) & I
}

\POS(0, 7)

\xymatrix@R+3pc{
\{ \text{isomorphism classes of simple modules} \} \ar@{<->}[d]^{1-1}\\
\{ \text{isomorphism classes of indecomposable injective modules} \}
}

\endxy\]
\end{enumerate}


%page 91
\subsection{The socle of a representation}
$(\G, \rho)$ quiver with relations $\rho$, $k$ a field, $J^t \subseteq
\langle \rho \rangle \subseteq J^2$, $\G_0 = \{ 1, 2, \ldots, n
\}$. $\Lambda = k\G/\langle \rho \rangle$, $\mathfrak{r} = \rad\Lambda
= J/\langle \rho \rangle \subseteq \Lambda$. 
\[\xymatrix{
(V, f) \ar@{|->}[rr] && F(V,f) \ar@{}[d]|-*[@]{\supseteq} \ar@{}[r]|-*[@]{=} & \bigoplus\limits_{i=1}^n V(i)\\
\Rep(\G, \rho) \ar@/^0.5pc/[r]^F & \ar@/^0.5pc/[l]^H \mod\Lambda & \soc F(V,f) \ar@{}[d]|-*[@]{=}\\
               &             & \{ m \in F(V,f) \mid \mathfrak{r}\cdot m = (0) \}
}\]
Let $m= (v_1, v_2, \ldots, v_n) \in F(V, f)$. Then\\
$ m \in \soc F(V, f) \iff \overline{\alpha} \cdot m = 0$, for all
$\alpha \colon i \to j \in \G $ since $\mathfrak{r} = J/\langle \rho
\rangle$, ($J = \langle$arrows$\rangle$).\\ 
$\iff (0, \ldots,0, 
\xymatrix@R-2pc@C-2.5pc{
f_\alpha(v_i), & 0, \ldots, 0)\\ 
& j\text{-th coord} \ar@/^1pc/[ul]}=0$, $\forall \alpha\colon i \to j \in \G$.\\
$\iff f_\alpha(v_i) = 0$, for all $\alpha \colon i \to j \in \G$.\\
$\iff v_i \in \bigcap\limits_{\alpha \colon i \to j \in \G} \Ker
f_\alpha$, $\forall i = 1, 2, \cdots, n$. 

Then $H(\soc F(V, f))$ is given by letting 
\[V'(i) = \bigcap\limits_{\alpha \colon i \to j} \Ker f_\alpha
  \subseteq V(i)\] 
and 
\[f'_\alpha = f_\alpha\mid_{V'(i)} = 0.\]

\begin{exam}
\[\]
$\G: \xymatrix{
1 \ar@/^0.5pc/[r]^\alpha \ar@/_0.5pc/[r]_\beta & 2 \ar[r]^\gamma & 3
}$, $\rho = \{ \gamma\beta \}$, $k$ a field, $\Lambda = k\G/\langle \rho \rangle$
\[
\Lambda \overline{e}_1: \vcenter{\xymatrix{
k \ar@/_0.5pc/[d]_-{\left(\begin{smallmatrix} 1\\0 \end{smallmatrix}\right)} \ar@/^0.5pc/[d]^-{\left(\begin{smallmatrix} 0\\1 \end{smallmatrix}\right)}\\
k^2 \ar[d]^-{\left(\begin{smallmatrix} 0 & 1 \end{smallmatrix}\right)}\\
k
}}
\hspace{2cm}
\Lambda \soc\overline{e}_1: \vcenter{\xymatrix{
0 \ar@/_0.5pc/[d]_-{0} \ar@/^0.5pc/[d]^-{0}\\
k \oplus (0) \ar[d]^-{0}\\
k
}}
\]
\end{exam}