\section{quiver with relations}
Can all algebras over a field k be represented as $k\G$?\\\newline
\underline{No}: 
$\Lambda = \frac{k[x]}{\langle x^2\rangle} \ncong k\G \text{ for all quivers }\G
\text{dim}_{k}\Lambda = 2, \Lambda\text{ not semisimple}.\\\newline \text{Assume that}  \Lambda \cong k\G.\\
2= \text{dim}_kk\G \ge \text{\# vertices in } \G $

$ \G \colon \xymatrix{1& 2} \implies k\G$ is semisimple \blitza\\
$\implies \G$ has one vertex, $\xymatrix{1\ar@(ur,dr)^\alpha} $ $\implies$ $\text{dim}_kk\G = \infty$ \blitza\\
\newline But, $\G\colon \xymatrix{1 \ar@(ur,dr)^\alpha}$, $\Lambda \simeq \frac{k\G}{\langle \alpha^2\rangle} $\\
\rule{\textwidth}{1pt}\\
Let $\G=(\G_0,\G_1)$ be a quiver, k field.\\
\begin{defin}
\begin{enumerate}[(a)]
	~\\ \item A \underline{relation} $\sigma$ in the quiver $\G$ over k is a k-linear combination of paths
	\begin{center}
		$\sigma = a_1p_1 + a_2p_2+ \cdots + a_tp_t$\\
	\end{center}
	where $a_i\in k$, $e(p_i)=e(p_1)$ and $s(p_i)=s(p_1)$ for all $i$, and $l(p_i)\geq 2$ (the length of the path $p_i$ )
	\item if $\varrho = \{\sigma\}_{l \in T}$ is a set of relations in $\G$ over k, then $(\G,\varrho)$ is a \underline{quiver with relations over k.}  
\end{enumerate}
\end{defin}
\begin{exam}
	$\xymatrix{ & 1\ar[dl]_\alpha\ar[dr]^\beta & \\
		\G \colon 2\ar[dr]_\gamma & & 3\ar[dl]^\delta\\
		& 4 & }$\newline $k$ field, $\sigma = \gamma\alpha - \delta\beta$.\\\newline
		$ \G=\frac{k\G}{\langle\sigma\rangle}$. Let $M$ be a left $k\Lambda$-module. Any left $\Lambda$-module is a left $k\Lambda$-module, since $k\G \xrightarrow{\pi} \frac{k\G}{\large\sigma\rangle} = \Lambda$.\\
		$\implies M$ gives rise to a representation of $\G$.\\
		
	$\xymatrix{ & e_1M\ar[dl]_{f_\alpha = \alpha\cdot-}\ar[dr]^{\beta\cdot-=f_\beta} & \\
	 e_2M\ar[dr]_{f_\gamma=\gamma\cdot-} & & e_3M\ar[dl]^{\delta\cdot-=f_\delta}\\
	& 4 & }$\newline

$\sigma\in k\G, m \in M.$\\ $ \sigma\cdot m\defeq\pi(\sigma)\cdot m = 0\cdot m = 0, \forall m \in M$\\
$m=e_1m+e_2m+e_3m+e_4m$, $sigma = e_4\sigma e_1$\\
$0 = \sigma\cdot m = (\gamma\alpha-\delta\beta)e_1m = \gamma(\alpha e_1m)-\delta(\beta e_1m) = f_\gamma f_\alpha(e_1m)-f_\sigma f_\beta(e_1m) = \underbrace{f_\gamma f_\alpha-f_\sigma f_\beta}_{f_\sigma}(e_im) \implies f_\sigma(e_1M)=0 \implies f_\sigma = 0 $.\\
Hence $\Lambda$-module $M$ corresponds to a representation of $\G$ satisfying the relation $\sigma(f_\sigma=0)$.\\
Conversely, we claim that a representation $(V,f)$ of $\G$ 	such that
\begin{center}
	$f_\sigma=f_\gamma f_\alpha-f_\sigma f_\beta=0$
\end{center} 
 gives a module over $\Lambda$.\\\newline
\underline{Recall:} $I \subseteq R$ ideal: $\frac{R}{I}$-module $M$ is the same as an $R$-module $M$ such that $I\cdot M = (0)$.\\
\begin{center}
	$M = V(1) \oplus V(2) \oplus V(3) \oplus V(4) \leftarrow k\G$-module\\
	$e_1\cdot(v_1,v_2,v_3,v_4)=(v_1,0,0,0)$\\
	$\alpha\cdot(v_1,v_2,v_3,v_4)=(0,f_\alpha(v_1),0,0)$\\
\end{center}
~\\$\sigma\cdot(v_1,v_2,v_3,v_4) = (\gamma\alpha-\delta\beta)\cdot(v_1,v_2,v_3,v_4) = (0,0,0,f_\gamma f_\alpha(v_1)-f_\sigma f_\beta(v_1)) =  (0,0,0,(f_\gamma f_\alpha-f_\sigma f_\beta)(v_1)) = (0,0,0,0) $\\\newline
$\implies M$ is a $\Lambda$-module $(\Lambda=\frac{k\G}{\langle\sigma\rangle})$.
\end{exam}

\begin{exam}
$\G\colon \xymatrix{1 \ar@(ur,dr)^\alpha}, P=\{\alpha^2 \}$, $k$ field. $ \Lambda=\frac{k\G}{\langle\alpha^2\rangle}$. Find all induced $\Lambda$-modules.\\
$M$ left $\Lambda$-module $\leadsto 
(V,f) $ representation of $\G$ satisfying the relation $\alpha^2, $i.e.  $ f_{\alpha^2}=(f_\alpha)^2$\\  
	
	$\xymatrix{V \ar@(ur,dr)^{f_\alpha}}, (f_\alpha)^2=0$\\\newline
	$\implies$ The minimal polynomial of $f_\alpha$ is $x$ or $x^2$\\
	$\implies$ The ivariant factor of $f_\alpha$ is $x$ or $x^2$\\
	$\implies$ The matrix of $f_\alpha$ is similar to a direct sum of companion matrices of $x$ or $x^2$, $M_{(x)}=0$ and $M_{(x^2)}= \begin{pmatrix}0&0\\1&0\end{pmatrix}$\\~\\
	
Let T be the matrix of $f_\alpha$ w.r.t some basis $\beta$.\\
Then $\exists$ an invertible matrix $P$ such that\newline
	
$ T = P \underbrace{ \begin{pmatrix} r\begin{cases} \begin{pmatrix} 0 \cdots 0  \\\vdots \ddots \vdots \\0 \cdots 0 \end{pmatrix} \end{cases} & \overbrace{\text{\huge0}}^{2s}\\\text{\huge0} & 
\begin{pmatrix}  
\begin{pmatrix} 0 & 0\\ 1 & 0\\ \end{pmatrix} & & & \text{\huge0}  \\
& \begin{pmatrix} 0 & 0\\ 1 & 0\\ \end{pmatrix} & & \\
& &\ddots &  & \\
\text{\huge0} & & &\begin{pmatrix} 0 & 0\\ 1 & 0\\ \end{pmatrix} \\
\end{pmatrix} \end{pmatrix}}_{T_0}P^{-1} $ \\$\implies TP= PT_0$ \nolinebreak[4]\\\newline
$\xymatrix{V \ar@(ur,dr)^{T}} \iff \xymatrix{V \ar@(ur,dr)^{T_0}}$ isomorphisme of representation

\underline{Show:} $\xymatrix{k \ar@(ur,dr)^{0}} \iff \frac{k\G}{\langle\alpha\rangle}$ and $\xymatrix{k^2 \ar@(ur,dr)^{\begin{pmatrix}0&0\\1&0\end{pmatrix}}} \iff \frac{k\G}{\langle\alpha^2\rangle}$

	
	
\end{exam}