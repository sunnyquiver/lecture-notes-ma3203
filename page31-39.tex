\section{Radical}

\begin{defin}
$\Lambda$ ring.  The \emph{(left) radical of $\Lambda$} is the left ideal
\[\mathfrak{r} = \rad \Lambda = \cap_{\mathfrak{m} \text{maximal left ideal}} \mathfrak{m}\]
(Also called the Jacobsen radical of $\Lambda$).
\end{defin}

\textbf{Know}: $\mathfrak{r}$ is a left ideal.

\textbf{Show}: $\mathfrak{r}$ is an ideal.

\begin{exam}
If $\Lambda$ is a division ring, then $\mathfrak{r} = (0)$. 
\end{exam} 

\begin{exam}
$\Lambda = \mathbb{Z}$, $\langle p\rangle$ -- maximal ideal if $p$ is
a prime.

\[\mathfrak{r} = \cap_{p \text{\ prime}} \langle p\rangle = \langle
n\rangle = (0)\]

\[\langle n\rangle \subseteq \langle p\rangle \Rightarrow p\mid n,
\forall p \text{\ prime} \Rightarrow n = 0.\]
\end{exam}

\begin{exam}
$\Lambda = \mathbb{Q}\times \mathbb{Q}$, $\mathfrak{m}_1 =
\mathbb{Q}\times (0)$, $\mathfrak{m}_2 = (0)\times\mathbb{Q}$ - both are
maximal ideals.

\[(0) = \mathfrak{m}_1 \cap \mathfrak{m}_2 \subseteq \mathfrak{r}
  \Rightarrow \mathfrak{r} = (0).\] 
\end{exam}
In general, if we can find a finite set of maximal ideals
$\{\mathfrak{m}_i\}_{i=1}^t$ such that $\cap_{i=1}^t \mathfrak{m}_i =
(0)$, then $\mathfrak{r} = (0)$. 

\begin{exer}
Show that $\Lambda$ semisimple $\Rightarrow$ $\rad\Lambda = (0)$.
\end{exer}

\begin{exam}
$\Gamma\colon \xymatrix{ & 1\ar[dl]^\alpha\ar[dr]^\beta & \\
2\ar[dr]^\gamma & & 3\ar[dl]^\delta \\
& 4 & 
}$, $\rho=\{\gamma\alpha - \delta\beta\}$, $k$ a field and $\Lambda =
k\Gamma/\langle \rho\rangle$.  What is $\rad\Lambda$?

\textbf{Know}: $1_\Lambda = \overline{e_1} + \overline{e_2} +
\overline{e_3} + \overline{e_4}$, where
$\overline{e_i}\cdot\overline{e_j} = \begin{cases} \overline{e_i},
  \text{\ if $i = j$}\\
0, i\neq j
\end{cases}$

\begin{exer}
\begin{align}
\Lambda & = \Lambda\overline{e_1} \oplus \Lambda\overline{e_2} \oplus \Lambda\overline{e_3} \oplus \Lambda\overline{e_4}\notag\\
\mathfrak{m}_1 & = \Lambda\{\alpha,\beta\} \oplus \Lambda\overline{e_2} \oplus \Lambda\overline{e_3} \oplus \Lambda\overline{e_4}\notag\\
\mathfrak{m}_2 & = \Lambda\overline{e_1} \oplus \Lambda\overline{\gamma} \oplus \Lambda\overline{e_3} \oplus \Lambda\overline{e_4}\notag\\
\mathfrak{m}_3 & = \Lambda\overline{e_1} \oplus \Lambda\overline{e_2}  \oplus \Lambda\overline{\delta} \oplus \Lambda\overline{e_4}\notag\\
\mathfrak{m}_4 & = \Lambda\overline{e_1} \oplus \Lambda\overline{e_2} \oplus \Lambda\overline{e_3} \oplus (0)\notag\\
\end{align}
\[\mathfrak{m}_1 \cap \mathfrak{m}_2 \cap \mathfrak{m}_3 \cap
  \mathfrak{m}_4 = \langle
  \overline{\alpha},\overline{\beta},\overline{\gamma},\overline{\delta}\rangle
  \xrightarrow{\text{?}} \rad\Lambda.\]
\end{exer}
\end{exam}

\begin{prop}\label{prop:radcharacterization}
For any ring $\Lambda$ and any $\lambda\in\Lambda$, the following are
equivalent.
\begin{enumerate}[\rm(i)]
\item $\lambda\in\rad\Lambda$, 
\item $1 - x\lambda$ is \emph{left invertible} for all $x\in \Lambda$
  (i.e.\ $\exists x'\in\Lambda$ such that $x'(1-x\lambda) = 1$), 
\item $\lambda S = (0)$ for any simple $\Lambda$-module $S$.
\end{enumerate}
\end{prop}
\begin{proof}
(i) $\Rightarrow$ (ii): Suppose $\exists x \in \Lambda$ such that $1 -
x\lambda$ is not left invertible with $\lambda\in\Lambda$.  Then
$\Lambda(1 - x\lambda)$ is a proper left ideal in $\Lambda$.  Any
proper left ideal is contained in a maximal left ideal.  If
$\lambda\in\mathfrak{m}$, then $1\in\mathfrak{m}$, contradiction. So
$\lambda\not\in\mathfrak{m}$ and in particular
$\lambda\not\in\rad\Lambda$. 

(ii) $\Rightarrow$ (iii): Suppose $\exists$ a simple $\Lambda$-module
$S$ such that $\lambda S \neq (0)$, i.e.\ $\exists 0\neq s \in S$ with
$\lambda s \neq 0$.  Have $(0)\neq \Lambda(\lambda s)\subseteq S$.  

$S$ simple $\Rightarrow$ $\Lambda \lambda s = S$.

Hence, $\exists x \in \Lambda$ such that $x\lambda s = s$
$\Rightarrow$ $(1 - x\lambda) s = 0$

If $1 - x\lambda$ is left invertible, then $s=0$ $\Rightarrow$
$1-x\lambda$ is not left invertible.

(iii) $\Rightarrow$ (i): Let $\mathfrak{m}$ be a maximal left ideal in
$\Lambda$.  Then $\Lambda/\mathfrak{m}$ is a simple left
$\Lambda$-module.  By assumption
\[\lambda\cdot \Lambda/\mathfrak{m} = (0),\]
in particular
\[\lambda(1 + \mathfrak{m}) = \lambda + \mathfrak{m} = \overline{0}\]
and $\lambda\in\mathfrak{m}$ for all maximal left ideals
$\mathfrak{m}$ in $\Lambda$.   Hence $\lambda\in\rad\Lambda$. 
\end{proof}

\begin{defin}
Let $M$ be a (left) $\Lambda$-module, and let 
\[\Ann_\Lambda(M) = \{\lambda\in\Lambda\mid \lambda m = 0, \forall
  m\in M\}.\]
The set $\Ann_\Lambda(M)$ is called the
\emph{annihilator}\index{module!annihilator}\index{annihilator} of
$M$. 
\end{defin}

\textbf{Note}: $\Ann_\Lambda(M)$ is a two-sided ideal in $\Lambda$. 

\begin{cor}
Given a ring $\Lambda$
\[\rad\Lambda = \cap_{S \text{\ simple}\\ \text{left
      $\Lambda$-module}} \Ann_\Lambda(S).\]
In particular, $\rad\Lambda$ is a two-sided ideal in $\Lambda$. 
\end{cor}
\begin{proof}
Follows from (i) $\Leftrightarrow$ (iii) in Proposition
\ref{prop:radcharacterization}. 
\end{proof}

Can we find $\rad\Lambda$ from this?

$S\simeq S' \Rightarrow \Ann_\Lambda(S) = \Ann_\Lambda(S')$. 

\begin{thm}[Nakayama Lemma]
Given a ring $\Lambda$ and a finitely generated $\Lambda$-module $M$.
If $\mathfrak{a}$ is an ideal in $\Lambda$ with $\mathfrak{a}\subseteq
\rad\Lambda$, then $\mathfrak{a} M = M$ implies that $M = (0)$. 
\end{thm}
\begin{proof}
Suppose that $M\neq (0)$ and $\mathfrak{a}M = M$.  Let
$\{m_1,m_2,\ldots,m_t\}$ be a minimal set of generators for $M$ as a
$\Lambda$.  Since $\mathfrak{a}M=M$, we have that 
\[m_1 = \sum_{i=1}^t \lambda_im_i\]
for $\lambda_i\in\mathfrak{a}\subseteq \rad\Lambda$. 
\[\Rightarrow (1 - \lambda_1)m_1 = \sum_{i=2}^t \lambda_im_i\]
Since $\lambda_1\in\mathfrak{a}\subseteq \rad\Lambda
\stackrel{\textrm{Proposition \ref{prop:radcharacterization}}}{\Rightarrow} 1 -
\lambda_1$ has a left inverse, say $u$.
\[\Rightarrow m_1 = u(1 - \lambda_1) m_1 = \sum_{i=2}^t
  u\lambda_im_i\]
\[\Rightarrow \text{\ $M$ can be generated by $\{m_2,\ldots,m_t\}$.  Contradiction!}\]
If $t=1$, then $M = (0)$.  Contradiction!  If $t > 1$, then we have a
contradiction to the choice of generating set
$\{m_1,m_2,\ldots,m_t\}$. $\Rightarrow \mathfrak{a} M \neq M$. 
\end{proof}

\begin{recall}
A left ideal $\mathfrak{a}\subseteq \Lambda$ is
\emph{nilpotent}\index{ideal!nilpotent}\index{nilpotent!ideal} if
$\exists n \geq 1$ such that $\mathfrak{a}^n = (0)$. 
\end{recall}

\begin{lem}
$\Lambda$ ring.
\begin{enumerate}[\rm(a)]
\item If $\Lambda$ is a left (right) artinian, then $\rad\Lambda$ is
  nilpotent. 
\item If $\mathfrak{a}\subseteq \Lambda$ is a nilpotent left ideal,
  then $\mathfrak{a} \subseteq \rad\Lambda$
\end{enumerate}
\end{lem}
\begin{proof}
(a) $\mathfrak{r} = \rad\Lambda$. 
\[\cdots \supseteq \mathfrak{r}^i \supseteq \mathfrak{r}^{i+1}
  \supseteq \cdots\]
is a descending chain of left ideals in $\Lambda$.

$\Lambda$ left artinian $\Rightarrow$ $\mathfrak{r}^m =
\mathfrak{r}^{m+1} = \cdots$ for some $m$
\[M=\mathfrak{r}^m = \mathfrak{r}^{m+1} = \mathfrak{r}\mathfrak{r}^m = \mathfrak{r}M\]
$\Lambda$ left artinian $\Rightarrow$ $\Lambda$ left noetherian.

$\mathfrak{r}^m = M \subseteq \Lambda$ left ideal $\Rightarrow$
$M=\mathfrak{r}^m$ finitely generated $\Lambda$-module.

Nakayama Lemma $\Rightarrow$ $\mathfrak{r}^m = (0)$ and $\rad\Lambda$
is nilpotent.

(b)  Assume that $\mathfrak{a}$ is a nilpotent left ideal in
$\Lambda$, say $\mathfrak{a}^n = (0)$ for some $n\geq 1$.  Let $a\in
\mathfrak{a}$. Then for all $x\in\Lambda$ we have $xa\in\mathfrak{a}$
and $(xa)^n=0$.

$\Rightarrow  (1 + (xa) + (xa)^2 + \cdots (xa)^{n-1} = 1 - (xa)^n = 1$

$\Rightarrow 1 -xa$ has a left inverse for all $x\in\Lambda$.

Proposition \ref{prop:radcharacterization} $\Rightarrow a\in\rad\Lambda \Rightarrow
\mathfrak{a}\subseteq \rad\Lambda$. 
\end{proof}

\begin{recall}
\begin{align}
\Lambda \textrm{\ semisimple}\index{ring!semisimple} & \Leftrightarrow {_\Lambda\Lambda}
                                \textrm{\ semisimple $\Lambda$-module}\notag\\
& \Leftrightarrow \Lambda\simeq M_{n_1}(D_1)\times
  M_{n_2}(D_2)\times\cdots \times M_{n_t}(D_t)\notag\\
& \hphantom{\Leftrightarrow}\qquad \qquad n_i\geq 1, D_i \textrm{\ division ring}, t
  < \infty\notag\\
& \Leftrightarrow \Lambda \textrm{\ left artinian and has no nilpotent
  (left) ideals}\notag
\end{align}
\end{recall}

\begin{thm}\label{thm:semisimplechar}
$\Lambda$ ring.

$\Lambda$ semisimple $\Leftrightarrow$ $\Lambda$ left artinian and
$\rad\Lambda = (0)$. 
\end{thm}
\begin{proof}
$\Mapsto$ 
\[\Lambda \textrm{\ semisimple} \Rightarrow \Lambda \textrm{\ left
  artinian}\Rightarrow \rad\Lambda \textrm{\ is nilpotent}\]
Using this we obtain:
\[\Lambda \textrm{\ semisimple} \Rightarrow \Lambda \textrm{\ has no non-zero nilpotent left ideals} 
\Rightarrow \rad\Lambda = (0)\]

$\Mapsfrom$ Assume that $\Lambda$ is left artinian with $\rad\Lambda =
(0)$

Lemma \ref{lem:} $\Rightarrow$ $\Lambda$ has no non-zero nilpotent
left ideals.

$\Rightarrow$ $\Lambda$ is semisimple.
\end{proof}

\begin{thm}
$\Lambda$ left artinian, $\mathfrak{r} = \rad\Lambda$.
Then
\begin{enumerate}[\rm(a)]
\item $\Lambda/\mathfrak{r}$ is a semisimple ring. 
\item A left $\Lambda$-module $M$ is semisimple if and only if
  $\mathfrak{r}M=(0)$. 
\item There are only finitely many non-isomorphic simple
  $\Lambda$-modules, and they all occur as direct summands of
  $\Lambda/\mathfrak{r}$. 
\item $\Lambda$ is left noetherian.
\end{enumerate}
\end{thm}
\begin{proof}
(a) $\Lambda$ left artinian $\Rightarrow$ $\Lambda/\mathfrak{r}$ left
artinian.

$\rad(\Lambda/\mathfrak{r}) = (\rad\Lambda)/\mathfrak{r}$.

Theorem \ref{thm:semisimplechar} $\Rightarrow$ $\Lambda/\mathfrak{r}$
is semisimple. 

(b)--(d):  Exercise, see the book Proposition 3.1 page 9. 
\end{proof}

\begin{recall}
$\Lambda$ left artinian $\Leftrightarrow$ $l(_\Lambda\Lambda) <
\infty$. 
\end{recall}

\begin{cor}
$\Lambda$ ring. TFAE:
\begin{enumerate}[\rm(a)]
\item $\Lambda$ left artinian.
\item Every finitely generated $\Lambda$-module has finite length.
\item $\mathfrak{r} = \rad\Lambda$ is nilpotent and
  $\mathfrak{r}^i/\mathfrak{r}^{i + 1}$ is finitely generated
  semisimple $\Lambda$-module for all $i\geq 0$. 
\end{enumerate}
\end{cor}
\begin{proof}
(b) $\Rightarrow$ (a): In particular, $\Lambda$ as a left
$\Lambda$-module has finite length. 

$\Rightarrow$ $\Lambda$ is left artinian (and left noetherian). 

(a) $\Rightarrow$ (c):  $\mathfrak{r}$ is nilpotent by Lemma
\ref{lem:} (a), say $\mathfrak{r}^n = (0)$. 
\begin{center}
\begin{tabular}{rl}
Theorem \ref{thm:} (d) $\Rightarrow$ & $\Lambda$ left noetherian.\\
$\Rightarrow$ & $\mathfrak{r}^i$ finitely generated for all $i$ (as a
left ideal)\\
$\Rightarrow$ & $\mathfrak{r}^i/\mathfrak{r}^{i+1}$ finitely generated for all $i$ (as a
left ideal)\\
\end{tabular}
\end{center}
Theorem \ref{thm:} (b) $\Rightarrow$
$\mathfrak{r}^i/\mathfrak{r}^{i+1}$ semisimple $\Lambda$-module for
all $i\geq 0$, since $\mathfrak{r}\cdot \mathfrak{r}^i/\mathfrak{r}^{i+1}=(0)$

(c) $\Rightarrow$ (b):  Suppose $\mathfrak{r}^n = (0)$ for some $n\geq
1$.  Consider: $\Lambda \supseteq \mathfrak{r}\supseteq \mathfrak{r}^2
\supseteq \mathfrak{r}^3\supseteq \cdots \supseteq
\mathfrak{r}^{n-1}\supseteq \mathfrak{r}^n =(0)$.

\label{2017-pages}

In particular, we have exact sequences 


\end{proof}
