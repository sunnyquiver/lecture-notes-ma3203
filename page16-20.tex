\section{Quiver with relations}
Can all algebras over a field k be represented as $k\G$?

\textbf{No}, since $\Lambda = k[x]/\langle x^2\rangle \ncong k\G$ for
all quivers $\G$.

Why? $\dim_{k}\Lambda = 2$ and $\Lambda$ is not semisimple.

Assume that  $\Lambda \cong k\G$.  We have that $2 = \dim_kk\G \ge
\text{\# vertices in\ } \G$.   If $\G \colon \xymatrix{1& 2}$,  then
$k\G$ is semisimple. This is a  contradiction. Hence $\G$ has one
vertex and contains the quiver  $\xymatrix{1\ar@(ur,dr)^\alpha}$.
This implies $\dim_kk\G = \infty$, another contradiction.  But,
$\G\colon \xymatrix{1\ar@(ur,dr)^\alpha}$, $\Lambda \simeq
\frac{k\G}{\langle\alpha^2\rangle} $.

Let $\G=(\G_0,\G_1)$ be a quiver, k field.\\
\begin{defin}
\begin{enumerate}[(a)]
	~\\ \item A \emph{relation} $\sigma$ in the quiver $\G$ over k is a k-linear combination of paths
	\begin{center}
		$\sigma = a_1p_1 + a_2p_2+ \cdots + a_tp_t$\\
	\end{center}
	where $a_i\in k$, $e(p_i)=e(p_1)$ and $s(p_i)=s(p_1)$ for all $i$, and $l(p_i)\geq 2$ (the length of the path $p_i$ )
	\item if $\varrho = \{\sigma\}_{l \in T}$ is a set of relations in $\G$ over k, then $(\G,\varrho)$ is a \emph{quiver with relations over $k$.}  
\end{enumerate}
\end{defin}
\begin{exam}
	$\xymatrix{ & 1\ar[dl]_\alpha\ar[dr]^\beta & \\
		\G \colon 2\ar[dr]_\gamma & & 3\ar[dl]^\delta\\
		& 4 & }$\newline $k$ field, $\sigma = \gamma\alpha - \delta\beta$.\\\newline
		$ \G=\frac{k\G}{\langle\sigma\rangle}$. Let $M$ be a left $k\Lambda$-module. Any left $\Lambda$-module is a left $k\Lambda$-module, since $k\G \xrightarrow{\pi} \frac{k\G}{\large\sigma\rangle} = \Lambda$.\\
		$\implies M$ gives rise to a representation of $\G$.\\
		
	$\xymatrix{ & e_1M\ar[dl]_{f_\alpha = \alpha\cdot-}\ar[dr]^{\beta\cdot-=f_\beta} & \\
	 e_2M\ar[dr]_{f_\gamma=\gamma\cdot-} & & e_3M\ar[dl]^{\delta\cdot-=f_\delta}\\
	& 4 & }$\newline

$\sigma\in k\G, m \in M.$\\ $ \sigma\cdot m\defeq\pi(\sigma)\cdot m = 0\cdot m = 0, \forall m \in M$\\
$m=e_1m+e_2m+e_3m+e_4m$, $sigma = e_4\sigma e_1$\\
$0 = \sigma\cdot m = (\gamma\alpha-\delta\beta)e_1m = \gamma(\alpha e_1m)-\delta(\beta e_1m) = f_\gamma f_\alpha(e_1m)-f_\sigma f_\beta(e_1m) = \underbrace{f_\gamma f_\alpha-f_\sigma f_\beta}_{f_\sigma}(e_im) \implies f_\sigma(e_1M)=0 \implies f_\sigma = 0 $.\\
Hence $\Lambda$-module $M$ corresponds to a representation of $\G$ satisfying the relation $\sigma(f_\sigma=0)$.\\
Conversely, we claim that a representation $(V,f)$ of $\G$ 	such that
\begin{center}
	$f_\sigma=f_\gamma f_\alpha-f_\sigma f_\beta=0$
\end{center} 
 gives a module over $\Lambda$.\\\newline
\textbf{Recall}: $I \subseteq R$ ideal: $\frac{R}{I}$-module $M$ is the same as an $R$-module $M$ such that $I\cdot M = (0)$.\\
\begin{center}
	$M = V(1) \oplus V(2) \oplus V(3) \oplus V(4) \leftarrow k\G$-module\\
	$e_1\cdot(v_1,v_2,v_3,v_4)=(v_1,0,0,0)$\\
	$\alpha\cdot(v_1,v_2,v_3,v_4)=(0,f_\alpha(v_1),0,0)$\\
\end{center}
~\\$\sigma\cdot(v_1,v_2,v_3,v_4) = (\gamma\alpha-\delta\beta)\cdot(v_1,v_2,v_3,v_4) = (0,0,0,f_\gamma f_\alpha(v_1)-f_\sigma f_\beta(v_1)) =  (0,0,0,(f_\gamma f_\alpha-f_\sigma f_\beta)(v_1)) = (0,0,0,0) $\\\newline
$\implies M$ is a $\Lambda$-module $(\Lambda=\frac{k\G}{\langle\sigma\rangle})$.
\end{exam}

\begin{exam}
$\G\colon \xymatrix{1 \ar@(ur,dr)^\alpha}, P=\{\alpha^2 \}$, $k$ field. $ \Lambda=\frac{k\G}{\langle\alpha^2\rangle}$. Find all induced $\Lambda$-modules.\\
$M$ left $\Lambda$-module $\leadsto 
(V,f) $ representation of $\G$ satisfying the relation $\alpha^2, $i.e.  $ f_{\alpha^2}=(f_\alpha)^2$\\  
	
	$\xymatrix{V \ar@(ur,dr)^{f_\alpha}}, (f_\alpha)^2=0$\\\newline
	$\implies$ The minimal polynomial of $f_\alpha$ is $x$ or $x^2$\\
	$\implies$ The ivariant factor of $f_\alpha$ is $x$ or $x^2$\\
	$\implies$ The matrix of $f_\alpha$ is similar to a direct sum of companion matrices of $x$ or $x^2$, $M_{(x)}=0$ and $M_{(x^2)}= \begin{pmatrix}0&0\\1&0\end{pmatrix}$\\~\\
	
Let T be the matrix of $f_\alpha$ w.r.t some basis $\beta$.\\
Then $\exists$ an invertible matrix $P$ such that\newline
	
$ T = P \underbrace{ \left(\begin{smallmatrix} r\begin{cases} \left(\begin{smallmatrix} 0 \cdots 0  \\\vdots \ddots \vdots \\0 \cdots 0 \end{smallmatrix}\right) \end{cases} & \overbrace{\text{\huge0}}^{2s}\\\text{\huge0} & 
\left(\begin{smallmatrix}  
\left( \begin{smallmatrix} 0 & 0\\ 1 & 0\\ \end{smallmatrix}\right) & & & \text{\huge0}  \\
& \begin{smallmatrix} 0 & 0\\ 1 & 0\\ \end{smallmatrix} & & \\
& &\ddots &  & \\
\text{\huge0} & & &\begin{smallmatrix} 0 & 0\\ 1 & 0\\ \end{smallmatrix} \\
\end{smallmatrix}\right) \end{smallmatrix}\right)}_{T_0}P^{-1} $ $\implies TP= PT_0$ \nolinebreak[4]\\\newline
$\xymatrix{V \ar@(ur,dr)^{T}} \iff \xymatrix{V \ar@(ur,dr)^{T_0}} \simeq (\xymatrix{k \ar@(ur,dr)^{0}})^r \oplus (\xymatrix{k^2 \ar@(ur,dr)^{\left(\begin{smallmatrix}0&0\\1&0\end{smallmatrix}\right)}})^s  $  isomorphisme of representation

\textbf{Show}: $\xymatrix{k \ar@(ur,dr)^{0}} \iff \frac{k\G}{\langle\alpha\rangle}$ and $\xymatrix{k^2 \ar@(ur,dr)^{\left(\begin{smallmatrix}0&0\\1&0\end{smallmatrix}\right)}} \iff \frac{k\G}{\langle\alpha^2\rangle} \implies \Lambda$ is of finite representation type.
\end{exam}
\section{Finite length}
$\Lambda$ ring, $A$ a (left) $\Lambda$-module.
\begin{defin}
	A has \emph{finite length}\index{finite length} if there exists a finite filtration.\\
	\begin{center}
			$\mathscr{F}\colon A = A_0 \supseteq A_1 \supseteq A_2 \supseteq \cdots A_{n-1} \supseteq A_{n} \supseteq A_{n+1} = (0) $\\
	\end{center}
of submodules of $A$ such that $\frac{A_i}{A_{i+1}} = 0$ or simple for i$=0,1,\cdots,n.$. 	$\mathscr{F}$ is a \emph{generalized composition series}\index{generalized composition series} of $A$, and if $\frac{A_i}{A_{i+1}} \ne 0$ for all i, then $\mathscr{F}$ is a \emph{composition series}\index{composition series}. If $S=\frac{A_i}{A_{i+1}} \ne 0$, then $S$ is called a \emph{composition factor of $A$}\index{composition factor}\\

Let $S$ be a simple $\Lambda$-module. Let
\begin{align}
m_s^{\mathscr{F}}(A) & \defeq \{i \mid \frac{A_i}{A_{i+1}}\simeq S \}\mid \text{ , } \notag\\
l_{\mathscr{F}} & \defeq \sum_{\mathclap{\substack{\small[S]
      \text{isomorphism} \\ \small\text{classes of
        simples}}}}m_s^{\mathscr{F}}(A)\notag\\
\intertext{ and }\notag\\
l(A) & \defeq \min_{\mathclap{\substack{\mathscr{F} \text{generalized} \\ \text{composition series}}}} l_{\mathscr{F}}(A)\notag
\end{align}
\end{defin}

\begin{exam}
	\begin{enumerate}[(1)]
		\item $\Lambda$ ring, $S$ simple $\Lambda$-module.\\
		Composition series: $S \supseteq (0)$\\
		composition factors: $\{S\}$\\
		$\implies m_T(S) = \left\{\!\begin{aligned}
			&1 &\text{if } T \simeq S\\[1ex]
			&0 &\text{otherwise}\\[1ex]
		\end{aligned}\right\}\implies l(S)=1$
		\item $\Lambda = k[x], f(x)$ irredicible\\
		$S_f=\frac{k[x]}{\langle f(x) \rangle}$ - simple $\Lambda$- module.\\
		$\implies l(S_f)=1$, while $\dim_kS_f = \deg f(x)$ \\
		\item $\G\colon  \xymatrix{1\ar[r]^\alpha & 2\ar[r]^\beta & }, k \text{ field } , \Lambda=k\G$\\
		$
		\xymatrix
		{
			& k\ar[d]^1 & 0 \ar[l]^0\ar[d]^0  & 0\ar[l]^0\ar[d]^0 & 0\ar[l]^0\ar[d]^0 \\
		\mathscr{F}\colon	&k \ar[d]^1  & k\ar[l]^1\ar[d]^1 & 0\ar[l]^0\ar[d]^0 & 0\ar[l]^0\ar[d]^0 \\
			&k \ar@{}[d]|-*[@]{=}& k\ar@{}[d]|-*[@]{=} & k\ar@{}[d]|-*[@]{=} & 0\ar@{}[d]|-*[@]{=} \\\
			&V_o\ar@{<~>}[d] & V_1\ar@{<~>}[d] & V_2\ar@{<~>}[d] & V_3\ar@{<~>}[d]\\
			M \ar@{}[r]|-*[@]{=} & M_0 \ar@{}[r]|-*[@]{\supseteq}  &  M_1 \ar@{}[r]|-*[@]{\supseteq} & M_2 \ar@{}[r]|-*[@]{\supseteq} & M_3 = (0) \\
		}$\\~\\
	
	$\xymatrix
	{
		& k\ar[d] & \\
		\frac{V_0}{V_1} \ar@{}[r]|-*[@]{\simeq}	& 0\ar[d]\ar@{<~>}[r] & S_1 , \\
		& 0 & \\
	}$
$\xymatrix
{
	& k\ar[d] & \\
	\frac{V_1}{V_2} \ar@{}[r]|-*[@]{\simeq}	& 0\ar[d]\ar@{<~>}[r] & S_2 , \\
	& 0 & \\
}$
$\xymatrix
{
	& k\ar[d] & \\
	\frac{V_2}{V_3} \ar@{}[r]|-*[@]{\simeq}	& 0\ar[d]\ar@{<~>}[r] & S_3 \\
	& 0 & \\
}$\\
$\implies l_{\mathscr{F}}(M)=3$\\~\\

\item $\G \xymatrix{ & 1\ar[dr]^{\beta} \ar[dl]_{\alpha} & \\ 2 & & 3
  },$ $k$ field, $M = k\G e_1 \leftrightsquigarrow \xymatrix{ & k\ar[dr]^{1} \ar[dl]_{1} & \\ k & & k }\\$

\[\xymatrix@C=2pt@R=10pt{
& k\ar[dr]^{1} \ar[dl]_{1} & & & & 0\ar[dr]^{0} \ar[dl]_{0} & & & & 0\ar[dr]^{0} \ar[dl]_{0} & & & & 0\ar[dr]^{0} \ar[dl]_{0} &\\
k & & k & \supseteq & k & & k & \supseteq & k & & 0 & \supseteq & 0 & & 0\\
& & & & & \ar@{}[u]|-*[@]{\subseteq} & & & & & & & & & \\
& & & & & 0\ar[dr]^{0} \ar[dl]_{0} & & & & 0\ar[dr]^{0} \ar[dl]_{0} & & & & & \\
& & & & 0 & & k & \supseteq & 0 & & 0 &  &  & & 
}\]
 Hence, we have two different composition series for the module $M$!
Computing the factors of neigbouring modules in the compositions
series, gives the same simple modules, but in a permuted order. 
\end{enumerate}
\end{exam}

\begin{note}
	\begin{enumerate}[(1)]
		\item Composition serice are not unique!
		\item $l_{\mathscr{F}}(M) = l_{\mathscr{G}}(M)$
		\item The set of composition factors is the same for $\mathscr{F}$ and $\mathscr{G}$
	\end{enumerate}
\end{note}

The proof of Jordan-Hølder theorem goes by induction on length and using short exact sequences.
\begin{defin}
	$\xymatrix{0 \ar[r] & A\ar[r]^f & B\ar[r]^g & C \ar[r] & 0}$
        is a \emph{(short) exact sequence}\index{exact sequence} of
        (left) $\Lambda$-module if 
	\begin{enumerate}[(i)]
		\item $f$ is injective (1-1),
		\item $g$ is surjective (onto),
		\item Im(f) = ker(g).
	\end{enumerate}
\end{defin} 
\begin{note}
	\begin{enumerate}[(1)]
		\item $A \subseteq B$ two  $\Lambda$-modules.  Then\\
		\begin{center}
			$\xymatrix{0 \ar[r] & A\ar@{^(->}[r] & B\ar[r] & B/A \ar[r] & 0}$
		\end{center}
	is an exact sequence.
	\item If $\xymatrix{0 \ar[r] & A\ar[r]^f & B\ar[r]^g & C \ar[r] & 0}$ is an exact sequence, then
	\begin{enumerate}[(a)]
        \item \begin{align}
                C & = \text{Im}(g)\notag\\
                  & \simeq \frac{B}{\text{Ker(g)}}\notag\\
                  & \simeq \frac{B}{\text{Im(f)}}\notag\\
                \text{Im}(f) & \simeq A\notag.
                   \end{align}
		\item $B = (0) \implies A = (0) \text{ and } C=(0). $ 
	\end{enumerate}
	\end{enumerate}
\end{note}

\begin{exam}
	\begin{enumerate}[(1)]
		\item  $\xymatrix{0 \ar[r] & \mathbb{Z}\ar[r]^{-\cdot n} & \mathbb{Z}\ar[r] & \frac{\mathbb{Z}}{n\mathbb{Z}} \ar[r] & 0}$ exact
		\item $M, N$ $\Lambda$-modules.
		$\xymatrix
		{0\ar[r] & M\ar[r]^{\left(\begin{smallmatrix} 1 \\ 0 \end{smallmatrix}\right)} & M\oplus N\ar[r]^{\left(\begin{smallmatrix} 1 & 0 \end{smallmatrix}\right)} & N\ar[r] & 0 \\
			  & m \ar@{|->}[r]& (m,0) \\
			  &   & (m,n) \ar@{|->}[r]    & n\\
		}$ exact
	\item $\Lambda = \frac{k\G}{\langle p \rangle}$, $\xymatrix{0 \ar[r] & A\ar[r]^f & B\ar[r]^g & C \ar[r] & 0}$ exact sequence of $\Lambda$-modules.
	$\xymatrix
	{
		 & V_A(i)\ar@{}[d]|-*[@]{=} \ar[r]^{f\vert_{V_A(i)}} & V_B(i)\ar@{}[d]|-*[@]{=} \ar[r]& V_C(i)\ar@{}[d]|-*[@]{=} & \\
		0\ar[r]& e_iA \ar[r]^{f\vert_{e_iA}}  & e_iB  \ar[r]^{g\vert_{e_iB}}  & e_iC \ar[r]  & 0\\
	}$ exact sequence for all i\\
Hence, 	$\xymatrix{0 \ar[r] & (V',f')\ar[r]^g & (V,f)\ar[r]^h & (V'',f'') \ar[r] & 0}$ is an exact sequence of representation if $\xymatrix{0 \ar[r] & V'(i)\ar[r]^{g(i)} & V(i)\ar[r]^{h(i)} & V''(i) \ar[r] & 0}$ is exact for all i $\in\G_0$.
	
	\end{enumerate}
\end{exam}

\begin{exer}
		$f\colon\xymatrix{A\ar[r] & B}$ and $g\colon\xymatrix{B\ar[r] & C}$, $\Lambda$-homomorphisme, $B' \subseteq B$ submodule.
		\begin{enumerate}[(1)]
			\item $f^{-1}(B')= \{ a\in A\mid f(a)\in B'\}\subseteq A$ submodule.
			\item $g(B')= \{ g(b')\mid b'\in B'\}\subseteq C$ submodule.
		\end{enumerate}
\end{exer}

Let $\xymatrix{0 \ar[r] & A\ar[r]^f & B\ar[r]^g & C \ar[r] & 0}$ be an exact sequence and let $\mathscr{F}$ be a generalized composition series of B.
\[\xymatrix
{0\ar[r] & A\ar[r]^f & B\ar[r]^g\ar & C\ar[r] & 0\\
			& A_o=f^{-1}(B_0)\ar@{}[u]|-*[@]{=}	  & B_0	\ar@{}[u]|-*[@]{=}				& g(B_0)= C_0\ar@{}[u]|-*[@]{=} \\
			& A_1=f^{-1}(B_1)\ar@{}[u]|-*[@]{\subseteq} & B_1\ar@{}[u]|-*[@]{\subseteq}& g(B_1)= C_1\ar@{}[u]|-*[@]{\subseteq} \\
			& A_2=f^{-1}(B_2)\ar@{}[u]|-*[@]{\subseteq} & B_2\ar@{}[u]|-*[@]{\subseteq}& g(B_2)= C_2\ar@{}[u]|-*[@]{\subseteq} \\
			& \vdots\ar@{}[u]|-*[@]{\subseteq} & \vdots\ar@{}[u]|-*[@]{\subseteq}& \vdots\ar@{}[u]|-*[@]{\subseteq} \\
			& A_n=f^{-1}(B_n)=(0)\ar@{}[u]|-*[@]{\subseteq} & B_n=(0)\ar@{}[u]|-*[@]{\subseteq}& g(B_n)= C_n=(0)\ar@{}[u]|-*[@]{\subseteq} \\
			& \mathscr{F}'\ar@{}[u]|-*[@]{\colon} & \mathscr{F}\ar@{}[u]|-*[@]{\colon}& \mathscr{F}''\ar@{}[u]|-*[@]{\colon} \\
}\]
We have $A_n=(0)$, since $f$ is $1$-$1$.
\addtocounter{thm}{1}
\begin{prop}\label{prop:7}
\begin{enumerate}[(a)]
\item $\mathscr{F}'$ is generalized composition series
  of $A$, and $\mathscr{F}''$ is generalized composition series of $C$.
\item $m^{\mathscr{F}}_S(B)=m^{\mathscr{F}'}_S(A) +
  m^{\mathscr{F}''}_S(C)$ for all simple module $S$. 
\end{enumerate}
\end{prop}
\begin{proof}
  (a) (I) \textbf{Claim}:
  \[(*)\colon 0\to A_i\extto{f|_{A_i}} B_i \extto{g|_{B_i}} C_i\to 0\]
  is exact.

  By definition we have that $f(A_i)\subseteq B_i$ and $g(B_i) = C_i$
  for all $i$.  Furthermore
  \begin{itemize}
    \item $f|_{A_i}\colon A_i\to B_i$ is $1$-$1$, since $f$ is
      $1$-$1$.
    \item $g|_{B_i}\colon B_i\to C_i$ is onto by the definition of
      $C_i$.
    \item Since $f(A_i) \subseteq B_i$ and $gf = 0$, then
      $g|_{B_i}f|_{A_i} = 0$. This implies that $\Im f|_{A_i}
      \subseteq \Ker g|_{B_i}$.
    \end{itemize}
Let $b\in \Ker g|_{B_i}$.  This implies that $b\in\Ker g \subseteq \Im
f$.  So there exists $a\in A$ such that  $f(a) = b$, that is, $a\in
f^{-1}(B_i) = A_i$.  This is turn implies that
\[\Ker g|_{B_i} \subseteq \Im f|_{A_i} \Rightarrow \Ker g|_{B_i} =
  \Im f|_{A_i}.\]
This proves that $(*)$ is exact.

(II) \textbf{Claim}: The following diagram is exact and commutative:
\[\xymatrix{
           & 0\ar[d]              & 0\ar[d]              & 0\ar[d]             & \\
0\ar[r] & A_{i+1} \ar[r]^f\ar@{^(->}[d] & B_{i+1} \ar[r]^g\ar@{^(->}[d] & C_{i+1} \ar[r]\ar@{^(->}[d] &
0\\
0\ar[r] & A_{i} \ar[r]^f\ar[d]^{p'_i} & B_{i} \ar[r]^g\ar[d]^{p_i} & C_{i} \ar[r]\ar[d]^{p''_i} &
0\\
0\ar[r] & A_{i}/A_{i+1} \ar[r]^{\overline{f}}\ar[d] & B_{i}/B_{i+1} \ar[r]^{\overline{g}}\ar[d] & C_{i}/C_{i+1} \ar[r]\ar[d] & 0\\
           & 0              & 0              & 0             & 
}\]
where $p'_i$, $p_i$ and $p''_i$ are the natural projections,
$\overline{f}(a_i + A_{i+1}) = f(a_i) + B_{i+1}$ and $\overline{g}(b_i
+ B_{i+1}) = g(b_i) + C_{i+1}$.  Denote by $\eta_i$ the lower row in
the above diagram.  Easy to see that the diagram is
commutative, given that the everything is well-defined.

(i) \textbf{$\overline{f}$ well-defined}: Assume that $a_i + A_{i+1} =
a'_i + A_{i+1}$, that is, $a_i - a'_i\in A_{i+1}$. Then
$f(a_i-a'_i) = f(a_i) - f(a'_i)\in B_{i+1}$.  This means that
\[\overline{f}(a_i + A_{i+1}) = f(a_i) + A_{i+1} = f(a'_i) + A_{i+1} =
  \overline{f}(a'_i + A_{i+1}).\]
This shows that $\overline{f}$ is well-defined.

(ii) \textbf{$\overline{g}$ well-defined}: Similar.

(iii) \textbf{$\eta_i$ is exact}: (1) $\overline{f}$ is $1$-$1$:
Assume that $\overline{f}(a_i+A_{i+1}) = f(a_i) + A_{i+1} =0$ for
$a_i\in A_i$. This means that $f(a_i)\in B_{i+1}$.  Then by definition
$a_i\in A_{i+1}$ and consequently $a_i+A_{i+1} = 0$ and $\overline{f}$
is $1$-$1$.

(2) \textbf{$\overline{g}$ is onto}: Since $g$ and $p''_i$ are onto,
the composition $p''_ig = \overline{g}p_i$ is onto.  It follows from
this that $\overline{g}$ is onto.

(3) $\Im \overline{f} = \Ker \overline{g}$: Since $gf=0$, we have that
\[\overline{g}\overline{f}(a_i+A_{i+1}) - \overline{g}(f(a_i)+B_{i+1})
    = gf(a_i) + C_{i+1}.\]
  This implies that $\Im \overline{f} \subseteq \Ker \overline{g}$.

  Let $b_i +B_{i+1} \in \Ker \overline{g}$ for $b_i\in B_i$, that is,
  \[0 = \overline{g}(b_i + B_{i+1}) = g(b_i) + C_{i+1},\]
  hence $g(b_i)\in C_{i+1}$.  Choose $b_{i+1}\in B_{i+1}$ such that
  $g(b_{i+1}) = g(b_i)$.  Then $b_i - b_{i+1} \in B_i$, since $b_i$
  and $b_{i+1}$ are in $B_i$.  We infer that $g(b_i - b_{i+1}) = 0$,
  that is, $b_i - b_{i+1}\in \Ker g = \Im f$.  Choose $a_i\in A_i$
  such that $f(a_i) = b_i - b_{i+1}\in B_i$.  It follows from this
  that
  \[\overline{f}(a_i + A_{i+1}) = f(a_i) + B_{i+1} = b_i - b_{i+1} +
    B_{i+1} = b_i + B_{i+1},\]
  since $b_{i+1}\in B_{i+1}$.  This proves that $\Ker \overline{g}
  \subseteq \Im \overline{f}$. Combining the above we obtain that $\Im
  \overline{f} = \Ker \overline{g}$ and that $\eta_i$ is exact.

  We have that
  \[B_i/B_{i+1} = (0) \Rightarrow A_i/A_{i+1} = C_i/C_{i+1} = (0).\]
  If $B_i/B_{i+1}\simeq S$ is simple, then $A_i/A_{i+1} = (0)$ or $A_i/A_{i+1}
  \simeq S$.  If $A_i/A_{i+1} = (0)$, then $(0) = \Im \overline{f} =
  \Ker \overline{g}$, which means that $S\simeq B_i/B_{i+1} \simeq
  C_i/C_{i+1}$.  If $A_i/A_{i+1} \simeq S$, then $B_i/B_{i+1} = \Im \overline{f} =
  \Ker \overline{g}$.  This implies that $\overline{g} = (0)$ and that
  $C_i/C_{i+1} = \Im\overline{g} = (0)$.  Summing up, we get that
  either is
  \[A_i/A_{i+1} = (0) \textrm{\ and\ } C_i/C_{i+1} \simeq S\]
  or
  \[A_i/A_{i+1} \simeq S \textrm{\ and\ } C_i/C_{i+1} = (0).\]
This proves that $\mathscr{F}'$ is a generalized composition series of
$A$ and that $\mathscr{F}''$ is a generalized composition series of
$C$.

(b) Immediate consequence of the proof of (a).
\end{proof}

\begin{cor}\label{cor:8}
Given $A$, $B$ and $C$ three $\L$-modules with $B$ of finite length
and an exact sequence
\[0\to A\to B\to C\to 0.\]
Then
\[l(A) + l(C) \leq l(B).\]
In particular, $A$ and $C$ have finite length. 
\end{cor}
\begin{proof}
  Let $\mathscr{F}$ be a generalized composition series of $B$ such
  that $l(B) = l_{\mathscr{F}}(B)$.  Keeping the notation from before,
  we get
  \[l_{\mathscr{F}'}(A) + l_{\mathscr{F}''}(C) = l_{\mathscr{F}}(B) =
    l(B).\]
  Since $l_{\mathscr{F}'} \geq l(A)$ and $l_{\mathscr{F}''}(C) \geq
  l(C)$, we obtain that 
\[l(A) + l(C) \leq l(B).\]
\end{proof}

\begin{thm}\label{thm:9}
Let $B$ be a $\L$-module of finite length with $\mathscr{F}$ and
$\mathscr{G}$ two generalized composition series of $B$. Then
\begin{enumerate}[\rm(a)]
\item $m^{\mathscr{F}}_S(B) = m^{\mathscr{G}}_S(B) \defeq m_S(B)$.
\item $l_{\mathscr{F}}(B) = l_{\mathscr{G}}(B) \defeq l(B)$. 
\end{enumerate}
\end{thm}
\begin{proof}
(a) We use induction on $l(B)$. 

(1) $l(B) =1$: Then $B=S$ is a simple module and $m^{\mathscr{F}}_S(B)
= 1$ for all $\mathscr{F}$ and $m^{\mathscr{F}}_{S'}(B) = 0$ for all
simple modules $S'\not\simeq S$.  Hence, (a) holds.

(2) Assume that $n = l(B) > 1$, and assume that (a) is shown for $C$
with $l(C) < n$.  Since $B$ is not simple, choose $A\subseteq B$ with
$(0)\subsetneq A \subsetneq B$.  Have an exact sequence
\[0\to A\to B\to B/A\to 0.\]
Then $A$ and $B/A$ have finite length by Corollary \ref{cor:8} and
$l(A) + l(B/A) \leq l(B)$.  Since $A\neq (0)$ and $B/A\neq (0)$, we
have $l(A) > 0$ and $l(B/A) > 0$. Let $\mathscr{F}'$ and
$\mathscr{G}'$, and $\mathscr{F}''$ and $\mathscr{G}''$ be the induced
generalized composition series for $A$ and $B/A$, respectively of
$\mathscr{F}$ and $\mathscr{G}$.  Then by Proposition \ref{prop:7} we
have
\begin{align}
m^{\mathscr{F}}_S(B) & = m^{\mathscr{F}'}_S(A) + m^{\mathscr{F}''}_S(B/A) \notag\\
\intertext{and}\notag\\
m^{\mathscr{G}}_S(B) & = m^{\mathscr{G}'}_S(A) +
                       m^{\mathscr{G}''}_S(B/A) \notag
\end{align}
By induction we have $m^{\mathscr{F}'}_S(A) = m^{\mathscr{G}'}_S(A)$
and $m^{\mathscr{F}''}_S(B/A) = m^{\mathscr{G}''}_S(B/A)$, since $l(A)
< l(B)$ and $l(B/A) < l(B)$.  It follows that
\[m^{\mathscr{F}}_S(B) = m^{\mathscr{G}}_S(B)\]
for all simple $\L$-modules $S$.  

(b) Follows directly from (a), and $l_{\mathscr{F}}(B) =
l_{\mathscr{G}}(B) \defeq l(B)$. 
\end{proof}

\begin{note}
(1) It follows that if $B$ has finite length, then the set of
composition factors is uniquely determined up to isomorphism and
multiplicity.

(2) If $B\simeq C$, then $l(B) = l(C)$. 
\end{note}

\begin{prop}\label{prop:10}
If $0\to A\to B\to C\to 0$ is exact and $l(B) < \infty$, then 
\[l(B) = l(A) + l(C).\]
\end{prop}
\begin{proof}
Let $\mathscr{F}$ be a generalized composition series of $B$. Then
\begin{align}
l(B) & = l_{\mathscr{F}}(B) \defeq \sum_{[S]}
  m^{\mathscr{F}}_S(B)\notag\\
& = \sum_{[S]} (m^{\mathscr{F}'}_S(A) +
  m^{\mathscr{F}''}_S(C))\notag\\
& = l_{\mathscr{F}'}(A) + l_{\mathscr{F}''}(C)\notag
& = l(A) + l(C)\notag
\end{align}
since $l_{\mathscr{F}'}(A) = l(A)$ and $l_{\mathscr{F}''}(C) = l(C)$. 
\end{proof}

\begin{exam}
(1) Let $\Gamma = \xymatrix{1\ar@(ur,dr)^\alpha }$, and let $\Lambda =
k\Gamma/\langle \alpha^2\rangle$ for some field $k$. 

Left $\Lambda$-module $M = \Lambda$ correspond to the representation
$\xymatrix{k^2\ar@(ur,dr)^{\begin{smallmatrix} 0 & 0\\ 1 &
      0\end{smallmatrix}}}$. 

\[M = M_0 = \xymatrix{k^2\ar@(ur,dr)^{\begin{smallmatrix} 0 & 0\\ 1 &
        0\end{smallmatrix}} & & \ar@{_(->}[l] k\ar@(ur,dr)^{0} & &
    \ar@{_(->}[l] 0\ar@(ur,dr)^{0}\\
& &  M_1\ar@{=}[u] & &  M_2\ar@{=}[u] & &}\] 
Then $M_0/M_1 \simeq M_1/M_2 \simeq \xymatrix{k\ar@(ur,dr)^{0}}$.
Then $M$ has composition factors $\{S_1,S_1\}$, that is, the simple
module $S_1$ has multiplicity $2$. 

(2) Let $\Gamma\colon \xymatrix{ & 1\ar[dl]_\alpha \ar[dr]^\beta & \\
  2\ar[dr]_\gamma & & 3\ar[dl]^\delta \\
& 4 & }$ with relations $\rho = \{ \gamma\alpha - \delta\beta\}$.  Let
$\Lambda = k\Gamma/\langle \rho\rangle$, and let $M = \Lambda e_1$.
Find $l(M)$. The left $\Lambda$-module $M$ correspond to the
representation 
\[\xymatrix@C=10pt{ & e_1M\ar[dl]_{\alpha\cdot-} \ar[dr]^{\beta\cdot-} & \\
  e_2M\ar[dr]_{\gamma\cdot -} & & e_3M\ar[dl]^{\delta\cdot - } \\
& e_4M & }  \quad \simeq \quad 
\xymatrix@C=10pt{ & k\overline{e_1} \ar[dl]_{\alpha\cdot-} \ar[dr]^{\beta\cdot-} & \\
  k\overline{\alpha}\ar[dr]_{\gamma\cdot -} & & k\overline{\beta}\ar[dl]^{\delta\cdot - } \\
& k\overline{\gamma\alpha} = k\overline{\delta\beta} & } 
\quad \simeq \xymatrix@C=10pt{ & k\ar[dl]_{1} \ar[dr]^{1} & \\
  k\ar[dr]_{1} & & k\ar[dl]^{1} \\
& k & } 
\]
A composition series of $M$ is given by 
\[\xymatrix@C=6pt{
& k\ar[dl]_{1} \ar[dr]^{1} &     &                  & & 0\ar[dl]_{0}
\ar[dr]^{0} &     &                  & & 0\ar[dl]_{0} \ar[dr]^{0} &
&                  && 0\ar[dl]_{0} \ar[dr]^{0} &   & &  &  0\ar[dl]_{0}
\ar[dr]^{0} &  \\
k\ar[dr]_{1} & & k\ar[dl]^{1} & \supseteq & k\ar[dr]_{1} & &
k\ar[dl]^{1} & \supseteq & 0\ar[dr]_{0} & & k\ar[dl]^{1} & \supseteq &
0\ar[dr]_{0} & & 0\ar[dl]^{0} & \supseteq & 0\ar[dr]_{0} & & 0\ar[dl]^{0} \\
& k &                                    &                 & & k &
&                 & & k &                                    &
& & k &  & & & 0 & 
}\]
where we label the modules as $M = M_0 \supseteq M_1 \supseteq M_2
\supseteq M_3 \supseteq M_4 = (0)$. Then the factor modules $M_0/M_1$,
$M_1/M_2$, $M_2/M_3$ and $M_3/M_4$ correspond to the representations 
\[
\xymatrix@C=10pt{ & k\ar[dl]_{0} \ar[dr]^{0} & \\
  0\ar[dr]_{0} & & 0\ar[dl]^{0} \\
& 0 & }, \quad 
\xymatrix@C=10pt{ & 0\ar[dl]_{0} \ar[dr]^{0} & \\
  k\ar[dr]_{0} & & 0\ar[dl]^{0} \\
& 0 & }, \quad 
\xymatrix@C=10pt{ & 0\ar[dl]_{0} \ar[dr]^{0} & \\
  0\ar[dr]_{0} & & k\ar[dl]^{0} \\
& 0 & }, \quad 
\xymatrix@C=10pt{ & 0\ar[dl]_{0} \ar[dr]^{0} & \\
  0\ar[dr]_{0} & & 0\ar[dl]^{0} \\
& k & }
\]
It follows that $l(M) = 4$ with composition factors
$\{S_1,S_2,S_3,S_4\}$. 
\end{exam}

\begin{note}
The module $M$ has composition factors $\{S_1,S_2,S_3,S_4\}$. The
module $N = S_1\oplus S_2\oplus S_3\oplus S_4$ has the same
composition factors:
\[N = N_0 = S_1\oplus S_2\oplus S_3\oplus S_4 \supseteq N_1 = S_2\oplus
  S_3\oplus S_4 \supseteq N_2 = S_3\oplus S_4 \supseteq N_3 = S_4
  \supseteq N_4 = (0)\]
We have that $N_i/N_{i+1} \simeq S_{i+1}$ for $i = 0,1,2,3$.  So in
general, a module is not uniquely determined by its composition factors.
\end{note}